\documentclass[a4paper,12pt]{article}
 \usepackage{amsmath}
 \usepackage{amsthm}
 \usepackage{amsfonts}
 
 
 \usepackage[utf8]{inputenc}
 \usepackage[czech]{babel}
 \usepackage{hyperref}
 
 \setlength{\hoffset}{-3cm} 
 \setlength{\voffset}{-3cm}
 \setlength{\textheight}{26.0cm} 
 \setlength{\textwidth}{19cm}
 \setlength{\parindent}{0in}
 \setlength{\parskip}{0.1in}
 
 \begin{document}
     

\newtheorem*{dusledek}{Důsledek}
\newtheorem*{lemma}{Lemma}
\newtheorem*{veta}{Věta}
\newtheorem*{tvrzeni}{Tvrzení}

\def \emph#1{\underbar{#1}}
\def \Prob{\operatorname{Prob}}
\def \var{\operatorname{var}}

\title{Datové struktury - haldy a třídící algoritmy}
\date{}
\maketitle


\section{Úvod}

\flushpar V praxi se \v casto setk\'av\'ame s n\'asleduj\'\i c\'\i m 
probl\'emem, kter\'y vznik\'a na uspo\v r\'adan\'em univerzu, jeho\v z 
uspo\v r\'ad\'an\'\i\ se v\v sak v pr\accent23ub\v ehu \v casu m\v en\'\i . \'Uloha se 
li\v s\'\i\ od slovn\'\i kov\'eho probl\'emu v tom, \v ze se nevy\v zaduje  
efektivn\'\i\ operace {\bf MEMBER}. Dokonce se p\v redpokl\'ad\'a, \v ze ope\-race 
dostane spolu s argumentem informaci o ulo\v zen\'\i\ 
zpracov\'avan\'eho prvku. 
Hlav\-n\'\i m po\v zadavkem je rychlost provededn\'\i\ ostatn\'\i ch operac\'\i\ a mal\'e 
pam\v e\v tov\'e n\'aroky. P\v ritom v praxi obvykle nesta\v c\'\i\ zn\'at jen 
asymptotickou slo\v zitost, d\accent23ule\v zitou roli hraje skute\v cn\'a 
rychlost, kterou v\v sak neum\'\i me obecn\v e spo\v c\'\i tat, proto\v ze je z\'avisl\'a na 
pou\v zit\'em syst\'emu a hardwaru. P\v resto je p\v ri pou\v zit\'\i\ 
n\'asleduj\'\i c\'\i ch struktur dobr\'e m\'\i t realistickou p\v redstavu o 
skute\v cn\'ych rychlostech operac\'\i\ a podle toho si vybrat 
vhodnou strukturu. 
\medskip

\flushpar Zad\'an\'\i\ probl\'emu: Nech\v t $U$ je univerzum. Je d\'ana mno\v zina 
$S\subseteq U$ a funkce $f:S@>>>\Bbb R$, kde $\Bbb R$ jsou re\'aln\'a \v c\'\i sla (tato 
funkce realizuje uspo\v r\'ad\'an\'\i\ na univerzu $U$ -- pro $u,
v\in U$ plat\'\i\ 
$u\le v$, pr\'av\v e kdy\v z $f(u)\le f(v)$; zm\v ena uspo\v r\'ad\'an\'\i\ se pak realizuje 
zm\v enou funkce $f$).
M\'ame navrhnout reprezentaci $S$ a $f$, kter\'a umo\v z\v nuje 
operace:\newline 
{\bf INSERT$(s,a)$} -- p\v rid\'a k mno\v zin\v e $S$ prvek $s$ tak, \v ze 
$f(s)=a$,\newline 
{\bf MIN} -- nalezne prvek $s\in S$ s nejmen\v s\'\i\ hodnotou 
$f(s)$,\newline 
{\bf DELETEMIN} -- odstran\'\i\ prvek $s\in S$ s nejmen\v s\'\i\ hodnotou 
$f(s)$,\newline 
{\bf DELETE$(s)$} -- odstran\'\i\ prvek $s\in S$ z mno\v ziny $S$,\newline 
{\bf DECREASE$(s,a)$} -- zmen\v s\'\i\ hodnotu $f(s)$ o $a$ (tj. 
$f(s):=f(s)-a$),\newline 
{\bf INCREASE$(s,a)$} -- zv\v et\v s\'\i\ hodnotu $f(s)$ o $a$ (tj. 
$f(s):=f(s)+a$).\newline 
P\v ri operaci {\bf INSERT$(s,a)$} se p\v redpokl\'ad\'a, \v ze $
s\notin S$, a tento 
p\v redpoklad operace {\bf INSERT} neov\v e\v ruje. P\v ri operac\'\i ch {\bf DE\-LE\-TE$
(s)$},
{\bf DECREASE$(s,a)$ a INCREASE$(s,a)$} se p\v red\-pokl\'ad\'a, \v ze $
s\in S$, 
a operace nav\'\i c dost\'av\'a informaci, jak naj\'\i t  
prvek $s$ v reprezentaci $S$ a $f$. Haldy jsou typ 
struktury, kter\'a se pou\v z\'\i v\'a pro \v re\v sen\'\i\ tohoto probl\'emu. 
\medskip

\flushpar\emph{Halda} je stromov\'a struktura, kde vrcholy 
reprezentuj\'\i\ prvky z $S$ a spl\v nuj\'\i\ lok\'aln\'\i\ podm\'\i nku na 
$f$. Obvykle se pou\v z\'\i v\'a n\'asleduj\'\i c\'\i\ podm\'\i nka nebo jej\'\i\ 
du\'aln\'\i\ verze:
\bigskip
\roster
\item"{(usp)}"
Pro ka\v zd\'y vrchol $v$ plat\'\i : kdy\v z $v$ reprezentuje prvek 
$s\in S$ a $\otec(v)$ reprezentuje $t\in S$, pak $f(t)\le f(s)$.
\endroster
\medskip

\flushpar Probereme n\v ekolik verz\'\i\ hald a budeme 
p\v redpokl\'adat, \v ze v\v zdy spl\v nuj\'\i\ tuto podm\'\i nku a \v ze po\v zadavek na 
proveden\'\i\ operac\'\i\ {\bf DELETE$(s)$}, {\bf DECREA\-SE$(s,a
)$} a {\bf INCREASE$(s,a)$ }
tak\'e zad\'av\'a ukazatel na vrchol repre\-zentuj\'\i c\'\i\ $s\in 
S$.  Nav\'\i c budeme 
uva\v zovat operace
\medskip

\flushpar {\bf MAKEHEAP$(S,f)$} -- operace vytvo\v r\'\i\ haldu 
reprezentuj\'\i c\'\i\ mno\v zinu $S$ a funkci $f$,\newline 
{\bf MERGE$(H_1,H_2)$} -- p\v redpokl\'ad\'a, \v ze halda $H_i$ reprezentuje 
mno\v zinu $S_i$ a funkci $f_i$ pro $i=1,2$ a $S_1\cap S_2=\emptyset$. Operace 
vytvo\v r\'\i\ haldu $H$ reprezentuj\'\i c\'\i\ $S_1\cup S_2$ a $
f_1\cup f_2$, 
p\v ri\v cem\v z neov\v e\v ruje disjunktnost $S_1$ a $S_2$.
\newpage

\centerline{\bigbigrm{II. Regul\'arn\'\i\ haldy}}
\bigskip

\flushpar Prvn\'\i\ pou\v zit\'e haldy byly bin\'arn\'\i\ neboli $
2$-regul\'arn\'\i\ haldy. 
Tyto haldy jsou velmi obl\'\i ben\'e pro svou jednoduchost a 
n\'azornost a pro velmi efektivn\'\i\ implementaci.
\medskip

\flushpar P\v redpokl\'adejme, \v ze $d>1$ je p\v rirozen\'e \v c\'\i slo. 
$d$-\emph{regul\'arn\'\i} \emph{strom} je ko\v renov\'y strom 
$(T,r)$, pro kter\'y existuje po\v rad\'\i\ syn\accent23u 
jednotliv\'ych vnit\v rn\'\i ch vrchol\accent23u takov\'e, \v ze 
o\v c\'\i slov\'an\'\i\ vrchol\accent23u pro\-hle\-d\'a\-v\'an\'\i m do \v s\'\i\v rky 
(ko\v ren $r$ je \v c\'\i slov\'an $1$) spl\v nuje n\'asleduj\'\i c\'\i\ vlastnosti
\roster
\item
ka\v zd\'y vrchol m\'a nejv\'y\v se $d$ syn\accent23u,
\item
kdy\v z vrchol nen\'\i\ list, tak v\v sechny vrcholy s men\v s\'\i m 
\v c\'\i slem maj\'\i\ pr\'av\v e $d$ syn\accent23u,
\item
kdy\v z vrchol m\'a m\'en\v e ne\v z $d$ syn\accent23u, pak v\v sechny 
vrcholy s v\v et\v s\'\i mi \v c\'\i sly jsou listy.
\endroster
Toto o\v c\'\i slov\'an\'\i\ se naz\'yv\'a \emph{p\v rirozen\'e} 
\emph{o\v c\'\i slov\'an\'\i} $d$-regul\'arn\'\i ho stromu. 
\medskip

\proclaim{Tvrzen\'\i}Ka\v zd\'y $d$-regul\'arn\'\i\ strom m\'a nejv\'y\v se jeden 
vrchol, kter\'y nen\'\i\ list a m\'a m\'en\v e ne\v z $d$ syn\accent23u. Kdy\v z 
$d$-regul\'arn\'\i\ strom m\'a $n$ vrchol\accent23u, pak jeho 
v\'y\v ska je $\lceil\log_d(n(d-1)+1)\rceil$. Nech\v t $o$ je p\v rirozen\'e o\v c\'\i slov\'an\'\i\ 
vrchol\accent23u $d$-regul\'arn\'\i ho stromu. Kdy\v z pro vrchol 
$v$ je $o(v)=k$, pak vrchol $w$ je syn vrcholu $v$, pr\'av\v e 
kdy\v z $o(w)\in \{(k-1)d+2,(k-1)d+3,\dots,kd+1\}$, a vrchol $u$ je 
otcem vrcholu $v$, pr\'av\v e kdy\v z $o(u)=1+\lfloor\frac {k-2}d
\rfloor$. 
\endproclaim

\demo{D\accent23ukaz}Prvn\'\i\ \v c\'ast tvrzen\'\i\ plyne p\v r\'\i mo z po\v zadavku 2) na 
$d$-regul\'arn\'\i\ strom. M\'a-li $d$-regul\'arn\'\i\ strom v\'y\v sku $
k$, 
pak m\'a alespo\v n $\Sigma_{i=0}^{k-1}d^i+1$ a nejv\'y\v se $\Sigma_{
i=0}^kd^i$ 
vrchol\accent23u. Proto
$$\frac {d^k-1}{d-1}<n\le\frac {d^{k+1}}{d-1}\quad ,\quad d^k-1<n
(d-1)\le d^{k+1}-1$$
a zlogaritmov\'an\'\i m dostaneme
$$k<\log_d(n(d-1)+1)\le k+1.$$
Odtud plyne 
druh\'a \v c\'ast tvrzen\'\i .  T\v ret\'\i\ \v c\'ast pro \v c\'\i sla syn\accent23u dok\'a\v zeme 
indukc\'\i\ podle o\v c\'\i slov\'a\-n\'\i .  Synov\'e ko\v rene maj\'\i\ \v c\'\i sla 
$2,3,\dots$,$d+1$,  proto\v ze ko\v ren m\'a \v c\'\i slo $1$.  Kdy\v z tvrzen\'\i\ plat\'\i\ 
pro vrchol s \v c\'\i slem $k$, pak synov\'e vrcholu s \v c\'\i slem $
k+1$ 
maj\'\i\ \v c\'\i sla $kd+2,kd+3,\dots,kd+d+1$, co\v z odpov\'\i d\'a 
\v c\'\i sl\accent23um $(k+1-1)d+2,(k+1-1)d+3,\dots,(k+1)d+1$, a tedy 
tvrzen\'\i\ plat\'\i . Posledn\'\i\ \v c\'ast pak plyne z toho, \v ze kdy\v z $
i\in \{(k-1)d+2,(k-1)d+3,\dots,kd+1\}$, pak 
$1+\lfloor\frac {i-2}d\rfloor =k$. \qed
\enddemo

\flushpar V\v simn\v eme si, \v ze speci\'aln\v e pro $d=2$ maj\'\i\ synov\'e vrcholu s \v c\'\i slem $
k$ 
\v c\'\i sla $2k$ a $2k+1$ a otec vrcholu s \v c\'\i slem $k$ m\'a \v c\'\i slo 
$\lfloor\frac k2\rfloor$. Tedy pro $2$-regul\'arn\'\i\ stromy je p\v redpis pro nalezen\'\i\ 
syn\accent23u a otce zvl\'a\v st\v e jednoduch\'y.  
\medskip

\flushpar\v Rekneme, \v ze mno\v zina $S$ s funkc\'\i\ $f$ je reprezentov\'ana 
$d$-regul\'ar\-n\'\i\ haldou $H$, kde $H$ je $d$-regul\'arn\'\i\ strom $
(T,r)$, kdy\v z 
p\v ri\v razen\'\i\ prvk\accent23u mno\v ziny $S$ vrchol\accent23um stromu 
$T$ je bijekce spl\v nuj\'\i c\'\i\ podm\'\i nku \thetag{usp}. Toto p\v ri\v razen\'\i\ 
je realizov\'ano funkc\'\i\ $\key$, kter\'a vrcholu p\v ri\v razuje j\'\i m 
reprezentovan\'y prvek.
\medskip

\flushpar Defince $d$-regul\'arn\'\i ho stromu umo\v z\v nuje velmi efektivn\'\i\ 
implementace $d$-regul\'arn\'\i ch hald.  M\v ej\-me mno\v zinu $
S$ 
reprezentovanou $d$-regul\'arn\'\i\ haldou $H$ s p\v rirozen\'ym 
o\v c\'\i s\-lo\-v\'a\-n\'\i m $o$ $d$-regul\'arn\'\i\-ho stromu $
(T,r)$.  Pak haldu $H$ 
m\accent23u\v zeme reprezentovat polem $H[1..|S|]$, kde pro vrchol 
stromu $v$, pro kter\'y $o(v)=i$, je $H(i)=(\key(v),f(\key(v))$.  Algoritmy 
budeme popisovat pro stromy, proto\v ze je to n\'azorn\v ej\v s\'\i .  
P\v reformulovat je pro pole je snadn\'e (viz 
o\v c\'\i slov\'an\'\i\ syn\accent23u a otce vrcholu $v$).  Pro jednoduchost 
budeme pro vrchol $v$ ps\'at $f(v)$ m\'\i sto $f(\key(v))$, neboli $
f(v)$ 
bude ozna\v covat $f(s)$, kde $s$ je reprezentov\'an vrcholem $v$.  U 
$d$-regul\'arn\'\i ho stromu p\v redpo\-kl\'a\-d\'ame, \v ze zn\'ame p\v rirozen\'e 
o\v c\'\i slov\'an\'\i , a fr\'aze `posledn\'\i\ vrchol', `p\v red\-ch\'azej\'\i c\'\i\ vrchol' atd.  
se vztahuj\'\i\ k tomuto o\v c\'\i slov\'an\'\i .  
\bigskip

\head
Algoritmy
\endhead

\flushpar Pro $d$-regul\'arn\'\i\ haldy nen\'\i\ zn\'ama efektivn\'\i\ implementace operace 
{\bf MERGE}.  Efektivn\'\i\ implementace 
ostatn\'\i ch operac\'\i\ jsou zalo\v zeny na pomocn\'ych 
operac\'\i ch {\bf UP$(v)$} a {\bf DOWN$(v)$}.  Ope\-race {\bf UP$
(v)$} posunuje 
prvek $s$ reprezentovan\'y vrcholem $v$ sm\v erem ke ko\v reni, 
dokud vrchol reprezentuj\'\i c\'\i\ prvek $s$ nespl\v nu\-je podm\'\i nku 
\thetag{usp}.  Operace {\bf DOWN$(v)$} je symetrick\'a.  
\bigskip
 

{\bf UP$(v)$}:\newline 
{\bf while} $v$ nen\'\i\ ko\v ren a $f(v)<f(\otec(v))$ {\bf do\newline 
\phantom{{\rm ---}}}vym\v e\v n $\key(v)$ a $\key(\otec(v))$\newline 
\phantom{---}$v:=\otec(v)$\newline 
{\bf enddo
\bigskip

DOWN$(v)$}:\newline 
{\bf if} $v$ nen\'\i\ list {\bf then\newline 
\phantom{{{\rm ---}}}$w:=$}syn vrcholu $v$ reprezentuj\'\i c\'\i\ prvek s nejmen\v s\'\i\ 
hodnotou $f(w)$\newline 
\phantom{---}{\bf while} $f(w)<f(v)$ a $v$ nen\'\i\ list {\bf do\newline 
\phantom{{\rm ------}}}vym\v e\v n $\key(v)$ a $\key(w)$, $v:=w$\newline 
\phantom{------}$w:=$syn vrcholu $v$ reprezentuj\'\i c\'\i\ prvek s nejmen\v s\'\i\ 
hodnotou $f(w)$\newline 
\phantom{---}{\bf enddo\newline 
endif
\bigskip

INSERT$(s)$}:\newline 
$v:=$nov\'y posledn\'\i\ list, $\key(v):=s$, {\bf UP$(v)$
\bigskip

MIN}:\newline 
{\bf V\'ystup} $\key(\text{\rm ko\v ren}(T))$
\bigskip

{\bf DELETEMIN}:\newline 
$v:=$posledn\'\i\ list, $r:=$ko\v ren, $\key(r):=\key(v)$\newline 
odstra\v n $v$\newline 
{\bf DOWN$(r)$
\bigskip

DELETE$(s)$}:\newline 
$v:=$vrchol reprezentuj\'\i c\'\i\ $s$\newline 
$w:=$posledn\'\i\ list\newline 
$t:=\key(w),$ $\key(v):=t,$ odstra\v n $w$\newline 
{\bf if} $f(t)<f(s)$ {\bf then UP$(v)$ else DOWN$(v)$ endif
\bigskip

DECREASE$(s,a)$}:\newline 
$v:=$vrchol reprezentuj\'\i c\'\i\ $s$\newline 
$f(s):=f(s)-a$, {\bf UP$(v)$
\bigskip

INCREASE$(s,a)$}:\newline 
$v:=$vrchol reprezentuj\'\i c\'\i\ $s$\newline 
$f(s):=f(s)+a$, {\bf DOWN$(v)$
\bigskip

MAKEHEAP$(S,f)$}:\newline 
$T:=d$-regul\'arn\'\i\ strom s $|S|$ vrcholy\newline 
zvol libovolnou reprezentaci $S$ vrcholy stromu $T$\newline 
$v:=$posledn\'\i\ vrchol, kter\'y nen\'\i\ list\newline 
{\bf while} $v$ je vrchol $T$ {\bf do\newline 
\phantom{{\rm ---}}DOWN$(v)$\newline 
\phantom{{\rm ---}}$v:=$}vrchol p\v redch\'azej\'\i c\'\i\ 
vrcholu $v$\newline 
{\bf enddo
\bigskip

}\flushpar Ov\v e\v r\'\i me korektnost algoritm\accent23u.  Je z\v rejm\'e, \v ze 
pomocn\'e operace jsou korektn\'\i\ -- skon\v c\'\i , kdy\v z podm\'\i nku 
\thetag{usp} spl\v nuje prvek $s$, kter\'y byl p\accent23uvodn\v e 
reprezentov\'an vrcholem $v$. Korektnost operace {\bf MIN} plyne 
p\v r\'\i mo z podm\'\i nky \thetag{usp}, proto\v ze ko\v ren reprezentuje nejmen\v s\'\i\ 
prvek mno\v ziny $S$. U operace {\bf INSERT} je podm\'\i nka 
\thetag{usp} spln\v ena pro v\v sechny vrcholy s v\'yjimkou nov\v e 
vytvo\v ren\'eho listu a operace {\bf UP} zajist\'\i\ jej\'\i\ spln\v en\'\i .  P\v ri 
operaci {\bf DELETEMIN} je pod\-m\'\i n\-ka \thetag{usp} spln\v ena pro 
v\v sechny vrcholy s v\'yjimkou ko\v rene a v tomto p\v r\'\i pad\v e ope\-race {\bf DOWN} zajist\'\i\ 
jej\'\i\ spln\v en\'\i .  Po proveden\'\i\ operac\'\i\ {\bf DE\-LE\-TE$
(s)$}, {\bf DECREASE$(s,a)$ }
a {\bf INCREASE$(s,a)$} je pod\-m\'\i n\-ka \thetag{usp} spln\v ena pro 
v\v sechny vrcholy s v\'yjimkou vrcholu $v$ a jej\'\i\ spln\v en\'\i\ 
op\v et zajist\'\i\ operace {\bf UP }
resp. {\bf DOWN}.  Pro operaci {\bf MAKEHEAP }
budeme uva\v zovat du\'aln\'\i\ formulaci podm\'\i nky \thetag{usp}:  
\roster
\item"{(d-usp)}"
kdy\v z $s$ je prvek reprezentovan\'y vrcholem $v$, pak 
$f(s)\le f(t)$ pro v\v sechny prvky reprezentovan\'e syny 
vrcholu $v$. 
\endroster
Pokud ka\v zd\'y vrchol spl\v nuje podm\'\i nku \thetag{d-usp}, pak 
spl\v nuje i pod\-m\'\i nku \thetag{usp}. Z\v rejm\v e ka\v zd\'y list spl\v nuje 
podm\'\i nku \thetag{d-usp} a kdy\v z ope\-race {\bf MAKEHEAP }
provede proceduru {\bf DOWN$(v)$}, pak je podm\'\i nka 
\thetag{d-usp} spln\v ena pro v\v sechny vrcholy s 
\v c\'\i sly alespo\v n tak velk\'ymi jako je \v c\'\i slo $v$. 
Operace {\bf MAKEHEAP} kon\v c\'\i\ proveden\'\i m operace {\bf DOWN} na 
ko\v ren a odtud plyne jej\'\i\ korektnost.
\bigskip

\head
Slo\v zitost operac\'\i
\endhead

\flushpar Vypo\v cteme \v casovou slo\v zitost operac\'\i :  Jeden b\v eh cyklu 
v operaci {\bf UP} vy\v zaduje \v cas $O(1)$ a v operaci {\bf DOWN }\v cas $
O(d)$.  
Proto operace {\bf UP} v nejhor\v s\'\i m p\v r\'\i pad\v e vy\v zaduje \v cas $
O(\log_d|S|)$ 
a operace {\bf DOWN }\v cas $O(d\log_d|S|)$.  
Operace {\bf MIN} vy\v zaduje \v cas $O(1)$,  {\bf INSERT} a {\bf DECREASE }
vy\v zadu\-j\'\i\ \v cas $O(\log_d|S|)$ a  {\bf DELETEMIN}, {\bf DELETE} a 
{\bf INCREASE }\v cas $O(d\log_d|S|)$.
\medskip
\flushpar Haldu m\accent23u\v zeme 
vytvo\v rit iterac\'\i\ operace {\bf INSERT}, co\v z 
vy\v zaduje \v cas $O(|S|\log_d(|S|))$.  Uk\'a\v ze\-me, \v ze slo\v zitost operace 
{\bf MAKEHEAP} je men\v s\'\i , ale pro mal\'e haldy je v\'yhodn\v ej\v s\'\i\ 
prov\'ad\v et opakovan\v e operaci {\bf INSERT}.  Operace {\bf DOWN$
(v)$} na vrchol 
ve v\'y\v sce $h$ vy\v zaduje v nej\-hor\v s\'\i m p\v r\'\i pad\v e \v cas $
O(hd)$.  
Vrchol\accent23u v hloubce $i$ je nejv\'y\v se $d^i$.  
P\v red\-pok\-l\'a\-dejme, \v ze strom m\'a v\'y\v sku $k$, pak vrchol v 
hloubce $i$ m\'a v\'y\v sku nejv\'y\v se $k-i$.  Tedy operace {\bf MAKEHEAP }
vy\v zaduje \v cas $O(\sum_{i=0}^{k-1}d^i(k-i)d)=O(\sum_{i=0}^{k-
1}d^{i+1}(k-i))$.  Oz\-na\v c\-me 
$A=\sum_{i=0}^{k-1}d^{i+1}(k-i)$, pak 
$$\align dA-A=&\sum_{i=0}^{k-1}d^{i+2}(k-i)-\sum_{i=0}^{k-1}d^{i+
1}(k-i)=\sum_{i=2}^{k+1}d^i(k-i+2)-\sum_{i=1}^kd^i(k-i+1)=\\
&d^{k+1}+\sum_{i=2}^kd^i(k-i+2-k+i-1)-dk=d^{k+1}+\sum_{i=2}^kd^i-
dk=\\
&d^{k+1}+d^2\frac {d^{k-1}-1}{d-1}-dk.\endalign$$
Tedy $A=\frac {d^{k+1}}{d-1}+\frac {d^{k+1}-d^2}{(d-1)^2}-\frac {
dk}{d-1}$. Proto\v ze 
$k=\lceil\log_d(|S|(d-1)+1)\rceil$, 
dost\'av\'ame, \v ze $d^{k+1}\le d^2((d-1)|S|+1)$, a proto $A\le 
2d^2|S|$. Tedy 
{\bf MAKEHEAP} vy\v zaduje v nejhor\v s\'\i m p\v r\'\i pad\v e jen \v cas 
$O(d^2|S|)$. 
\bigskip

\head
Aplikace 
\endhead

\flushpar T\v r\'\i d\v en\'\i : prostou posloupnost 
\v c\'\i sel $x_1,x_2,\dots,x_n$ lze set\v r\'\i dit n\'asleduj\'\i c\'\i m 
algoritmem pou\v z\'\i\-vaj\'\i c\'\i m haldu ($f$ bude v tomto p\v r\'\i pad\v e 
identick\'a funkce).
\bigskip

$d${\bf -HEAPSORT$(x_1,x_2,\dots,x_n)$}:\newline 
{\bf MAKEHEAP$(\{x_i\mid i=1,2,\dots,n\},f)$\newline 
$i=1$\newline 
while} $i\le n$ {\bf do\newline 
\phantom{{\rm ---}}$y_i:=$MIN}, {\bf DELETEMIN}, $i:=i+1$\newline 
{\bf enddo\newline 
V\'ystup}: $y_1,y_2,\dots,y_n$
\bigskip


\flushpar Teoreticky lze uk\'azat, \v ze pou\v zit\'\i\ $d$-regul\'arn\'\i ch hald v 
algoritmu {\bf HEAPSORT} pro $d=3$ a $d=4$ je v\'yhodn\v ej\v s\'\i\ 
ne\v z $d=2$. Experimenty uk\'azaly, \v ze optim\'aln\'\i\ 
algoritmus pro posloupnosti d\'elek do 1 000 000 by m\v el  
pou\v z\'\i vat $d=6$ nebo $d=7$ (v experimentech byl m\v e\v ren 
skute\v cn\v e spot\v rebovan\'y \v cas, nikoli po\v cet porovn\'an\'\i\ a 
v\'ym\v en prvk\accent23u). 
Pro del\v s\'\i\ posloupnosti se optim\'aln\'\i\ 
hodnota $d$ m\accent23u\v ze zmen\v sit.
\bigskip

\flushpar Dal\v s\'\i m p\v r\'\i kladem je nalezen\'\i\ nejkrat\v s\'\i ch cest v grafu z dan\'eho 
bodu. \v Re\v sme n\'asleduj\'\i c\'\i\ \'ulohu: \newline 
Vstup: orientovan\'y ohodnocen\'y graf $(X,R,c)$, kde $c$ je funkce z $
R$ do mno\v ziny kladn\'ych 
re\'aln\'ych \v c\'\i sel, a vrchol $z\in X$.\newline 
\'Ukol: nal\'ezt pro ka\v zd\'y bod $x\in X$ d\'elku 
nejkrat\v s\'\i\ cestu ze $z$ do $x$, kde d\'elka cesty je sou\v cet 
$c$-ohodnocen\'\i\ hran na cest\v e.
\bigskip

{\bf Dijkstr\accent23uv algoritmus}:\newline 
$d(z):=0$, $U:=\{z\}$\newline 
{\bf for every} $x\in X\setminus \{z\}$ {\bf do} $d(x):=+\infty$ {\bf enddo\newline 
while} $U\ne\emptyset$ {\bf do\newline 
\phantom{{\rm ---}}}najdi vrchol $u\in U$ s nejmen\v s\'\i\ hodnotou $
d(u)$\newline 
\phantom{---}odstra\v n $u$ z $U$\newline 
\phantom{---}{\bf for every} $(u,v)\in R$ {\bf do \newline 
\phantom{{\rm ------}}if} $d(u)+c(u,v)<d(v)$ {\bf then\newline 
\phantom{{\rm ---------}}if} $d(v)=+\infty$ {\bf then} vlo\v z $v$ do $
U$ {\bf endif\newline 
\phantom{{\rm ---------}}$d(v):=d(u)+c(u,v)$\newline 
\phantom{{\rm ------}}endif\newline 
\phantom{{\rm ---}}enddo\newline 
enddo
\bigskip

}\flushpar Korektnost algoritmu je zalo\v zena na kombinatorick\'em 
lemmatu, kter\'e \v r\'\i k\'a, \v ze kdy\v z odstra\-\v nuje\-me z $
U$ prvek $x$ s 
nejmen\v s\'\i\ hodnotou $d(x)$, pak vzd\'alenost ze $z$ do $x$ je pr\'av\v e 
$d(x)$. Proto kdy\v z $U=\emptyset$, pak $d(x)$ jsou d\'elky nejkrat\v s\'\i ch cest ze 
$z$ do $x$ pro v\v sechna $x\in X$. Tedy pr\'ace s mno\v zinou $U$ vy\v zaduje 
nejv\'y\v se $|X|$ operac\'\i\ {\bf INSERT}, {\bf MIN} a {\bf DELETEMIN} a $
|R|$ operac\'\i\ 
{\bf DECREASE} a v\v zdy plat\'\i\ $|U|\le |X|$. Vypo\v cteme \v casovou slo\v zitost 
{\bf Dijkstrova algoritmu} za p\v redpokladu, \v ze $U$ reprezentujeme jako $
d$-regul\'arn\'\i\ 
haldu. Kdy\v z $d=2$, pak dost\'av\'ame, \v ze algoritmus vy\v zaduje \v cas 
$O(|X|\log(|X|)+|R|\log(|X|))$. Kdy\v z $d=\max\{2,\lfloor\frac {
|R|}{|X|}\rfloor \}$, pak algoritmus 
vy\v zaduje \v cas $O(|R|\log_d|X|)$. V p\v r\'\i pad\v e, \v ze $
(X,R)$ je hust\'y graf, 
tj. $|R|>|X|^{1+\varepsilon}$ pro $\varepsilon >0$, pak $\log_d|X
|=O(1)$ a algoritmus 
je line\'arn\'\i\ (tj. vy\v zaduje \v cas $O(|R|)$).
\bigskip
 
\centerline{\bigbigrm{III. Leftist haldy}}
\medskip

\flushpar Dal\v s\'\i m typem hald, se kter\'ymi se sezn\'am\'\i me, jsou lefist 
haldy (nezn\'ame vhodn\'y \v cesk\'y p\v reklad, proto z\accent23ust\'av\'ame 
u anglick\'eho n\'azvu). Je to velmi elegantn\'\i\ a jednoduch\'y typ 
hald. V\v sechny operace jsou stejn\v e jako u regul\'arn\'\i ch hald 
zalo\v zeny na 
dvou z\'akladn\'\i ch operac\'\i ch, z nich\v z v tomto p\v r\'\i pad\v e hlavn\'\i\ je {\bf MERGE} a  
druhou je {\bf DECREASE}. Pou\v zit\'\i\ {\bf MERGE }
p\v ri n\'avrhu jin\'ych operac\'\i\ je b\v e\v zn\'e i v dal\v s\'\i ch hald\'ach. 
Operace {\bf MERGE} vyu\v z\'\i v\'a speci\'aln\'\i ch vlastnost\'\i\ leftist 
hald a idea operace {\bf DECREASE} je stejn\'a jako ve 
Fibonacciho hald\'ach. Nejprve form\'aln\v e pop\'\i\v seme strukturu 
leftist hald. 
\medskip

\flushpar M\v ejme bin\'arn\'\i\ ko\v renov\'y strom $(T,r)$ (to znamen\'a, \v ze $
r$ je ko\v ren, 
ka\v zd\'y vrchol m\'a nejv\'y\v se dva syny a u ka\v zd\'eho syna 
v\'\i me, zda je to prav\'y nebo lev\'y syn). Pro vrchol $v$ 
ozna\v cme $\npl(v)$ d\'elku nejkrat\v s\'\i\ cesty z $v$ do vrcholu, kter\'y m\'a 
nejv\'y\v se jednoho syna, tak\v ze nap\v r. pro list $l$ plat\'\i\ $\npl
(l)=0$. 
\medskip

\flushpar M\v ejme $S\subseteq U$ a funkci $f:S@>>>\Bbb R$. Pak bin\'arn\'\i\ strom 
$(T,r)$ takov\'y, \v ze
\roster
\item
kdy\v z vrchol $v$ m\'a jen jednoho syna, pak je to lev\'y syn,
\item
kdy\v z vrchol $v$ m\'a dva syny, pak 
$$\npl(\text{\rm prav\'y syn }v)\le\npl(\text{\rm lev\'y syn }v),$$
\item
existuje jednozna\v cn\'e p\v ri\v razen\'\i\ prvk\accent23u $S$  
vrchol\accent23um $T$, kter\'e spl\v nuje podm\'\i nku \thetag{usp} 
(toto p\v ri\v razen\'\i\ 
je reprezentov\'ano funkc\'\i\ $\key$, kter\'a vrcholu $v$ p\v ri\v rad\'\i\ prvek 
z mno\v zi\-ny $S$ reprezentovan\'y vrcholem $v$)
\endroster
je \emph{leftist} \emph{halda} 
reprezentuj\'\i c\'\i\ mno\v zinu $S$ a funkci $f$. 
\medskip

\flushpar Struktura vrcholu $v$ v leftist hald\v e:\newline 
S vrcholem $v$ jsou 
spojeny ukazatel\'e $\otec(v)$, $\levy(v)$ a $\pravy(v)$ na otce a 
na lev\'eho a prav\'eho syna vrcholu $v$.  
Kdy\v z ukazatel nen\'\i\ definov\'an, pak p\'\i\v seme, \v ze jeho hodnota je 
$NIL$.  D\'ale jsou s vrcholem spojeny funkce\newline 
$\npl(v)$ -- prom\v enn\'a s hodnotou $\npl(v)$,\newline 
$\key(v)$ -- prvek reprezentovan\'y vrcholem $v$,\newline 
$f(v)$ -- prom\v enn\'a obsahuj\'\i c\'\i\ hodnotu $f(\key(v))$.
\medskip

\flushpar Uvedeme z\'akladn\'\i\ vlastnost leftist haldy, kter\'a 
umo\v z\v nuje efektivn\'\i\ implementace operac\'\i . 
Posloupnost vrchol\accent23u $v_0,v_1,\dots,v_k$ se naz\'yv\'a 
\emph{prav\'a} \emph{cesta} z vrcholu $v$, kdy\v z $v=v_0$, $v_{i
+1}$ 
je prav\'y syn $v_i$ pro ka\v zd\'e $i=0,1,\dots,k-1$ a $v_k$ nem\'a 
prav\'eho syna. Pak podstrom vrcholu $v$ do hloubky $k$ je 
\'upln\'y bin\'arn\'\i\ strom a m\'a tedy  
alespo\v n $2^{k+1}-1$ vrchol\accent23u. Proto plat\'\i\ 

\proclaim{Tvrzen\'\i}V leftist hald\v e je 
d\'elka prav\'e cesty z ka\v zd\'eho vrcholu $v$ nejv\'y\v se rovna 
$$\log(\text{\rm velikost podstromu ur\v cen\'eho vrcholem }v).$$
\endproclaim
\bigskip

\head
Algoritmy a slo\v zitost operac\'\i
\endhead

\flushpar Z\'akladn\'\i\ operac\'\i\ pro leftist haldy je {\bf MERGE}. Tato 
operace je definov\'ana rekurzivn\v e a hloubka rekurze je omezena 
pr\'av\v e d\'elkami prav\'ych cest.
\bigskip

{\bf MERGE$(T_1,T_2)$}:\newline 
{\bf if} $T_1=\emptyset$ {\bf then V\'ystup$=T_2$} konec {\bf endif\newline 
if} $T_2=\emptyset$ {\bf then V\'ystup$=T_1$} konec {\bf endif\newline 
if} $\key(\text{\rm ko\v ren }T_1)>\key(\text{\rm ko\v ren }T_2)$ {\bf then\newline 
\phantom{{\rm ---}}}zam\v e\v n $T_1$ a $T_2$\newline 
{\bf endif\newline 
$T':=$MERGE$(\text{{{\rm podstrom prav\'eho syna ko\v rene }}}T_1
,T_2)$\newline 
$\pravy(\text{\rm{{\rm ko\v ren }}}T_1):=\text{\rm{{\rm ko\v ren }}}
T'$\newline 
$\otec(\text{\rm {\rm ko\v ren }}T'):=\text{\rm {\rm ko\v ren }}T_
1$\newline 
if} $\npl(\pravy(\text{\rm ko\v ren }T_1))>\npl(\levy(\text{\rm ko\v ren }
T_1))$ {\bf then\newline 
\phantom{{\rm ---}}}vym\v e\v n lev\'eho a prav\'eho syna ko\v rene $
T_1$\newline 
{\bf endif\newline 
$\npl(\text{{\rm {\rm ko\v ren }}}T_1):=\npl(\pravy(\text{{\rm {\rm ko\v ren }}}
T_1)+1$
\bigskip

INSERT$(x)$}:\newline 
Vytvo\v r haldu $T_1$ reprezentuj\'\i c\'\i\ $\{x\}$\newline 
$\bold M\bold E\bold R\bold G\bold E(T,T_1)$
\bigskip

{\bf MIN}:\newline 
{\bf V\'ystup}: $\key(\text{\rm ko\v ren }T)$
\bigskip

{\bf DELETEMIN}:\newline 
$T_1:=$podstrom lev\'eho syna ko\v rene $T$\newline 
$T_2:=$podstrom prav\'eho syna ko\v rene $T$\newline 
{\bf MERGE$(T_1,T_2)$
\bigskip

MAKEHEAP$(S,f)$}:\newline 
$Q:=$pr\'azdn\'a fronta\newline 
{\bf for every} $s\in S$ {\bf do\newline 
\phantom{{\rm ---}}}vlo\v z leftist haldu $T_s$ reprezentuj\'\i c\'\i\ $
\{s\}$ do $Q$\newline 
{\bf enddo\newline 
while} $|Q|>1$ {\bf do\newline 
\phantom{{\rm ---}}}vezmi haldy $T_1$ a $T_2$ z vrcholu $Q$ 
(odstra\v n je)\newline 
\phantom{---}{\bf MERGE$(T_1,T_2)$} vlo\v z do $Q$\newline 
{\bf enddo
\bigskip


}\flushpar Vypo\v cteme \v casovou slo\v zitost p\v redchoz\'\i ch 
algoritm\accent23u. Ka\v zd\'y b\v eh 
algoritmu {\bf MERGE} (bez rekurzivn\'\i ho vol\'an\'\i ) vy\v zaduje \v cas 
$O(1)$. Po\v cet rekurzivn\'\i ch vol\'an\'\i\ je sou\v cet d\'elek prav\'ych 
cest, proto algoritmus {\bf MERGE} vy\v zaduje \v cas $O(\log(|S_
1|+|S_2|))$, 
kde $S_i$ je mno\v zina reprezentovan\'a haldou $T_i$ pro $i=1,2$.
Odtud d\'ale plyne, \v ze \v cas algoritm\accent23u {\bf INSERT} a 
{\bf DELETE\-MIN} je v 
nejhor\v s\'\i m p\v r\'\i pad\v e $O(\log(|S|))$. Operace {\bf MIN} vy\v zaduje 
\v cas $O(1)$. Pro odhad slo\v zitosti {\bf MA\-KEHEAP} budeme 
uva\v zovat, \v ze na za\v c\'atku algoritmu je na vrcholu fronty 
speci\'aln\'\i\ znak, kter\'y se jen p\v renese na konec fronty. 
Odhadneme \v cas, kter\'y spot\v rebuj\'\i\ {\bf while-}cykly mezi dv\v e\-ma 
p\v renesen\'\i mi speci\'aln\'\i ho znaku. P\v redpokl\'adejme, \v ze se 
spe\-ci\'al\-n\'\i\ znak p\v renesl po $k$-t\'e. V tomto okam\v ziku 
maj\'\i\ v\v sechny haldy ve front\v e a\v z na jednu velikost $2^{
k-1}$. 
Proto ve front\v e $Q$ je $\big\lceil \frac {|S|}{2^{k-1}}\big\rceil $ hald a jeliko\v z jedna operace 
{\bf MERGE} vy\v zadu\-je $O(k)$ \v casu, tak {\bf while}-cykly vy\v zaduj\'\i\ 
\v cas $O(k\frac {|S|}{2^{k-1}})$. M\accent23u\v zeme tedy shrnout, \v ze operace 
{\bf MAKEHEAP} pot\v rebuje \v cas 
$$O(\sum_{k=1}^{\infty}k\frac {|S|}{2^{k-1}})=O(|S|\sum_{k=1}^{\infty}\frac 
k{2^{k-1}})=O(|S|).$$
\v Rada $\sum_{k=1}^{\infty}$$\frac k{2^{k-1}}$ konverguje nap\v r. podle pod\'\i lov\'eho d'Alambertova 
krit\'eria a lze jednodu\v se spo\v c\'\i tat (nap\v r. stejnou metodou 
jako pro regul\'arn\'\i\ haldy), \v ze sou\v cet je $4$.
\medskip

\flushpar Implementace operac\'\i\ {\bf DECREASE} a {\bf INCREASE} pomoc\'\i\ 
operac\'\i\ {\bf UP} a {\bf DOWN} jako v $d$-regul\'arn\'\i ch hald\'ach nen\'\i\ 
efektivn\'\i , proto\v ze d\'elka cesty z ko\v rene do listu v leftist hald\v e 
m\accent23u\v ze b\'yt a\v z $|S|$.  Proto navrhneme pro tyto operace 
efektivn\v ej\v s\'\i\ algoritmus zalo\v zen\'y na jin\'em principu.  Tento princip 
je pak pou\v zit i pro Fibonacciho haldy.  \medskip

\flushpar Nejprve pop\'\i\v seme pomocnou operaci {\bf Oprav$(T,v
)$}, 
kter\'a vy\-tvo\-\v r\'\i\ lef\-tist haldu z bin\'arn\'\i ho stromu $
T'$ vznikl\'eho z 
leftist haldy $T$ odtr\v zen\'\i m podstromu s ko\v renem ve vrcholu $
v$.
\bigskip

{\bf Oprav$(T,v)$}:\newline 
$t:=\otec(v)$, $\npl(t):=0$\newline 
{\bf if} $\pravy(t)\ne v$ {\bf then} $\levy(t):=\pravy(t)$ {\bf endif\newline 
$\pravy(t):=NIL$\newline 
while} se zmen\v silo $\npl(t)$ a $t$ nen\'\i\ ko\v ren {\bf do\newline 
\phantom{{\rm ---}}$t:=\otec(t)$\newline 
\phantom{{\rm ---}}if} $\npl(\pravy(t))>\npl(\levy(t))$ {\bf then\newline 
\phantom{{\rm ------}}}vym\v e\v n $\levy(t)$ a $\pravy(t)$\newline 
\phantom{---}{\bf endif\newline 
\phantom{{\rm ---}}$\npl(t):=\npl(\pravy(t))+1$\newline 
enddo
\bigskip

}\flushpar Po proveden\'\i\ operace {\bf Oprav} maj\'\i\ v\v sechny vrcholy 
spr\'avn\'e \v c\'\i slo $\npl$ a podm\'\i nky kladen\'e na leftist 
haldu jsou spln\v eny. Tedy po proveden\'\i\ {\bf Oprav }
je $T$ op\v et leftist halda. Kdy\v z $t$ je posledn\'\i\ vrchol, u 
kter\'eho se zmen\v silo $\npl$, pak  v\v sechny vrcholy, kde se zmen\v silo 
$\npl$, tvo\v r\'\i\ pravou cestu z vrcholu $t$. To znamen\'a, \v ze 
{\bf while}-cyklus se prov\'ad\v el nejv\'y\v se $\log(|S|)$-kr\'at a ka\v zd\'y 
b\v eh {\bf while}-cyklu vy\v zadoval \v cas $O(1)$. Proto algoritmus 
{\bf Oprav} vy\v zaduje \v cas $O(\log(|S|))$.
\medskip

\flushpar Pop\'\i\v seme ostatn\'\i\ algoritmy.
\bigskip

{\bf DECREASE$(s,a)$}:\newline 
$v:=$prvek reprezentuj\'\i c\'\i\ $s$\newline 
$T_1$$:=$podstrom $T$ ur\v cen\'y vrcholem $v$, $f(v):=f(v)-a$\newline 
$T_2:=${\bf Oprav$(T,v)$}, $T:=${\bf MERGE$(T_1,T_2)$
\bigskip

INCREASE$(s,a)$}:\newline 
$v:=$prvek reprezentuj\'\i c\'\i\ $s$\newline 
$T_1:=$podstrom $T$ ur\v cen\'y vrcholem $\levy(v)$\newline 
$T_2:=$podstrom $T$ ur\v cen\'y  vrcholem $\pravy(v)$\newline 
$T_3:=$leftist halda reprezentuj\'\i c\'\i\ prvek $s$\newline 
$f(v):=f(v)+a$, $T_4:=${\bf Oprav$(T,v)$}, $T_1:=${\bf MERGE$(T_1
,T_3)$\newline 
$T_2:=$MERGE$(T_2,T_4)$}, $T:=${\bf MERGE$(T_1,T_2)$
\bigskip

DELETE$(s,a)$}:\newline 
$v:=$prvek reprezentuj\'\i c\'\i\ $s$\newline 
$T_1:=$podstrom $T$ ur\v cen\'y vrcholem $\levy(v)$\newline 
$T_2:=$podstrom $T$ ur\v cen\'y  vrcholem $\pravy(v)$\newline 
$T_3:=${\bf MERGE$(T_1,T_2)$}, $T_4:=${\bf Oprav$(T,v)$\newline 
$T:=$MERGE$(T_3,T_4)$
\bigskip

}\flushpar Proto\v ze algoritmy {\bf MERGE} a {\bf Oprav} vy\v zaduj\'\i\ \v cas 
$O(\log(|S|)$ a proto\v ze zbyl\'e \v c\'asti algoritm\accent23u pro operace 
{\bf DECREASE}, {\bf INCREASE} a {\bf DELETE} vy\v zaduj\'\i\ $O(
1)$ \v casu, 
m\accent23u\v zeme shrnout v\'ysledky:  
\medskip

\proclaim{V\v eta}V leftist hald\'ach existuje implementace operace 
{\bf MIN}, kter\'a v nejhor\v s\'\i m p\v r\'\i pad\v e vy\v zaduje \v cas $
O(1)$, 
implementace ope\-rac\'\i\ {\bf INSERT}, {\bf DELETEMIN}, {\bf DELETE}, 
{\bf MER\-GE}, {\bf DECREA\-SE} a {\bf INCREA\-SE}, kter\'e vy\v zaduj\'\i\ v nejhor\v s\'\i m 
p\v r\'\i\-pa\-d\v e \v cas $O(\log(|S|))$, a implementace operace {\bf MAKE\-HEAP},
kter\'a vy\v zaduje \v cas $O(|S|)$, kde $S$ je reprezentovan\'a mno\v zina.
\endproclaim
\bigskip

\centerline{\bigbigrm{IV. Amortizovan\'a slo\v zitost}}
\medskip

\flushpar Pop\'\i\v seme bankovn\'\i\ paradigma pro po\v c\'\i t\'an\'\i\ s 
amortizovanou slo\v zitost\'\i . P\v redpokl\'adejme, \v ze m\'ame funkci $
h$, 
kter\'a ohodnucuje konfigurace a kvantitativn\v e vystihuje 
jejich vhodnost pro proveden\'\i\ operace $o$. 
Kdy\v z na konfiguraci $D$ aplikujeme operaci $o$ 
a dostaneme konfiguraci $D'$, pak amortizovan\'a slo\v zitost 
$am(o)$ ope\-race $o$ m\'a vystihovat nejen \v casovou n\'aro\v cnost 
operace, ale i to, jak se zm\v enila vhodnost konfigurace pro 
tuto operaci. Proto ji defi\-nujme jako $am(o)=t(o)+h(D')-h(D)$, kde $
t(o)$ je 
\v cas pot\v rebn\'y pro proveden\'\i\ operace $o$. P\v redpokl\'adejme, 
\v ze chceme prov\'est posloupnost operac\'\i\ $o_1,o_2,\dots,o_n$ na 
konfiguraci $D_0$. Zn\'azorn\'\i me si to takto:
$$D_0@>{o_1}>>D_1@>{o_2}>>D_2@>{o_3}>>\dots@>{o_n}>>D_n.$$
P\v redpokl\'adejme, \v ze pro ka\v zd\'e $i=1,2,\dots,n$ m\'ame odhad 
$c(o_i)$ amortizovan\'e slo\v zitos\-ti operace $o_i$, tj. $am(o_
i)\le c(o_i)$ 
pro v\v sechna $i=1,2,\dots,n$. Pak
$$\sum_{i=1}^nam(o_i)=\sum_{i=1}^n\big(t(o_i)+h(D_i)-h(D_{i-1})\big
)=h(D_n)-h(D_0)+\sum_{i=1}^nt(o_i)\le\sum_{i=1}^nc(o_i).$$
Z toho plyne, \v ze 
$$\sum_{i=1}^nt(o_i)\le\sum_{i=1}^nc(o_i)-h(D_n)+h(D_0).$$
\medskip

\flushpar Obvykle je $h(D)\ge 0$ pro v\v sechny konfigurace $D$ 
nebo naopak $h(D)\le 0$ 
pro v\v sechny konfigurace $D$. Kdy\v z $h(D)\ge 0$, pak 
$$\sum_{i=1}^nt(o_i)\le\sum_{i=1}^nc(o_i)+h(D_0),$$
kdy\v z $h(D)\le 0$, pak 
$$\sum_{i=1}^nt(o_i)\le\sum_{i=1}^nc(o_i)-h(D_n).$$
To znamen\'a, \v ze odhad amortizovan\'e slo\v zitosti d\'av\'a tak\'e odhad 
na \v casovou slo\v zitost posloupnosti operac\'\i , kter\'y b\'yv\'a lep\v s\'\i\ ne\v z odhad slo\v zitosti v 
nejhor\v s\'\i m p\v r\'\i pad\v e.  Tato skute\v cnost vysv\v etluje \v radu 
p\v r\'\i pad\accent23u, kdy 
v\'ysledky byly lep\v s\'\i\ ne\v z teoretick\'y v\'ypo\v cet.  Ukazuje se, \v ze 
slo\v zitost posloupnosti operac\'\i\ v nejhor\v s\'\i m p\v r\'\i pad\v e je \v casto podstatn\v e 
men\v s\'\i\ ne\v z sou\v cet slo\v zitost\'\i\ v nejhor\v s\'\i m p\v r\'\i pad\v e pro 
jednotliv\'e operace.  
\bigskip

\centerline{\bigbigrm{V. Binomi\'aln\'\i\ haldy}}
\medskip

\flushpar Dal\v s\'\i\ typ hald je motivov\'an s\v c\'\i tan\'\i m p\v rirozen\'ych \v c\'\i sel.  
Binomi\'aln\'\i\ halda reprezentuj\'\i c\'\i\ $n-$prvkovou mno\v zinu se 
toti\v z chov\'a 
podobn\v e jako \v c\'\i slo $n$.  Tento typ hald je tak\'e po zobecn\v en\'\i\ v jist\'em 
smyslu vzorem pro Fibonacciho haldy.  
\medskip

\flushpar Pro $i=0,1,\dots$ definujeme rekurentn\v e binomi\'aln\'\i\ 
stromy $H_i$.  Jsou to ko\v renov\'e stromy takov\'e, \v ze $H_0$ je 
jednoprvkov\'y strom a  strom $H_{i+1}$ vznikne ze dvou 
disjunktn\'\i ch strom\accent23u $H_i$, kde ko\v ren jednoho stromu  
se stane dal\v s\'\i m synem (nejlev\v ej\v s\'\i m nebo nejprav\v ej\v s\'\i m) ko\v rene druh\'eho stromu.  Viz 
Obr.  1. 
\medskip

\midinsert
\centerline{\input fig10.tex}
\botcaption{Obr. 1}
\endcaption
\endinsert

\flushpar Nejprve uvedeme z\'akladn\'\i\ vlastnosti t\v echto 
strom\accent23u.
\medskip

\proclaim{Tvrzen\'\i}Pro ka\v zd\'e p\v rirozen\'e \v c\'\i slo $
i=0,1,\dots$ 
plat\'\i :
\roster
\item
strom $H_i$ m\'a $2^i$ vrchol\accent23u,
\item
ko\v ren stromu $H_i$ m\'a $i$ syn\accent23u,
\item
d\'elka nejdel\v s\'\i\ cesty z ko\v rene do listu ve stromu $H_i$ je 
$i$ (tj. v\'y\v ska $H_i$ je $i$),
\item
podstromy ur\v cen\'e syny ko\v rene stromu $H_i$ jsou izomorf\-n\'\i\ 
se stromy $H_0,H_1,\dots,H_{i-1}$.
\endroster
\endproclaim

\demo{D\accent23ukaz}Tvrzen\'\i\ plat\'\i\ pro strom $H_0$ a 
jednoduchou indukc\'\i\ se dok\'a\v ze i pro dal\v s\'\i\ stromy. 
Skute\v cn\v e, kdy\v z $H_i$ m\'a $2^i$ vrchol\accent23u, pak $H_{
i+1}$ m\'a 
$2(2^i)=2^{i+1}$ vrchol\accent23u. Ko\v ren stromu $H_{i+1}$ m\'a o jednoho 
syna v\'\i ce ne\v z ko\v ren stromu $H_i$ a nejdel\v s\'\i\ cesta do listu je o $
1$ 
del\v s\'\i . Proto\v ze podstrom syna, kter\'y p\v ribyl ko\v reni stromu 
$H_{i+1}$, je izomorfn\'\i\ s $H_i$ a jinak se nic nem\v enilo, je 
d\accent23ukaz kompletn\'\i . \qed
\enddemo
\medskip

\flushpar\emph{Binomi\'aln\'\i} \emph{halda} $\Cal H$ reprezentuj\'\i c\'\i\ 
mno\v zinu $S$ je soubor (seznam) strom\accent23u $\{T_1,T_2,\dots
,T_k\}$ 
takov\'y, \v ze 
\roster
\item"{}"
celkov\'y po\v cet vrchol\accent23u v t\v echto stromech je 
roven velikosti $S$ a 
existuje a je d\'ano jednozna\v cn\'e p\v ri\v razen\'\i\ prvk\accent23u 
z $S$ vrchol\accent23um strom\accent23u takov\'e, \v ze plat\'\i\ 
podm\'\i nka \thetag{usp} -- toto p\v ri\v razen\'\i\ je realizov\'ano 
funkc\'\i\ 
$\key,$ kter\'a vrcholu stromu p\v ri\v razuje prvek j\'\i m reprezentovan\'y;
\item"{}"
ka\v zd\'y strom $T_i$ je izomorfn\'\i\ s n\v ejak\'ym stromem $H_
j$;
\item"{}"
$T_i$ nen\'\i\ izomorfn\'\i\ s \v z\'adn\'ym $T_j$ pro $i\ne j$.
\endroster
\medskip

\flushpar Z bin\'arn\'\i ho z\'apisu p\v rirozen\'ych \v c\'\i sel plyne, \v ze 
pro ka\v zd\'e p\v riro\-ze\-n\'e \v c\'\i slo $n>0$ existuje prost\'a 
posloupnost $i_1,i_2,\dots,i_k$ p\v riro\-ze\-n\'ych \v c\'\i sel takov\'a, \v ze 
$n=\sum_{j=1}^k2^{i_j}$. Z toho plyne, \v ze pro ka\v zdou nepr\'azd\-nou 
mno\v zinu $S$ existuje binomi\'aln\'\i\ halda repre\-zentuj\'\i c\'\i\ $
S$.
Tato halda obsahuje strom izomorfn\'\i\ s $H_i$, pr\'av\v e kdy\v z v 
bin\'arn\'\i m z\'apise \v c\'\i sla $|S|$ je na $i$-t\'em m\'\i st\v e zprava $
1$.
\bigskip

\head
Algoritmy a slo\v zitost operac\'\i
\endhead

\flushpar Operace pro binomi\'aln\'\i\ haldy jsou  
stejn\v e jako pro leftist haldy zalo\v zeny na ope\-raci {\bf MER\-GE}. Ope\-race 
{\bf MERGE} pro binomi\'aln\'\i\ haldy je analogi\'\i\ s\v c\'\i t\'an\'\i\ 
p\v rirozen\'ych \v c\'\i sel v bin\'ar\-n\'\i m z\'apise.
\bigskip

{\bf MERGE$(\Cal H_1,\Cal H_2)$}:\newline 
(koment\'a\v r: $\Cal H_i$ reprezentuje mno\v zinu $S_i$ pro $i=1
,2$ a $S_1\cap S_2=\emptyset$)\newline 
$i:=0$, $T:=$pr\'azdn\'y strom, $\Cal H:=\emptyset$\newline 
{\bf while} $i<\log(|S_1|+|S_2|)$ {\bf do\newline 
\phantom{{\rm ---}}if} existuje $U\in \Cal H_1$ izomorfn\'\i\ s $
H_i$ {\bf then\newline 
\phantom{{\rm ------}}$U_1:=U$\newline 
\phantom{{\rm ---}}else\newline 
\phantom{{\rm ------}}$U_1:=$}pr\'azdn\'y strom\newline 
\phantom{---}{\bf endif\newline 
\phantom{{\rm ---}}if} existuje $U\in \Cal H_2$ izomorfn\'\i\ s $
H_i$ {\bf then\newline 
\phantom{{\rm ------}}$U_2:=U$\newline 
\phantom{{\rm ---}}else\newline 
\phantom{{\rm ------}}$U_2:=$}pr\'azdn\'y strom\newline 
\phantom{---}{\bf endif\newline 
\phantom{{\rm ---}}case\newline 
\phantom{{\rm ------}}}(existuje pr\'av\v e jeden nepr\'azdn\'y strom 
$V\in \{T,U_1,U_2\}$) {\bf do}:\newline 
\phantom{---------}vlo\v z $V$ do $\Cal H$, $T:=$pr\'azdn\'y strom\newline 
\phantom{------}(existuj\'\i\ pr\'av\v e dva nepr\'azdn\'e stromy 
$V_1,V_2\in \{T,U_1,U_2\})$ {\bf do}:\newline 
\phantom{---------}$T:=${\bf spoj$(V_1,V_2)$\newline 
\phantom{{\rm ------}}}(v\v sechny stromy $T$, $U_1$ a $U_2$ jsou nepr\'azdn\'e) {\bf do}:\newline 
\phantom{---------}vlo\v z $T$ do $\Cal H$, $T:=${\bf spoj$(U_1,U_
2)$\newline 
\phantom{{\rm ---}}endcase\newline 
\phantom{{\rm ---}}$i:=i+1$\newline 
enddo\newline 
if} $T\ne$pr\'azdn\'y strom {\bf then} vlo\v z $T$ do $\Cal H$ {\bf endif\newline 
V\'ystup}:$\Cal H$
\bigskip

{\bf spoj$(T_1,T_2)$}:\newline 
{\bf if} $f(\text{\rm ko\v ren }T_1)>f(\text{\rm ko\v ren }T_2)$ {\bf then\newline 
\phantom{{\rm ---}}}vym\v e\v n stromy $T_1$ a $T_2$\newline 
{\bf endif\newline 
}
vytvo\v r nov\'eho  syna $v$ ko\v rene $T_1$\newline 
$v:=$ko\v ren $T_2$
\bigskip

\flushpar Je vid\v et, \v ze kdy\v z oba stromy $T_1$ a $T_2$ jsou izomorfn\'\i\ 
s $H_i$, pak v\'ysledn\'y strom operace {\bf spoj} je izomorfn\'\i\ 
s $H_{i+1}$. Korektnost ope\-race {\bf MERGE} plyne z tohoto 
pozorov\'an\'\i\ a z faktu, \v ze $\Cal H_j$ obsahuje strom izomorfn\'\i\ s 
$H_i$, pr\'av\v e kdy\v z v bin\'arn\'\i m z\'apise \v c\'\i sla $
|S_j|$ je na $i$-t\'em 
m\'\i st\v e zprava $1$, a \v ze $T$ je nepr\'azdn\'y strom, kdy\v z se 
prov\'ad\'\i\ posun \v r\'adu p\v ri s\v c\'\i t\'an\'\i . Proto\v ze ka\v zd\'y b\v eh 
cyklu vy\v zaduje \v cas $O(1)$, algoritmus {\bf MER\-GE} vy\v zaduje \v cas 
$O(\log(|S_1|+|S_2|))$. Implementace dal\v s\'\i ch algoritm\accent23u 
je podobn\'a jako pro leftist haldy. 
\bigskip

{\bf INSERT$(x)$}:\newline 
Vytvo\v r haldu $\Cal H_1$ reprezentuj\'\i c\'\i\ $\{x\}$\newline 
$\bold M\bold E\bold R\bold G\bold E(\Cal H,\Cal H_1)$
\bigskip

{\bf MIN}:\newline 
Prohledej prvky reprezentovan\'e ko\v reny v\v sech strom\accent23u 
v $\Cal H$\newline 
{\bf V\'ystup}: nejmen\v s\'\i\ z t\v echto prvk\accent23u  
\bigskip

{\bf DELETEMIN}:\newline 
Prohledej prvky reprezentovan\'e ko\v reny v\v sech strom\accent23u 
v $\Cal H$\newline 
$T:=$ strom, jeho\v z ko\v ren reprezentuje nejmen\v s\'\i\ prvek\newline 
$\Cal H_1:=\Cal H\setminus \{T\}$\newline 
$\Cal H_2:=$ halda tvo\v ren\'a podstromy $T$ ur\v cen\'ymi syny ko\v rene $
T$\newline 
{\bf MERGE$(\Cal H_1,\Cal H_2)$
\bigskip

}\flushpar Z podm\'\i nky \thetag{usp} je z\v rejm\'e, \v ze 
nejmen\v s\'\i\ prvek v $S$ je reprezentov\'an v ko\v reni n\v ejak\'eho  
stromu haldy. T\'\i m je d\'ana korekt\-nost operace {\bf MIN}.
Z \'uvodn\'\i ho tvrzen\'\i\ plyne, \v ze $\Cal H_2$ v operaci {\bf DELE\-TE\-MIN} je 
binomi\'aln\'\i\ halda, a odtud plyne korekt\-nost ope\-race 
{\bf DE\-LE\-TEMIN}. Operace {\bf DECREASE} se implementuje pomoc\'\i\ 
ope\-race {\bf UP} a ope\-race {\bf INCREASE} pomoc\'\i\ 
operace {\bf DOWN} stejn\v e jako v regul\'arn\'\i ch hald\'ach. 
Struktura binomi\'aln\'\i\ haldy nepodporuje p\v r\'\i mo operaci 
{\bf DELETE} -- ta se d\'a realizovat jedin\v e jako posloupnost 
operac\'\i\ {\bf DECREASE$(s,\infty )$} a {\bf DELETEMIN}.  
Operace {\bf MAKEHEAP} se prov\'ad\'\i\ opakov\'an\'\i m operace {\bf INSERT}.
\medskip

\flushpar V\'ypo\v cet \v casov\'e slo\v zitosti operac\'\i\ pro binomi\'aln\'\i\ haldy 
vyu\v z\'\i v\'a n\v ekolik zn\'am\'ych fakt\accent23u.  Ope\-race {\bf MERGE }
simuluje s\v c\'\i t\'an\'\i\ p\v rirozen\'ych \v c\'\i sel v bin\'arn\'\i m z\'apise a m\'a 
tedy stejnou slo\v zitost.  Odhad slo\v zitosti vytv\'a\v ren\'\i\ haldy vyu\v z\'\i v\'a 
zn\'am\'eho faktu, \v ze amortizovan\'a slo\v zitost p\v ri\v c\'\i t\'an\'\i\ $
1$  
k bin\'arn\'\i mu \v c\'\i slu je $O(1)$.  
Odhad slo\v zitosti operac\'\i\ {\bf MIN} a {\bf DELETEMIN} je zalo\v zen na 
pozorov\'an\'\i , \v ze binomi\'aln\'\i\ halda reprezentuj\'\i c\'\i\ mno\v zinu $
S$ m\'a 
tolik strom\accent23u, kolik je jedni\v cek v bin\'arn\'\i m z\'apise $
|S|$, a 
to je nejv\'y\v se $\log(|S|)$.  
\medskip

\flushpar Z tvrzen\'\i\ tak\'e plyne, \v ze v\'y\v ska v\v sech strom\accent23u v 
binomi\'aln\'\i\ hald\v e je $\le\log(|S|)$ a po\v cet syn\accent23u 
ko\v rene ka\v zd\'eho stromu je 
tak\'e $\le\log(|S|)$, p\v ri\v cem\v z tento odhad se ned\'a zlep\v sit. Odtud 
dost\'av\'ame slo\v zitost operac\'\i\ {\bf DECREASE} a {\bf INCREASE} v nejhor\v s\'\i m 
p\v r\'\i pad\v e. M\accent23u\v zeme tedy shrnout:
\medskip


\proclaim{V\v eta}Pro binomi\'aln\'\i\ haldy algoritmy operac\'\i\ 
{\bf IN\-SERT}, {\bf MIN}, {\bf DELETEMIN}, {\bf DECREASE} a {\bf MERGE  }
vy\-\v za\-duj\'\i\ \v cas $O(\log(|S|))$, algoritmus operace {\bf INCRE\-ASE }
vy\-\v za\-duje \v cas $O(\log^2(|S|))$ a algoritmus operace {\bf MAKEHEAP 
}\v cas $O(|S|)$.
\endproclaim
\medskip


\flushpar Z t\v echto v\'ysledk\accent23u je vid\v et, \v ze p\v redchoz\'\i\ 
typy hald maj\'\i\ efektivn\v ej\v s\'\i\ chov\'an\'\i\ ne\v z binomi\'aln\'\i\ haldy. 
V\'yznam binomi\'aln\'\i ch hald tak spo\-\v c\'\i\-v\'a p\v redev\v s\'\i m v 
tom, \v ze se daj\'\i\ d\'ale zobecnit (t\'\i mto zobecn\v en\'\i m jsou Fibonacciho 
haldy) a \v ze na nich lze kr\'asn\v e 
ilustrovat princip, \v ze s \v radou \'uprav je v\'yhodn\'e po\v ckat 
a neprov\'ad\v et je okam\v zit\v e.
\bigskip

\head
L\'\i n\'a implentace operac\'\i
\endhead

\flushpar N\'asleduj\'\i c\'\i\ algoritmy jsou zalo\v zeny na ideji, \v ze 
`vyva\v zov\'an\'\i ' sta\v c\'\i\ prov\'ad\v et jen p\v ri operac\'\i ch {\bf MIN} a 
{\bf DELETEMIN}, kdy je stejn\v e zapot\v reb\'\i\ prohledat v\v sechny 
stromy. Z tohoto d\accent23uvodu zeslab\'\i me podm\'\i nky na 
binomi\'aln\'\i\ haldy.
\medskip

\flushpar\emph{L\'\i n\'a} \emph{binomi\'aln\'\i} \emph{halda} $\Cal H$ 
reprezentuj\'\i c\'\i\ mno\v zinu $S$ je seznam strom\accent23u 
$\{T_1,T_2,\dots,T_k\}$ takov\'y, \v ze  
\roster
\item"{}"
celkov\'y po\v cet vrchol\accent23u v t\v echto stromech je 
roven velikosti $S$ a 
existuje jednozna\v cn\'e p\v ri\v razen\'\i\ prvk\accent23u mno\v ziny 
$S$ vrchol\accent23um strom\accent23u, kter\'e spl\v nuje 
podm\'\i nku \thetag{usp} -- toto p\v ri\v razen\'\i\ je jako obvykle 
realizov\'ano funkc\'\i\ $\key$;
\item"{}"
ka\v zd\'y strom $T_i$ je izomorfn\'\i\ s n\v ejak\'ym stromem $H_
j$.
\endroster
\medskip

\flushpar V l\'\i n\'e binomi\'aln\'\i\ hald\v e je vynech\'an p\v redpoklad 
neizo\-morf\-nosti strom\accent23u tvo\v r\'\i c\'\i ch haldu. Tento fakt se 
projev\'\i\ ve velmi jednoduch\'em algoritmu pro operaci 
{\bf MERGE}.
\bigskip

{\bf MERGE$(\Cal H_1,\Cal H_2)$}:\newline 
Prove\v d konkatenaci seznam\accent23u $\Cal H_1$ a $\Cal H_2$
\bigskip

\flushpar Samotn\'y algoritmus pro operaci {\bf INSERT} se nezm\v en\'\i , jen 
provede tuto implementaci ope\-race {\bf MER\-GE}.  Ope\-race 
{\bf MIN} a {\bf DELETE\-MIN} pou\v zij\'\i\ n\'asleduj\'\i c\'\i\ pomocnou 
proceduru {\bf vyvaz}.  Jej\'\i m vstu\-pem je soubor 
seznam\accent23u $\{O_i\mid i=0,1,\dots,k\}$, kde seznam $O_i$ 
obsahuje jen stromy izomorfn\'\i\ se stromem $H_i$.  
Procedura {\bf vyvaz} pak z t\v echto strom\accent23u vytvo\v r\'\i\ 
klasickou binomi\'aln\'\i\ hal\-du. 
\bigskip

{\bf vyvaz$(\{O_i\mid i=0,1,\dots,k\})$}:\newline 
$i:=0$, $\Cal H:=\emptyset$\newline 
{\bf while} existuje $O_i\ne\emptyset$ {\bf do\newline 
\phantom{{\rm ---}}while} $|O_i|>1$ {\bf do\newline 
\phantom{{\rm ------}}}vezmi dva r\accent23uzn\'e stromy $T_1$ a $
T_2$ z 
$O_i$\newline 
\phantom{------}odstra\v n je z $O_i$\newline 
\phantom{------}{\bf spoj$(T_1,T_2)$} vlo\v z do $O_{i+1}$\newline 
\phantom{---}{\bf enddo\newline 
\phantom{{\rm ---}}if} $O_i\ne\emptyset$ {\bf then\newline 
\phantom{{\rm ------}}}strom $T\in O_i$ odstra\v n z $O_i$ a vlo\v z do $
\Cal H$\newline 
\phantom{---}{\bf endif},\newline 
\phantom{---}$i:=i+1$\newline 
{\bf enddo\newline 
V\'ystup}: $\Cal H$
\bigskip

{\bf MIN}:\newline 
Prohledej prvky reprezentovan\'e ko\v reny v\v sech 
strom\accent23u  
v $\Cal H$\newline 
{\bf V\'ystup}: nejmen\v s\'\i\ z t\v echto prvk\accent23u\newline 
stromy rozd\v el do mno\v zin $O_i=\{$v\v sechny stromy v 
$\Cal H$ izomorfn\'\i\ s $H_i\}$\newline 
{\bf vyvaz$(\{O_i\mid i=0,1,\dots,\lfloor\log(|S|)\rfloor \})$
\bigskip

DELETEMIN}:\newline 
Prohledej prvky reprezentovan\'e ko\v reny v\v sech 
strom\accent23u v $\Cal H$\newline 
$T:=$ strom, jeho\v z ko\v ren repre\-zentuje nejmen\v s\'\i\ 
prvek\newline 
stromy rozd\v el do mno\v zin $O_i=\{$v\v sechny stromy v 
$\Cal H$ izomorfn\'\i\ s $H_i$ r\accent23uzn\'e od $T\}\cup \{$podstrom $
T$ ur\v cen\'y n\v ejak\'ym synem ko\v rene $T$ 
izomorfn\'\i\ s $H_i\}$\newline 
{\bf vyvaz$(\{O_i\mid i=0,1,\dots,\lfloor\log(|S|)\rfloor \})$
\bigskip

\flushpar}\v Casov\'a slo\v zitost operac\'\i\ {\bf INSERT} a {\bf MER\-GE} p\v ri l\'\i n\'e 
implementaci je $O(1)$, ale \v casov\'a slo\v zitost operac\'\i\ {\bf MIN} a 
{\bf DELETEMIN} je v nejhor\v s\'\i m p\v r\'\i pad\v e $O(|S|)$. Tento odhad je 
velmi \v spatn\'y, ale uk\'a\v zeme, \v ze amortizovan\'a slo\v zitost m\'a rozumn\'e 
hodnoty. 
P\v ripom\'\i n\'ame, \v ze amortizovan\'a slo\v zi\-tost je \v cas operace plus 
ohodnocen\'\i\ v\'ysledn\'e struktury minus ohodnocen\'\i\ 
po\v c\'ate\v cn\'\i\ struktury. 
Konfiguraci ohodnot\'\i me po\v ctem strom\accent23u v 
hald\v e. 
Proto\v ze ope\-race {\bf MERGE} nem\v en\'\i\ po\v cet strom\accent23u a 
proto\v ze operace {\bf INSERT} p\v rid\'a jen jeden strom, je 
amortizovan\'a slo\v zitost operac\'\i\ {\bf MERGE} a {\bf INSERT} st\'ale $
O(1)$.
Uk\'a\v zeme, \v ze amortizovan\'a slo\v zitost operac\'\i\ {\bf MIN} a {\bf DELETEMIN }
p\v ri l\'\i n\'e implementaci binomi\'aln\'\i ch hald je $O(\log
(|S|)$. 
Proto\v ze ka\v zd\'y b\v eh 
vnit\v rn\'\i ho {\bf while}-cyklu v operaci {\bf vyvaz} vy\v zaduje \v cas $
O(1)$ 
a zmen\v s\'\i\ po\v cet strom\accent23u v seznamech $O_i$ o $1$, 
ope\-ra\-ce {\bf vyvaz} vy\v zaduje \v cas $O(k+\sum_{i=0}^k|O_i|
)$. Ope\-ra\-ce {\bf MIN}
bez podprocedury {\bf vyvaz} vy\v zaduje \v cas $O(|\Cal H|)$ a operace 
{\bf DELETEMIN} bez podprocedury {\bf vyvaz }\v cas $O(\Cal H+i)$ pro 
takov\'e $i$, \v ze $T$ je izomorfn\'\i\ s $H_i$. Podle tvrzen\'\i\ je $
i\le\log(|S|)$, a 
tedy ope\-ra\-ce {\bf MIN} vy\v zaduje \v cas $O(|\Cal H|)$ a ope\-race 
{\bf DELETEMIN }\v cas $O(|\Cal H|+\log(|S|))$. Proto\v ze  
ohodnocen\'\i\ klasick\'e binomi\'aln\'\i\ haldy je nejv\'y\v se $\log
(|S|)$ 
(obsahuje tolik strom\accent23u, kolik je $1$ v bin\'arn\'\i m z\'apise 
\v c\'\i sla $|S|$), dost\'av\'ame, \v ze amortizovan\'a slo\v zitost ope\-race {\bf MIN} je 
$O(|\Cal H|-|\Cal H|+\log(|S|))=O(\log(|S|))$ a amortizovan\'a slo\v zitost 
operace {\bf DELETEMIN} je $O(|\Cal H|+\log(|S|)-|\Cal H|+\log(|S
|))=O(\log(|S|))$.
\medskip

\flushpar Proto\v ze si funkci ohodnocen\'\i\ vol\'\i me, 
m\accent23u\v zeme pou\v z\'\i t takov\'e multiplikativn\'\i\ 
koeficienty, aby jednotka \v casu odpov\'\i dala jednotce v 
amortizovan\'e slo\v zitosti. Proto lze $|\Cal H|$ od sebe ode\v c\'\i st.
\bigskip

\centerline{\bigbigrm{VI. Fibonacciho haldy}}
\medskip

\flushpar V\'yznam Fibonacciho hald ur\v cuje fakt, \v ze amortizovan\'a 
slo\v zitost operac\'\i\ {\bf INSERT} a {\bf DECREASE} v t\v echto hald\'ach je 
$O(1)$ a amortizovan\'a slo\v zitost operace {\bf DELETEMIN} je $
O(\log(|S|)$. 
Proto se hodn\v e pou\v z\'\i vaj\'\i\ v grafov\'ych algoritmech, 
kde umo\v z\v nuj\'\i\ v mnoha p\v r\'\i padech dos\'ahnout asymptoticky t\'em\v e\v r line\'arn\'\i\ 
slo\v zitosti. Nezn\'ame v\v sak \v z\'adn\'e  
expe\-riment\'aln\'\i\ v\'ysledky, kter\'e by porovn\'avaly pou\v zit\'\i\ Fibonacciho 
hald a nap\v r. $d$-regul\'arn\'\i ch hald v t\v echto grafov\'ych 
algoritmech v praxi. Tak\v ze nezn\'ame podm\'\i nky, za 
kter\'ych jsou Fibonacciho 
haldy lep\v s\'\i\ ne\v z t\v reba $d$-regul\'arn\'\i\ haldy, 
ani nev\'\i me, do jak\'e m\'\i ry je to jen teoretick\'y 
v\'ysledek a do jak\'e m\'\i ry jsou opravdu prakticky pou\v ziteln\'e. 
\medskip

\flushpar Neform\'aln\v e \v re\v ceno, je Fibonacciho halda mno\v zina 
strom\accent23u, jejich\v z n\v ekter\'e vrcholy r\accent23uzn\'e od 
ko\v ren\accent23u jsou ozna\v ceny, a kde existuje 
jednozna\v cn\'a korepondence mezi prvky $S$ a vrcholy 
strom\accent23u (realizov\'ana funkc\'\i\ $\key$), kter\'a spl\v nuje 
podm\'\i nku \thetag{usp}.  Toto je v\v sak jen p\v ribli\v z\-n\'e 
vyj\'ad\v ren\'\i .  Existuj\'\i\ toti\v z struktury, na kter\'e se tento 
popis hod\'\i , ale 
nevznikly z pr\'azd\-n\'e Fibonacciho haldy aplikac\'\i\ 
posloupnosti haldov\'ych  
operac\'\i .  P\v ritom d\accent23ukaz efekti\-vity Fibonacciho hald se 
dosti v\'yrazn\v e op\'\i r\'a o fakt, \v ze halda vznikla z pr\'azdn\'e 
haldy aplikac\'\i\ algoritm\accent23u pro Fibonacciho 
haldy. Proto nejprve pop\'\i\v seme algoritmy pro tyto 
ope\-race, a pak budeme
definovat \emph{Fibonacciho} \emph{haldy} jako struktury 
vznikl\'e z pr\'azdn\'e haldy aplikac\'\i\ posloupnosti 
t\v echto algoritm\accent23u.
\bigskip

\head
Algoritmy
\endhead

\flushpar V algoritmech p\v redpokl\'ad\'ame, \v ze Fibonacciho halda je 
seznam strom\accent23u, kde n\v ekter\'e vrcholy r\accent23uzn\'e od 
ko\v ren\accent23u jsou ozna\v ceny.  Vrchol je ozna\v cen, pr\'av\v e kdy\v z nen\'\i\ 
ko\v ren a kdy\v z mu byl n\v ekdy d\v r\'\i ve odtr\v zen n\v ekter\'y jeho syn. Toto se 
nezaznamen\'av\'a pro ko\v reny strom\accent23u. Proto kdy\v z se 
vrchol stane ko\v renem (odtr\v zen\'\i m podstromu ur\v cen\'eho 
t\'\i mto vrcholem),  
zapomene  se tento \'udaj a za\v cne se znovu zaznamen\'avat, 
a\v z kdy\v z vrchol p\v restane 
b\'yt ko\v renem. \v Rekneme, \v ze strom m\'a \emph{rank} $i$, kdy\v z 
jeho ko\v ren 
m\'a $i$ syn\accent23u. Tento fakt nahrazuje test pou\v z\'\i van\'y 
v binomi\'aln\'\i ch 
hald\'ach, \v ze strom je izomorfn\'\i\ se stromem $H_i$. 
\medskip

\flushpar Algoritmy pro operace {\bf MERGE}, {\bf INSERT}, {\bf MIN} a 
{\bf DELE\-TEMIN} jsou zalo\v zeny na stejn\'ych idej\'\i ch jako algoritmy pro l\'\i nou implementaci 
v binomi\'aln\'\i ch hald\'ach, pouze po\v za\-da\-vek, aby strom byl 
izomorfn\'\i\ s $H_i$, je nahrazen po\v zadavkem, \v ze m\'a rank $
i$. 
Algoritmy pro ope\-race {\bf DECREASE}, {\bf INCREASE} a {\bf DELETE }
vych\'azej\'\i\ z algoritm\accent23u pro tyto operace v leftist 
hald\'ach. V algoritmech p\v redpokl\'ad\'ame, \v ze $c=\log^{-1}
(\frac 32)$.
\bigskip

{\bf MERGE$(\Cal H_1,\Cal H_2)$}:\newline 
Prove\v d konkatenaci seznam\accent23u $\Cal H_1$ a $\Cal H_2$
\bigskip

{\bf INSERT$(x)$}:\newline 
Vytvo\v r haldu $\Cal H_1$ reprezentuj\'\i c\'\i\ $\{x\}$\newline 
$\bold M\bold E\bold R\bold G\bold E(\Cal H,\Cal H_1)$
\bigskip

{\bf MIN}:\newline 
Prohledej prvky reprezentovan\'e ko\v reny v\v sech 
strom\accent23u v $\Cal H$\newline 
{\bf V\'ystup}: nejmen\v s\'\i\ z t\v echto prvk\accent23u\newline 
stromy rozd\v el do mno\v zin $O_i=\{$v\v sechny stromy v 
$\Cal H$ s rankem $i\}$\newline 
{\bf vyvaz1$(\{O_i\mid i=0,1,\dots,\lfloor c\log(\sqrt 5|S|+1)\rfloor 
\})$
\bigskip

DELETEMIN}:\newline 
Prohledej prvky reprezentovan\'e ko\v reny v\v sech 
strom\accent23u v $\Cal H$\newline 
$T:=$ strom, jeho\v z ko\v ren reprezentuje nejmen\v s\'\i\ 
prvek\newline 
stromy rozd\v el do mno\v zin $O_i=\{$v\v sechny stromy v 
$\Cal H$ s rankem $i$ r\accent23uzn\'e od $T\}\cup \{$podstrom $T$ ur\v cen\'y 
n\v ekter\'ym synem ko\v rene $T$ s 
rankem $i$$\}$\newline 
{\bf vyvaz1$(\{O_i\mid i=0,1,\dots,\lfloor c\log(\sqrt 5|S|+1)\rfloor 
\})$
\bigskip

vyvaz1$(\{O_i\mid i=0,1,\dots,k\})$}:\newline 
$i:=0$, $\Cal H:=\emptyset$\newline 
{\bf while} existuje $O_i\ne\emptyset$ {\bf do\newline 
\phantom{{\rm ---}}while} $|O_i|>1$ {\bf do\newline 
\phantom{{\rm ------}}}vezmi dva r\accent23uzn\'e stromy $T_1$ a $
T_2$ z 
$O_i$\newline 
\phantom{------}odstra\v n je z $O_i$\newline 
\phantom{------}{\bf spoj$(T_1,T_2)$} vlo\v z do $O_{i+1}$\newline 
\phantom{---}{\bf enddo\newline 
\phantom{{\rm ---}}if} $O_i\ne\emptyset$ {\bf then\newline 
\phantom{{\rm ------}}}strom $T\in O_i$ odstra\v n z $O_i$ a vlo\v z ho do $
\Cal H$\newline 
\phantom{---}{\bf endif\newline 
\phantom{{\rm ---}}$i:=i+1$\newline 
enddo\newline 
V\'ystup}: $\Cal H$
\bigskip

{\bf spoj$(T_1,T_2)$}:\newline 
{\bf if} $f(\text{\rm ko\v ren }T_1)>f(\text{\rm ko\v ren }T_2)$ {\bf then\newline 
\phantom{{\rm{\rm ---}}}}vym\v e\v n stromy $T_1$ a $T_2$\newline 
{\bf endif\newline 
}
vytvo\v r nov\'eho syna $v$ ko\v rene $T_1$\newline 
$v:=$ko\v ren $T_2$
\bigskip

{\bf DECREASE$(s,a)$}:\newline 
$T:=$strom v $\Cal H$, kter\'y obsahuje vrchol reprezentuj\'\i c\'\i\ $
s$\newline 
$v:=$vrchol stromu $T$ reprezentuj\'\i c\'\i\ $s$\newline 
{\bf if} $v$ nen\'\i\ ko\v ren {\bf then \newline 
\phantom{{\rm ---}}}odtrhni podstrom $T'$ ur\v cen\'y vrcholem $v$\newline 
\phantom{---}{\bf vyvaz2$(T,v)$\newline 
\phantom{{\rm ---}}if} $v$ byl ozna\v cen {\bf then} zru\v s ozna\v cen\'\i\ $
v$ {\bf endif\newline 
\phantom{{\rm ---}}}vlo\v z $T'$ do $\Cal H$\newline 
{\bf endif\newline 
$f(v):=f(v)-a$
\bigskip

INCREASE$(s,a)$}:\newline 
$T:=$strom v $\Cal H$, kter\'y obsahuje vrchol reprezentuj\'\i c\'\i\ $
s$\newline 
$v:=$vrchol stromu $T$ reprezentuj\'\i c\'\i\ $s$\newline 
{\bf if} $v$ nen\'\i\ list {\bf then \newline 
\phantom{{\rm ---}}}odtrhni podstrom $T'$ ur\v cen\'y vrcholem $v$\newline 
\phantom{---}{\bf if} $v$ nen\'\i\ ko\v ren {\bf then vyvaz2$(T,v
)$ endif\newline 
\phantom{{\rm ---}}if} $v$ byl ozna\v cen {\bf then} zru\v s ozna\v cen\'\i\ $
v$ {\bf endif\newline 
\phantom{{\rm ---}}}zru\v s ozna\v cen\'\i\ v\v sech syn\accent23u vrcholu $
v$\newline 
\phantom{---}odtrhni podstromy $T'$ ur\v cen\'e v\v semi syny $v$ 
a vlo\v z je do $\Cal H$\newline 
\phantom{{\bf ---}}do $\Cal H$ vlo\v z strom maj\'\i c\'\i\ jen vrchol $
v$\newline 
{\bf endif\newline 
$f(v):=f(v)+a$
\bigskip

DELETE$(s)$}:\newline 
$T:=$strom v $\Cal H$, kter\'y obsahuje vrchol reprezentuj\'\i c\'\i\ $
s$\newline 
$v:=$vrchol stromu $T$ reprezentuj\'\i c\'\i\ $s$\newline 
{\bf if} $v$ nen\'\i\ list {\bf then\newline 
\phantom{{\rm ---}}}zru\v s ozna\v cen\'\i\ syn\accent23u vrcholu $
v$\newline 
\phantom{---}odtrhni podstromy ur\v cen\'e v\v semi syny 
vrcholu $v$ a vlo\v z je do $\Cal H$\newline 
{\bf endif\newline 
if} $v$ nen\'\i\ ko\v ren {\bf then vyvaz2$(T,v)$ endif\newline 
}
zru\v s vrchol $v$
\bigskip

{\bf vyvaz2$(T,v)$}:\newline 
$u:=\otec v$\newline 
{\bf while} $u$ je ozna\v cen {\bf do\newline 
\phantom{{\rm ---}}$u':=\otec(u)$}, zru\v s ozna\v cen\'\i\ $u$\newline 
\phantom{---}odtrhni podstrom $T'$ ur\v cen\'y vrcholem $u$\newline 
\phantom{---}vlo\v z $T'$ do $\Cal H$, $u:=u'$\newline 
{\bf enddo\newline 
if} $u$ nen\'\i\ ko\v ren $T$ {\bf then} ozna\v c $u$ {\bf endif
\bigskip

}\flushpar V\v simn\v eme si, \v ze kdy\v z stromy $T_1$ a $T_2$ maj\'\i\ rank 
$i$, pak procedura {\bf spoj$(T_1,T_2)$} vytvo\v r\'\i\ strom s rankem 
$i+1$.  Aby algoritmy pro operace {\bf MIN} a {\bf DELETEMIN} byly 
korektn\'\i , mus\'\i me uk\'azat, \v ze v\v sechny stromy ve 
Fibonacciho hald\v e $\Cal H$ reprezentuj\'\i c\'\i\ mno\v zinu $
S$ maj\'\i\ rank 
nejv\'y\v se $c\log(\sqrt 5|S|+1)$. Jen tak zajist\'\i me, aby v\'ysledn\'a 
halda reprezentovala $S$, res\-pektive 
$S\setminus \{\text{\rm prvek s nejmen\v s\'\i\ hodnotou }f\}$.  Operace {\bf vyvaz1 }
zaji\v s\v tuje, \v ze od ka\v zd\'eho vrcholu stromu r\accent23uzn\'eho 
od ko\v rene byl v tomto strom\v e odtr\v zen podstrom nejv\'y\v se 
jednoho syna -- v tom p\v r\'\i pad\v e je tento prvek ozna\v cen a 
kdy\v z se mu odtrh\'av\'a podstrom dal\v s\'\i ho syna, bude odtr\v zen 
i cel\'y podstrom tohoto vrcholu (t\'\i m se tento vrchol stane ko\v renem 
stromu).  Kdy\v z se pozd\v eji stane tento vrchol zase 
vrcholem r\accent23uzn\'ym od ko\v rene, cel\'y proces se 
opakuje. 
\bigskip

\head
Slo\v zitost operac\'\i
\endhead

\flushpar Na\v s\'\i m c\'\i lem bude odhadnout amortizovanou slo\v zitost t\v echto 
operac\'\i , proto\v ze slo\v zitost v nejhor\v s\'\i m p\v r\'\i pad\v e nen\'\i\ 
pou\v ziteln\'y v\'ysledek.  Abychom to mohli ud\v elat, spo\v c\'\i t\'ame parametry 
slo\v zitosti jednotliv\'ych operac\'\i :  
\medskip

\flushpar {\bf MERGE} -- \v casov\'a slo\v zitost $O(1)$, nevznik\'a \v z\'adn\'y 
nov\'y strom, 
ozna\-\v ce\-n\'e vrcholy se nem\v en\'\i ;\newline 
{\bf INSERT} -- \v casov\'a slo\v zitost $O(1)$, p\v ribyl jeden strom, 
ozna\v cen\'e vrcholy se nem\v en\'\i ;\newline 
{\bf MIN} -- \v casov\'a slo\v zitost $O(|\Cal H|)$, po proveden\'\i\ operace 
r\accent23uzn\'e stro\-my v hald\v e maj\'\i\ r\accent23uzn\'e 
ranky, ozna\v cen\'e vrcholy se nem\v en\'\i ;\newline 
{\bf DELETEMIN} -- \v casov\'a slo\v zitost $O(|\Cal H|+\text{\rm po\v cet syn\accent23u }
v)$,
kde $v$ reprezentoval prvek s nej\-men\v s\'\i\ hodnotou $f$. Po 
proveden\'\i\ ope\-ra\-ce r\accent23uzn\'e stromy v hald\v e maj\'\i\ 
r\accent23uzn\'e ranky, \v z\'adn\'y nov\'y vrchol nebyl 
ozna\v cen, n\v ekter\'e ozna\v cen\'e vrcholy p\v restaly b\'yt ozna\v cen\'e;\newline 
{\bf DECREASE} -- \v casov\'a slo\v zitost $O(1+c)$, kde $c$ je po\v cet 
vrchol\accent23u, kter\'e p\v restaly b\'yt ozna\v cen\'e. 
Bylo p\v rid\'ano $1+c$ nov\'ych strom\accent23u a byl ozna\v cen 
nejv\'y\v se jeden vrchol;\newline 
{\bf INCREASE} -- \v casov\'a slo\v zitost $O(1+c+d)$, kde $c$ je po\v cet 
vrchol\accent23u, kter\'e p\v restaly b\'yt ozna\v cen\'e, $d$ je 
po\v cet syn\accent23u vrcholu $v$ reprezentuj\'\i c\'\i ho prvek, 
jeho\v z hodnota se zvy\v suje. Bylo p\v rid\'ano nejv\'y\v se $1
+c+d$ 
nov\'ych strom\accent23u a byl ozna\v cen nejv\'y\v se jeden 
vrchol;\newline 
{\bf DELETE} -- \v casov\'a slo\v zitost $O(1+c+d)$, kde $c$ je po\v cet 
vrchol\accent23u, kter\'e p\v restaly b\'yt ozna\v cen\'e, $d$ je 
po\v cet syn\accent23u vrcholu $v$ reprezentuj\'\i c\'\i ho prvek, 
kter\'y se m\'a odstranit. Bylo p\v rid\'ano nejv\'y\v se $c+d$ 
nov\'ych strom\accent23u a byl ozna\v cen nejv\'y\v se jeden 
vrchol.
\medskip

\flushpar Pro v\'ypo\v cet amortizovan\'e slo\v zitosti 
mus\'\i me nejprve navrhnout funkci ohodnocuj\'\i c\'\i\ 
konfigurace.  P\v ri vy\v set\v rov\'an\'\i\ l\'\i n\'e implementace binomi\'aln\'\i ch 
hald se uk\'azalo, \v ze vhodn\'ym ohodnocen\'\i m je po\v cet strom\accent23u 
v hald\v e. Kdy\v z si ale prohl\'edneme algoritmus pro operaci 
{\bf DECREASE}, vid\'\i me, \v ze zde je vhodn\'e br\'at do ohodnocen\'\i\ i 
po\v cet ozna\v cen\'ych vrchol\accent23u, a to dokonce tak, aby 
se pokryl 
nejen \v cas, ale i p\v r\'\i r\accent23ustek strom\accent23u. To vede k 
n\'asleduj\'\i c\'\i mu ohodnocen\'\i\ konfigurace: ohodnocen\'\i\ je po\v cet 
strom\accent23u v konfiguraci plus dvojn\'asobek po\v ctu 
ozna\v cen\'ych vrchol\accent23u.
\medskip
\flushpar Nech\v t $\rho (n)$ je maxim\'aln\'\i\ 
po\v cet syn\accent23u vrcholu ve Fibonacciho hald\v e 
reprezentuj\'\i c\'\i\ $n$-prvkovou mno\v zinu.  Pak amortizovan\'a 
slo\v zitost operac\'\i\ {\bf MERGE}, {\bf INSERT} a {\bf DECREASE} je $
O(1)$ a 
operac\'\i\ {\bf MIN}, {\bf DELETEMIN}, {\bf INCREASE} a {\bf DELETE} je $
O(\rho (n))$.
\medskip

\flushpar Abychom spo\v c\'\i tali odhad $\rho (n)$, vyu\v zijeme toho, \v ze 
Fibonacciho halda vznikla z pr\'azdn\'e haldy pomoc\'\i\ 
popsan\'ych algoritm\accent23u. Nejprve uvedeme jedno technick\'e 
lemma.
\medskip

\proclaim{Lemma}Nech\v t $v$ je vrchol stromu ve 
Fibonacciho hald\v e a nech\v t $u$ je $i$-t\'y nejstar\v s\'\i\ syn 
vrcholu $v$. Pak $u$ m\'a aspo\v n $i-2$ syn\accent23u.
\endproclaim

\demo{D\accent23ukaz}V moment\v e, kdy se $u$ st\'aval synem $v$, se 
aplikovala operace {\bf spoj}, $u$ a $v$ byly ko\v reny 
strom\accent23u a m\v ely stejn\'y po\v cet syn\accent23u. 
Podle p\v redpoklad\accent23u m\v el vrchol $v$ alespo\v n $i-1$ 
syn\accent23u (jinak by $u$ nebyl $i$-t\'y nejstar\v s\'\i\ syn), a 
proto\v ze se od $u$ mohl odtrhnout jen jeden syn, 
dost\'av\'ame, \v ze $u$ mus\'\i\ m\'\i t alespo\v n $i-2$ syn\accent23u. \qed
\enddemo
\medskip

\proclaim{Tvrzen\'\i}Nech\v t $v$ je vrchol stromu ve 
Fibonacciho hald\v e, kte\-r\'y m\'a pr\'av\v e $i$ syn\accent23u. Pak 
podstrom ur\v cen\'y vrcholem $v$ m\'a aspo\v n $F_{i+2}$ 
vrchol\accent23u.
\endproclaim

\demo{D\accent23ukaz}Tvrzen\'\i\ dok\'a\v zeme indukc\'\i\ 
podle maxim\'aln\'\i\ d\'elky cesty z vrcholu $v$ do n\v ekter\'eho 
listu.  Tato d\'elka je $0$, pr\'av\v e kdy\v z $v$ je list.  V tom 
p\v r\'\i pad\v e $v$ nem\'a syna a podstrom ur\v cen\'y vrcholem $
v$ m\'a 
jedin\'y vrchol.  Proto\v ze $1=F_2=F_{0+2}$, tvrzen\'\i\ plat\'\i .  
M\v ejme nyn\'\i\ vrchol $v$, kter\'y m\'a $k$ 
syn\accent23u, a nech\v t maxim\'aln\'\i\ d\'elka cesty z vrcholu $
v$ do 
list\accent23u je $j$. P\v redpokl\'adej\-me, \v ze tvrzen\'\i\ plat\'\i\ pro 
v\v sechny vrcholy, pro n\v e\v z tato d\'elka  
je men\v s\'\i\ ne\v z $j$,  tedy plat\'\i\ i pro 
v\v sechny syny vrcholu $v$.  Pak pro $i>1$ m\'a $i$-t\'y nejstar\v s\'\i\ 
syn vrcholu $v$ podle p\v redchoz\'\i ho lemmatu alespo\v n $i-2$ 
syn\accent23u a podle induk\v cn\'\i ho p\v redpokladu podstrom ur\v cen\'y 
t\'\i mto synem m\'a alespo\v n $F_i$ vrchol\accent23u.  Odtud 
dost\'av\'ame, \v ze podstrom ur\v cen\'y vrcholem $v$ m\'a alespo\v n 
$$1+F_2+\sum_{i=2}^kF_i=1+\sum_{i=1}^kF_i$$
vrchol\accent23u, proto\v ze $F_1=F_2$ (na lev\'e stran\v e prvn\'\i\ $
1$ 
je za vrchol $v$ a 
prvn\'\i\ $F_2$ je za nejstar\v s\'\i\ vrchol). Indukc\'\i\ pak dostaneme, \v ze 
$$1+\sum_{i=1}^nF_i=F_{n+2}$$
pro v\v sechna $n\ge 0$. Skute\v cn\v e, pro $n=0$ plat\'\i\ 
$$1+\sum_{i=1}^0F_i=1=F_2=F_{0+2},$$
pro $n=1$ m\'ame 
$$1+\sum_{i=1}^1F_i=1+F_1=2=F_3=F_{1+2}$$
a z induk\v cn\'\i ho p\v redpokladu a z vlastnost\'\i\ Fibonacciho 
\v c\'\i sel plyne, \v ze 
$$1+\sum_{i=1}^nF_i=1+\sum_{i=1}^{n-1}F_i+F_n=F_{n+1}+F_n=F_{n+2}
.$$
Kdy\v z shrneme tato fakta, dost\'av\'ame, \v ze podstrom 
ur\v cen\'y vrcholem $v$ m\'a alespo\v n $F_{i+2}$ vrchol\accent23u, a 
tvrzen\'\i\ je dok\'az\'ano. \qed
\enddemo
\medskip

\flushpar Vezm\v eme nyn\'\i\ nejmen\v s\'\i\ $i$ takov\'e, \v ze $
n<F_i$. Proto\v ze 
posloupnost $\{F_i\}_{i=1}^{\infty}$ je rostouc\'\i , plyne z p\v redchoz\'\i ho 
tvrzen\'\i , \v ze ka\v zd\'y vrchol ve Fibonacciho hald\v e 
reprezentuj\'\i c\'\i\ $n-$prvkovou mno\v zinu m\'a m\'en\v e ne\v z $
i-2$ 
syn\accent23u (kdy\v z vrchol $v$ Fibonacciho haldy m\'a $i-2$ 
syn\accent23u, pak podstrom vrcholu $v$ reprezentuje 
mno\v zinu alespo\v n s $F_i$ prvky). Proto $\rho (n)<i-2$. K odhadu 
velikosti $i$ pou\v zijeme explicitn\'\i\ vzorec 
pro $i$-t\'e Fibonacciho \v c\'\i slo: 
$$F_i=\frac {\big(\frac {1+\sqrt 5}2\big)^i-\big(\frac {1-\sqrt 5}
2\big)^i}{\sqrt 5}=\frac 1{\sqrt 5}\big(\frac {1+\sqrt 5}2\big)^i
-\frac 1{\sqrt 5}\big(\frac {1-\sqrt 5}2\big)^i.$$
Proto\v ze $0>\frac {1-\sqrt 5}2>-\frac 34$ a proto\v ze $\sqrt 5
>2$, dost\'av\'ame, \v ze $|\frac 1{\sqrt 5}\big(\frac {1-\sqrt 5}
2\big)^i|<\frac 38$ pro v\v sechna 
$i=1,2,\dots$, a tedy 
$$\frac 1{\sqrt 5}\big(\frac {1+\sqrt 5}2\big)^i-\frac 38<F_i<\frac 
1{\sqrt 5}\big(\frac {1+\sqrt 5}2\big)^i+\frac 38.$$
Odtud plyne, \v ze kdy\v z $i$ spl\v nuje 
$$n\le\frac 1{\sqrt 5}\big(\frac {1+\sqrt 5}2\big)^i-\frac 38,$$
pak $n<F_i$. 
P\v reveden\'\i m $\frac 38$ na druhou stranu v\'yrazu, jeho 
vyn\'asoben\'\i m $\sqrt 5$ a zlogaritmov\'an\'\i m dostaneme n\'asleduj\'\i c\'\i\ 
ekvivalenci: 
$$\log_2(\sqrt 5n+\frac {3\sqrt 5}8)\le i\log_2\big(\frac {1+\sqrt 
5}2\big)\quad\Leftrightarrow\quad n\le\frac 1{\sqrt 5}\big(\frac {
1+\sqrt 5}2\big)^i-\frac 38.$$
Z $\frac {3\sqrt 5}8<1$ a z $\frac 32<\frac {1+\sqrt 5}2$ plyne, \v ze 
$$\frac {\log_2(\sqrt 5n+\frac {3\sqrt 5}8)}{\log_2\frac {1+\sqrt 
5}2}<\frac {\log_2(\sqrt 5n+1)}{\log_2\frac 32}.$$
Tedy plat\'\i\ n\'asleduj\'\i c\'\i\ implikace
$$\frac {\log_2(\sqrt 5n+1)}{\log_2\frac 32}<i\quad\implies\frac {\log_
2(\sqrt 5n+\frac {3\sqrt 5}8)}{\log_2\big(\frac {1+\sqrt 5}2\big)}
<i.$$
Proto kdy\v z $\frac {\log_2(\sqrt 5n+1)}{\log_23-1}<i$, pak $n<F_
i$, a tedy $\rho (n)<i-2$.
\medskip

\flushpar V\'ysledky shrneme do n\'asleduj\'\i c\'\i\ v\v ety: 
\medskip

\proclaim{V\v eta}Ve Fibonacciho hald\v e, kter\'a   
reprezentuje $n$-prvkovou mno\v zinu, m\'a ka\v zd\'y vrchol 
stu\-pe\v n men\v s\'\i\ ne\v z 
$$\frac {\log_2(\sqrt 5n+1)}{(\log_23)-1}-2.$$
Amortizovan\'a slo\v zitost operac\'\i\ {\bf INSERT}, {\bf MERGE} a {\bf DECREASE }
je $O(1)$ a amortizovan\'a slo\v zitost operac\'\i\ {\bf MIN}, 
{\bf DE\-LE\-TE\-MIN}, {\bf INCREASE} a {\bf DELETE} je $O(\log n
)$. Operace {\bf MIN }
a {\bf DELETEMIN} jsou korektn\'\i .
\endproclaim
\medskip

\flushpar Pro \'uplnost dok\'a\v zeme, \v ze $F_i=\frac {\big(\frac {
1+\sqrt 5}2\big)^i-\big(\frac {1-\sqrt 5}2\big)^i}{\sqrt 5}$ .\newline 
Pro $i=1$ plat\'\i\ 
$$\frac {\big(\frac {1+\sqrt 5}2\big)^1-\big(\frac {1-\sqrt 5}2\big
)^1}{\sqrt 5}=\frac {1+\sqrt 5-1+\sqrt 5}{2\sqrt 5}=\frac {2\sqrt 
5}{2\sqrt 5}=1=F_1.$$
Pro $i=2$ plat\'\i\ 
$$\frac {\big(\frac {1+\sqrt 5}2\big)^2-\big(\frac {1-\sqrt 5}2\big
)^2}{\sqrt 5}=\frac {1+2\sqrt 5+5-1+2\sqrt 5-5}{4\sqrt 5}=\frac {
4\sqrt 5}{4\sqrt 5}=1=F_2.$$
Induk\v cn\'\i\ krok:
$$\align&\frac {\big(\frac {1+\sqrt 5}2\big)^i-\big(\frac {1-\sqrt 
5}2\big)^i}{\sqrt 5}=\frac {\big(\frac {1+\sqrt 5}2\big)^{i-2}\big
(\frac {1+\sqrt 5}2\big)^2-\big(\frac {1-\sqrt 5}2\big)^{i-2}\big
(\frac {1-\sqrt 5}2\big)^2}{\sqrt 5}=\\
&\frac {\big(\frac {1+\sqrt 5}2\big)^{i-2}\big(\frac {3+\sqrt 5}2\big
)-\big(\frac {1-\sqrt 5}2\big)^{i-2}\big(\frac {3-\sqrt 5}2\big)}{\sqrt 
5}=\\
&\frac {\big(\frac {1+\sqrt 5}2\big)^{i-2}\big(1+\frac {1+\sqrt 5}
2\big)-\big(\frac {1-\sqrt 5}2\big)^{i-2}\big(1+\frac {1-\sqrt 5}
2\big)}{\sqrt 5}=\\
&\frac {\big(\frac {1+\sqrt 5}2\big)^{i-2}+\big(\frac {1+\sqrt 5}
2\big)^{i-1}-\big(\frac {1-\sqrt 5}2\big)^{i-2}-\big(\frac {1-\sqrt 
5}2\big)^{i-1}}{\sqrt 5}=\\
&\frac {\big(\frac {1+\sqrt 5}2\big)^{i-2}-\big(\frac {1-\sqrt 5}
2\big)^{i-2}}{\sqrt 5}+\frac {\big(\frac {1+\sqrt 5}2\big)^{i-1}-\big
(\frac {1-\sqrt 5}2\big)^{i-1}}{\sqrt 5}=F_{i-2}+F_{i-1}=F_i.\endalign$$
Tedy indukc\'\i\ dost\'av\'ame po\v zadovan\'y vztah.
\bigskip

\head
Aplikace 
\endhead


\flushpar Vr\'at\'\i me se k Dijkstrov\v e algoritmu.  
Mno\v zinu $U$ bude\-me reprezentovat pomoc\'\i\ Fibonacciho haldy.  
Proto\v ze ohodnocen\'\i\ je nez\'aporn\'e a ohodnocen\'\i\ po\v c\'ate\v cn\'\i\ haldy 
je $0$, d\'av\'a odhad amortizovan\'e slo\v zitosti tak\'e odhad \v casov\'e 
slo\v zitosti (viz odstavec IV.).  Proto Dijkstr\accent23uv 
algoritmus s pou\v zit\'\i m 
Fibonacciho haldy vy\v zaduje v nejhor\v s\'\i m p\v r\'\i pad\v e \v cas 
$O(|X|(1+\log|X|)+|R|)=O(|R|+|X|\log|X|)$.  Stejn\'y v\'ysledek 
dostane\-me i pro konstrukci nejmen\v s\'\i\ napnut\'e kostry grafu.  
\medskip

\flushpar Ot\'azka je, kdy v Dijkstrov\v e algoritmu nebo v 
algoritmu konstruuj\'\i\-c\'\i m nejmen\v s\'\i\ napnutou kostru pou\v z\'\i t Fibonacciho haldu a kdy 
nap\v r.  $d$-regu\-l\'ar\-n\'\i\ haldy. Lze 
\v r\'\i ci, \v ze Fibonacciho halda by m\v ela b\'yt v\'yrazn\v e lep\v s\'\i\ pro 
v\v et\v s\'\i , ale \v r\'\i dk\'e grafy (tj. grafy s mal\'ym po\v ctem hran). 
D\'a se p\v redpokl\'adat, \v ze $d$-regul\'arn\'\i\ haldy budou lep\v s\'\i\ 
(d\'\i ky sv\'ym jednodu\v s\v s\'\i m algoritm\accent23um) pro hust\'e 
grafy (tj. grafy, kde po\v cet hran je $|X|^{1+\varepsilon}$ pro vhodn\'e 
$\varepsilon >0$). Probl\'em je, pro kter\'e hodnoty nast\'av\'a zlom. 
Nev\'\i m o \v z\'adn\'ych experiment\'aln\'\i ch ani teoretick\'ych 
v\'ysledc\'\i ch tohoto typu.
\bigskip

\head
Historick\'y p\v rehled
\endhead

\flushpar Bin\'arn\'\i\ neboli $2$-regul\'arn\'\i\ haldy 
zavedl Williams 1964.  Jejich zobecn\v en\'\i\ na $d$-regul\'arn\'\i\ haldy 
poch\'az\'\i\ od Johnsona 1975.  Leftist haldy definoval Crane 1972 a 
detailn\v e popsal Knuth 1975.  Binomi\'aln\'\i\ haldy navrhnl Vuillemin 
1978, Brown 1978 je implementoval a prok\'azal jejich praktickou 
pou\v zitelnost.  Fibonacciho haldy byly zavedeny Fredmanem a 
Tarjanem 1987.  
\newpage

\centerline{\bigbigrm{VII. T\v r\'\i dic\'\i\ algoritmy}}
\bigskip

\flushpar Jednou z nej\v cast\v eji \v re\v sen\'ych \'uloh p\v ri pr\'aci s 
daty je set\v r\'\i d\v en\'\i\ 
posloupnosti prvk\accent23u n\v ejak\'eho typu. Proto velk\'a pozornost byla a je v\v enov\'ana 
t\v r\'\i dic\'\i m algoritm\accent23um \v re\v s\'\i c\'\i m tuto \'ulohu, kter\'a 
sv\'ym charakterem a sv\'ymi 
po\v zadavky na algoritmy je \v razena do datov\'ych 
struktur. Byla navr\v zena \v rada algoritm\accent23u, kter\'e se 
st\'ale je\v st\v e analyzuj\'\i\ a optimalizuj\'\i . Anal\'yzy jsou velmi detailn\'\i\ a algoritmy se 
studuj\'\i\ za r\accent23uzn\'ych vstupn\'\i ch p\v redpoklad\accent23u. 
Krom\v e toho t\v r\'\i d\v en\'\i\ je jedna z m\'ala \'uloh, pro kterou 
alespo\v n za jist\'ych 
p\v redpoklad\accent23u um\'\i me spo\v c\'\i tat doln\'\i\ odhad slo\v zitosti.
\medskip

\flushpar Formulace \'ulohy:\newline 
Nech\v t $U$ je tot\'aln\v e uspo\v r\'adan\'e univerzum.\newline 
Vstup: Prost\'a posloupnost $\{a_1,a_2,\dots,a_n\}$ prvk\accent23u z 
univerza $U$.\newline 
V\'ystup: Rostouc\'\i\ posloupnost $\{b_1,b_2,\dots,b_n\}$ takov\'a, \v ze 
$\{a_i\mid i=1,2,\dots,n\}=\{b_i\mid i=1,2,\dots,n\}$.\newline 
Tento probl\'em se naz\'yv\'a \emph{t\v r\'\i d\v en\'\i}. 
V praxi se setk\'av\'ame s \v radou jeho modifikac\'\i , a nich\v z 
asi nejb\v e\v zn\v ej\v s\'\i\ je vynech\'an\'\i\ p\v redpokladu, \v ze 
vstupem je prost\'a posloupnost. Pak jsou dv\v e varianty 
\v re\v sen\'\i\ -- bu\v d se ve v\'ystupn\'\i\ posloupnosti odstran\'\i\ 
duplicity nebo v\'ystupn\'\i\ posloupnost zachov\'a \v cetnost prvk\accent23u ze 
vstupn\'\i\ posloupnosti.
\medskip

\flushpar Z\'akladn\'\i\ algoritmy, kter\'e  \v re\v s\'\i\ t\v r\'\i dic\'\i\ 
probl\'em, jsou {\bf QUICKSORT}, {\bf MERGESORT} a {\bf HEAP\-SORT}. 
\bigskip

\head
HEAPSORT
\endhead

\flushpar S algoritmem {\bf HEAPSORT} jsme se sezn\'amili p\v ri 
aplikac\'\i ch hald.  Byl to prvn\'\i\ algoritmus pou\v z\'\i vaj\'\i c\'\i\ haldy 
(bin\'arn\'\i\ regul\'ar\-n\'\i\ haldy byly definov\'any pr\'av\v e p\v ri n\'avrhu 
{\bf HEAP\-SORTU}).  Pod\'\i v\'ame se detailn\v eji na jednu z jeho 
implementac\'\i , kter\'a t\v r\'\i d\'\i\ takzvan\v e na m\'\i st\v e. 
\medskip

\flushpar T\v r\'\i dic\'\i\ algoritmy se \v casto pou\v z\'\i vaj\'\i\ jako 
podprocedura p\v ri \v re\v sen\'\i\ jin\'ych \'uloh. V takov\'em p\v r\'\i pad\v e je 
obvykle vstupn\'\i\ posloupnost ulo\v zena v poli v pracovn\'\i\ 
pam\v eti programu a po\v za\-dav\-kem  je set\v r\'\i dit ji 
bez pou\v zit\'\i\ dal\v s\'\i\ 
pam\v eti pouze s v\'yjimkou omezen\'eho (mal\'eho) po\v ctu pomocn\'ych 
prom\v enn\'ych. Pro \v re\v sen\'\i\ tohoto probl\'emu se hod\'\i\ 
{\bf HEAPSORT}. Zvol\'\i me implementaci {\bf HEAPSORTU} pomoc\'\i\ 
$d$-regul\'arn\'\i ch hald, kter\'e jsou reprezentov\'any  
polem, v n\v em\v z je ulo\v zena vstupn\'\i\ posloupnost (viz odstavec 
Aplikace v kapitole o $d$-regul\'arn\'\i ch hald\'ach). Pou\v zijeme 
algoritmus s jedinou zm\v enou -- budeme po\v zadovat du\'aln\'\i\ 
podm\'\i nku na uspo\v r\'ad\'an\'\i\ (to znamen\'a, \v ze prvek reprezentovan\'y 
vrcholem bude men\v s\'\i\ ne\v z prvek reprezentovan\'y jeho 
otcem) a nahrad\'\i me operace {\bf MIN} a {\bf DELETEMIN }
operacemi {\bf MAX} a {\bf DELETEMAX}. V algoritmu v\v zdy um\'\i st\'\i me 
odebran\'e maximum na m\'\i sto prvku v posled\-n\'\i m listu 
haldy (tj. prvku, kter\'y ho p\v ri operaci {\bf DELETEMAX }
nahradil) m\'\i sto toho, abychom ho vlo\v zili 
do v\'ystupn\'\i\ posloupnosti. Mus\'\i me si ale 
pamatovat, kde v poli kon\v c\'\i\ reprezentovan\'a halda. Ka\v zd\'a 
aplikace operace {\bf DELETEMAX} zkr\'at\'\i\ po\v c\'ate\v cn\'\i\ \'usek pole 
reprezentuj\'\i c\'\i ho haldu o jedno m\'\i sto 
a z\'arove\v n o toto m\'\i sto zv\v et\v s\'\i\ druhou \v c\'ast, ve kter\'e je ulo\v zena ji\v z 
set\v r\'\i d\v en\'a \v c\'ast posloupnosti. 
\medskip

\flushpar {\bf HEAPSORTU} je st\'ale v\v enov\'ana velk\'a pozornost a 
bylo navr\v ze\-no n\v ekolik jeho modifikac\'\i , sna\v z\'\i c\'\i ch se 
nap\v r. minimalizovat po\v cet porovn\'an\'\i\ prvk\accent23u apod.
\bigskip

\head
MERGESORT
\endhead

\flushpar Nejstar\v s\'\i\ z uveden\'ych algoritm\accent23u je 
{\bf MERGESORT}, kter\'y je star\v s\'\i\ ne\v z je po\v c\'\i ta\v cov\'a \'era, 
nebo\v t n\v ekter\'e jeho 
verze se pou\v z\'\i valy u\v z p\v ri mecha\-nick\'em t\v r\'\i d\v en\'\i . Pop\'\i\v seme 
jednu jeho itera\v cn\'\i\ verzi, tzv. p\v rirozen\'y {\bf MERGESORT}. 
\bigskip

{\bf MERGESORT$(a_1,a_2,\dots,a_n)$}:\newline 
$Q$ je pr\'azdn\'a fronta, $i=1$\newline 
{\bf while} $i\le n$ {\bf do\newline 
\phantom{{\rm ---}}$j:=i$\newline 
\phantom{{\rm ---}}while} $i<n$ a $a_{i+1}>a_i$ {\bf do} $i:=i+1$ {\bf enddo\newline 
\phantom{{\rm ---}}}posloupnost $P=(a_j,a_{j+1},\dots,a_i)$ vlo\v z do $
Q$\newline 
\phantom{---}$i:=i+1$\newline 
{\bf enddo\newline 
while} $|Q|>1$ {\bf do\newline 
\phantom{{\rm ---}}$ $}vezmi $P_1$ a $P_2$ dv\v e posloupnosti z vrcholu $
Q$\newline 
\phantom{---}odstra\v n $P_1$ a $P_2$ z $Q$\newline 
\phantom{---}{\bf MERGE$(P_1,P_2)$} vlo\v z na konec $Q$\newline 
{\bf enddo\newline 
V\'ystup}: posloupnost z $Q$
\bigskip

{\bf MERGE$(P_1=(a_1,a_2,\dots,a_n),P_2=(b_1,b_2,\dots,b_m))$}:\newline 
$P:=$ je pr\'azdn\'a posloupnost, $i:=1$, $j:=1$, $k:=1$\newline 
{\bf while} $i\le n$ a $j\le m$ {\bf do\newline 
\phantom{{\rm ---}}if} $a_i<b_j$ {\bf then\newline 
\phantom{{\rm ------}}$c_k:=a_i$}, $i:=i+1$, $k:=k+1$\newline 
\phantom{---}{\bf else\newline 
\phantom{{\rm ------}}$c_k:=b_j$}, $j:=j+1$, $k:=k+1$\newline 
\phantom{---}{\bf endif\newline 
enddo\newline 
while} $i\le n$ {\bf do\newline 
\phantom{{\rm ---}}$c_k:=a_i$}, $i:=i+1$, $k:=k+1$\newline 
{\bf enddo\newline 
while} $j\le m$ {\bf do\newline 
\phantom{{\rm ---}}$c_k:=b_j$}, $j:=j+1$, $k:=k+1$\newline 
{\bf enddo\newline 
V\'ystup:} $P=(c_1,c_2,\dots,c_{n+m})$
\bigskip

\flushpar V\v simn\v eme si, \v ze v\v sechny posloupnosti v $Q$ jsou 
rostouc\'\i\ a \v ze sjednocen\'\i m v\v sech jejich prvk\accent23u  
je v\v zdy na za\v c\'atku b\v ehu cyklu {\bf while 
$|Q|>1$ do} mno\v zina $\{a_i\mid i=1,2,\dots,n\}$. 
Proto\v ze po\v cet 
posloupnost\'\i\ ve front\v e $Q$ je nejv\'y\v se roven d\'elce vstupn\'\i\ 
posloupnosti a ka\v zd\'y pr\accent23ub\v eh tohoto 
cyklu zmen\v s\'\i\ jejich po\v cet o $1$, je algoritmus {\bf MERGE\-SORT} korektn\'\i .
\medskip

\flushpar Spo\v c\'\i t\'ame \v casovou slo\v zitost {\bf MERGESORTU}. 
Nejprve vy\v set\v r\'\i me slo\-\v zi\-tost podprocedury {\bf MER\-GE}. Proto\v ze 
ur\v cen\'\i\ prvku $c_k$ vy\v za\-du\-je \v cas $O(1)$ (provede se 
nejv\'y\v se jedno porovn\'an\'\i ) a 
proto\v ze maxim\'aln\'\i\ hodnota $k$ je $n+m$, dost\'av\'ame, \v ze 
podprocedura {\bf MERGE} vy\v zaduje \v cas $O(n+m)$ (nejv\'y\v se $
n+m$ 
porovn\'an\'\i ), kde $n$ a $m$ jsou d\'elky vstupn\'\i ch posloupnost\'\i .
\medskip

\flushpar Nyn\'\i\ vypo\v cteme slo\v zitost hlavn\'\i\ procedury. 
Z\v rejm\v e prvn\'\i\ cyklus vy\v zaduje line\'arn\'\i\ \v cas. Vy\v set\v r\'\i me 
druh\'y cyklus prob\'\i haj\'\i c\'\i\ p\v res frontu $Q$. P\v redpokl\'adejme, \v ze p\v red prvn\'\i m b\v ehem 
tohoto cyklu je na vrcholu $Q$ speci\'aln\'\i\ znak $\natural$, kter\'y se v\v zdy 
pouze p\v renese z vrcholu $Q$ na jej\'\i\ konec. Proto\v ze mezi dv\v ema 
p\v renosy $\natural$ projde ka\v zd\'y prvek vstupn\'\i\ posloupnosti 
podprocedurou {\bf MERGE} pr\'av\v e jednou, vy\v zaduj\'\i\ jednotliv\'e b\v ehy cyklu 
\v cas $O(n)$, kde 
$n$ je d\'elka vstupn\'\i\ posloupnosti (a z\'arove\v n sou\v cet v\v sech d\'elek 
posloupnost\'\i\ v $Q$). V\v sech\-ny posloupnosti v $Q$ maj\'\i\  
na po\v c\'atku d\'elku $\ge 1$. Odtud jednoduchou indukc\'\i\ 
dostaneme, \v ze po $i$-t\'em p\v renosu znaku $\natural$ maj\'\i\ 
d\'elku $\ge 2^{i-1}$. Proto po\v cet p\v renos\accent23u je 
nejv\'y\v se $\lceil\log_2n\rceil$, a tedy algoritmus {\bf MERGESORT }
vy\v zaduje \v cas $O(n\log n)$ (provede se nejv\'y\v se $n\log n$ 
porovn\'an\'\i ).
\medskip

\flushpar Vzhledem k 
po\v ctu porovn\'an\'\i\ je {\bf MERGESORT} optim\'aln\'\i\ t\v r\'\i dic\'\i\ algoritmus. 
Nav\'\i c v t\'eto verzi je adaptivn\'\i\ na p\v redt\v r\'\i d\v en\'e 
posloupnosti, kter\'e maj\'\i\ jen mal\'y po\v cet dlouh\'ych 
set\v r\'\i d\v en\'ych \'usek\accent23u (b\v eh\accent23u). P\v ri 
konstantn\'\i m po\v ctu b\v eh\accent23u m\'a slo\v zitost $O(n)$. Jin\'a 
jeho verze, kter\'a za\v c\'\i n\'a sl\'ev\'an\'\i\ v\v zdy od 
jednoprvkov\'ych posloupnost\'\i\ (tzv. p\v r\'\i m\'y {\bf MERGESORT}) 
tuto vlastnost nem\'a. 
\bigskip

\head
QUICKSORT
\endhead

\flushpar Nyn\'\i\ pop\'\i\v seme patrn\v e v\accent23ubec 
nejpou\v z\'\i van\v ej\v s\'\i\ t\v r\'\i dic\'\i\ algoritmus, kter\'ym je {\bf QUICKSORT}. 
D\accent23uvodem je, \v ze pro obecnou posloupnost je 
nejrychlej\v s\'\i , p\v ri rovnom\v ern\'em rozlo\v zen\'\i\ vstupn\'\i ch 
polsoupnost\'\i\ m\'a nejmen\v s\'\i\ o\v cek\'avan\'y \v cas. 
\bigskip

{\bf Quick$(a_i,a_{i+1},\dots,a_j)$}:\newline 
{\bf if} $i=j$ {\bf then\newline 
\phantom{{\rm ---}}V\'ystup}: $(a_i)$\newline 
{\bf else\newline 
\phantom{{\rm ---}}}zvol $k$ takov\'e, \v ze $i\le k\le j$, $a:=a_k$, 
vym\v e\v n $a_i$ a $a_k$, $l:=i+1$, $q:=j$\newline 
\phantom{---}{\bf while} true {\bf do\newline 
\phantom{{\rm ------}}while} $a_l<a$ {\bf do} $l:=l+1$ {\bf enddo\newline 
\phantom{{\rm ------}}while} $a_q>a$ {\bf do} $q:=q-1$ {\bf enddo\newline 
\phantom{{\rm ------}}if} $l\ge q$ {\bf then\newline 
\phantom{{\rm ---------}}}exit\newline 
\phantom{------}{\bf else\newline 
\phantom{{\rm ---------}}}vym\v e\v n $a_l$ a $a_q$, $l:=l+1$, $q
:=q-1$\newline 
\phantom{{\rm------}}{\bf endif\newline 
\phantom{{\rm ---}}enddo\newline 
\phantom{{\rm---}}if} $i+1=l$ {\bf then\newline 
\phantom{{\rm ------}}V\'ystup$(a,$Quick$(a_{q+1},a_{q+2},\dots,a_
j))$\newline 
\phantom{{\rm ---}}else\newline 
\phantom{{\rm ------}}if} $j=q$ {\bf then\newline 
\phantom{{\rm ---------}}V\'ystup$($Quick$(a_{i+1},a_{i+2},\dots,
a_{l-1}),a)$\newline 
\phantom{{\rm ------}}else\newline 
\phantom{{\rm ---------}}V\'ystup$($Quick$(a_{i+1},a_{i+2},\dots,
a_{l-1}),a,$Quick$(a_{q+1},\dots,a_j))$\newline 
\phantom{{\rm ------}}endif\newline 
\phantom{{\rm ---}}endif\newline 
endif
\bigskip

QUICKSORT$(a_1,a_2,\dots,a_n)$}:\newline 
{\bf V\'ystup$($Quick$(a_1,a_2,\dots,a_n))$
\bigskip

}\flushpar Algoritmus {\bf Quick} set\v r\'\i d\'\i\ posloupnost 
$(a_i,a_{i+1},\dots,a_j)$ tak, \v ze pro prvek $a=a_k$ vytvo\v r\'\i\ posloupnost 
$(a_i,a_{i+1},\dots,a_{l-1})$ v\v sech prvk\accent23u men\v s\'\i ch ne\v z $
a$ 
a posloupnost
$(a_{q+1},\dots,a_j)$ v\v sech prvk\accent23u 
v\v et\v s\'\i ch ne\v z $a$. Na tyto posloupnosti pak 
zavol\'a s\'am sebe a do v\'ysledn\'e posloupnosti ulo\v z\'\i\ 
nejprve set\v r\'\i d\v enou prvn\'\i\ posloupnost, pak prvek $a$ a nakonec 
set\v r\'\i d\v enou druhou posloupnost. Korektnost procedury 
{\bf Quick}  i algoritmu {\bf QUICKSORT} je tedy z\v rejm\'a, proto\v ze 
$l\le j$ a $i\le q$. 
\medskip

\flushpar Procedura {\bf Quick} bez rekurzivn\'\i ho vol\'an\'\i\ 
vy\v zaduje \v cas $O(j-i)$. Tedy kdyby $a_k$ byl medi\'an (tj. 
prost\v redn\'\i\ prvek) posloupnosti $(a_i,a_{i+1},\dots,a_j)$, pak 
by algoritmus {\bf QUICKSORT} v nejhor\v s\'\i m p\v r\'\i pad\v e vy\-\v zadoval \v cas $
O(n\log n)$. Jak 
uvid\'\i me pozd\v eji, medi\'an lze sice nal\'ezt v line\'ar\-n\'\i m \v case, ale 
pou\v zit\'\i\ jak\'ekoli procedury pro jeho nalezen\'\i\ m\'a za 
n\'asledek, \v ze algoritmy {\bf MERGESORT} a {\bf HEAPSORT }
budou rychlej\v s\'\i\ (nikoliv asymptoticky, ale 
multiplikativn\'\i\ konstanta bude v tomto p\v r\'\i pad\v e vysok\'a). Proto je t\v reba 
vybrat prvek $a_k$ (tzv. pivot) co 
nejrychleji. P\accent23uvodn\v e se bral prvn\'\i\ nebo 
posledn\'\i\ prvek posloupnosti. P\v ri t\'eto volb\v e a p\v ri 
rovnom\v ern\'em rozd\v elen\'\i\ vstup\accent23u je 
o\v cek\'avan\'y \v cas {\bf QUICKSORTU} $O(n\log n)$ a algoritmus je 
obvykle rychlej\v s\'\i\ ne\v z {\bf MERGESORT} a 
{\bf HEAPSORT}. Av\v sak \v cas v nejhor\v s\'\i m p\v r\'\i pad\v e je 
kvadratick\'y a dokonce pro ur\v cit\'a rozd\v elen\'\i\ vstupn\'\i ch dat 
je i o\v cek\'avan\'y \v cas kvadratick\'y. 
Proto tuto volbu pivota nen\'\i\ vhodn\'e pou\v z\'\i vat pro \'ulohy, kdy 
nezn\'ame rozd\v elen\'\i\ vstupn\'\i ch dat (mohlo by se st\'at, \v ze je 
nevhodn\'e). Jednodu\v se to lze napravit tak, \v ze budeme volit $
k$ n\'ahodn\v e. Bohu\v zel 
pou\v zit\'\i\ pseudon\'ahodn\'eho gener\'atoru tak\'e vy\v zaduje jist\'y \v cas, a pak 
u\v z by algoritmus zase nemusel b\'yt rychlej\v s\'\i\ ne\v z algoritmy 
{\bf MERGESORT} a {\bf HEAPSORT} (nav\'\i c takto n\'ahodn\v e zvolen\'y 
prvek nen\'\i\ skute\v cn\v e n\'ahodn\'y, ale to v tomto p\v r\'\i pad\v e 
nevad\'\i ). D\accent23usledkem je n\'avrh vyb\'\i rat pivota jako 
medi\'an ze t\v r\'\i\ nebo p\v eti pevn\v e zvolen\'ych prvk\accent23u 
posloupnosti. Praxe uk\'azala, \v ze tento v\'yb\v er pivota je 
nej\-prakti\v ct\v ej\v s\'\i , d\'a se prov\'est rychle a zaji\v s\v tuje 
dostate\v cnou n\'a\-hod\-nost.
\medskip

\flushpar Proto\v ze p\v ri ka\v zd\'em vol\'an\'\i\ m\'a {\bf Quick} jako 
argument krat\v s\'\i\ 
vstupn\'\i\ posloupnost, lze uk\'azat, \v ze:
\roster
\item{}p\v ri ka\v zd\'e volb\v e 
pivota je nejhor\v s\'\i\ \v cas algoritmu {\bf QUICKSORT} $O(n^2
)$,
\item{}pokud je pivot vybr\'an jedno\-du\-ch\'ym a rychl\'ym 
zp\accent23usobem (to plat\'\i , i kdy\v z se vol\'\i\ n\'ahodn\v e), pak 
exi\-stuj\'\i\ vstupn\'\i\ posloupnosti, kter\'e vy\v zaduj\'\i\ \v cas $
O(n^2)$,
\item{}o\v cek\'avan\'y \v cas je $O(n\log n)$.
\endroster
\medskip

\flushpar N\'asledn\'a anal\'y\-za o\v cek\'avan\'eho p\v r\'\i padu  
je pro n\'a\-hod\-n\v e zvolen\'eho pivota (bez dal\v s\'\i ho p\v red\-pokladu na 
vstupn\'\i\ data) nebo pro p\v r\'\i pad, kdy pivot je pevn\v e zvolen a data jsou 
rovno\-m\v ern\v e rozd\v e\-lena.
\medskip

\flushpar Uk\'a\v zeme dva zp\accent23usoby v\'ypo\v cty o\v cek\'avan\'eho \v casu.  Jeden 
je zalo\v zen na n\v ekolika jednoduch\'ych pozorov\'an\'\i ch a nen\'\i\ v 
n\v em mnoho po\v c\'\i t\'an\'\i , druh\'y na rekurzivn\'\i m v\'ypo\v ctu. Ten 
je po\v cetn\v e n\'aro\v cn\v ej\v s\'\i , ale postup je standardn\'\i . Hlavn\'\i\ idea v obou 
p\v r\'\i padech spo\v c\'\i v\'a v tom, \v ze  
o\v cek\'avan\'y \v cas algoritmu {\bf QUICKSORT} je \'um\v ern\'y 
o\v cek\'avan\'emu po\v ctu porovn\'an\'\i\ v algoritmu {\bf QUICKSORT}.
Tento fakt plyne p\v r\'\i mo z popisu algoritmu.  Budeme 
tedy po\v c\'\i tat  
o\v cek\'a\-van\'y po\v cet porovn\'an\'\i\ pro algoritmus {\bf QUICKSORT}.  
\medskip

\flushpar Prvn\'\i\ zp\accent23usob v\'ypo\v ctu:\newline 
Ka\v zd\'e dva prvky $a_i$ a $a_j$ algoritmus {\bf QUICKSORT} porovn\'a 
p\v ri t\v r\'\i d\v en\'\i\ posloupnosti $(a_1,a_2,\dots,a_n)$ nejv\'y\v se jednou, 
p\v ri\v cem\v z kdy\v z porovn\'av\'a $a_i$ a $a_j$, 
pak pro n\v ejak\'y b\v eh podprocedury {\bf Quick} je $a_i$ nebo $
a_j$ 
pivot, ale v p\v redchoz\'\i ch b\v ez\'\i ch {\bf Quick} $a_i$ ani $
a_j$ 
nebyl pivotem (proto\v ze pivot je v\v zdy vy\v razen z n\'asleduj\'\i c\'\i ch 
vol\'an\'\i\ t\'eto podprocedury).
\medskip

\flushpar Nech\v t $(b_1,b_2,\dots,b_n)$ je v\'ysledn\'a posloupnost.  
Ozna\v cme $X_{i,j}$ boolskou prom\v enou, kter\'a m\'a hodnotu $
1$, 
kdy\v z {\bf QUICKSORT} provedl porovn\'an\'\i\ mezi prvky $b_i$ a $
b_j$, a 
jinak m\'a hodnotu $0$.  P\v redpokl\'adejme, \v ze je to n\'ahodn\'a 
veli\v cina. Kdy\v z $p_{i,j}$ je prav\-d\v e\-podobnost, \v ze $
X_{i,j}=1$, 
pak o\v cek\'avan\'a hodnota $X_{i,j}$ je 
$$\bold E(X_{i,j})=0(1-p_{i,j})+1p_{i,j}=p_{i,j}.$$
Proto\v ze po\v cet porovn\'an\'\i\  p\v ri b\v ehu algoritmu {\bf QUICKSORT} je
$$\sum_{i=1}^n\sum_{j=i+1}^nX_{i,j}$$
a proto\v ze o\v cek\'avan\'a hodnota sou\v ctu n\'ahodn\'ych 
prom\v enn\'ych je sou\v ctem o\v cek\'avan\'ych hodnot, dost\'av\'ame, \v ze 
o\v cek\'avan\'y po\v cet porovn\'an\'\i\ v algoritmu {\bf QUICKSORT} je
$$\sum_{i=1}^n\sum_{j=i+1}^n\bold E(X_{i,j})=\sum_{i=1}^n\sum_{j=
i+1}^np_{i,j}.$$
\medskip

\flushpar Abychom spo\v c\'\i tali $p_{i,j}$, pop\'\i\v seme chov\'an\'\i\ algoritmu 
{\bf QUICKSORT} pomoc\'\i\ modifikace stromu v\'ypo\v ctu. Bude to bin\'arn\'\i\ 
strom, v n\v em\v z ka\v zd\'y vrchol odpov\'\i d\'a jednomu 
b\v ehu podprocedury {\bf Quick}. Vrchol $v$ bude vnit\v rn\'\i m 
vrcholem, 
kdy\v z odpov\'\i da\-j\'\i\-c\'\i\ podprocedura volila pivota, a tento 
pivot bude ohodnocen\'\i m $v$. V podstromu lev\'eho syna vrcholu $
v$
budou pr\'av\v e v\v sechna n\'asleduj\'\i c\'\i\ rekurzivn\'\i\ vol\'an\'\i\ podprocedury 
{\bf Quick} nad \v c\'ast\'\i\ posloupnosti, kter\'a p\v redch\'az\'\i\ pivotu. 
Analogicky v podstromu 
prav\'eho syna vrcholu $v$ budou pr\'av\v e v\v sechna n\'asleduj\'\i c\'\i\ 
rekurzivn\'\i\ vol\'an\'\i\ procedury {\bf Quick} nad \v c\'ast\'\i\ 
posloupnosti, kter\'a n\'asleduje po pivotu. 
Listy stromu odpov\'\i daj\'\i\ vol\'an\'\i\ procedury {\bf Quick} nad 
jednoprvkov\'ymi posloupnost\-mi a ka\v zd\'y takov\'y jednotliv\'y 
prvek ohodnocuje p\v r\'\i slu\v sn\'y  
list. Kdy\v z vrchol $v$ odpov\'\i d\'a vol\'an\'\i\ {\bf Quick} nad posloupnost\'\i\ 
$(a_i,a_{i+1},\dots,a_j)$, pak vrcholy v podstromu lev\'eho syna $
v$ jsou 
ohodnoceny prvky z posloupnosti $(a_i,a_{i+1},\dots,a_{l-1})$ a vrcholy v 
podstromu prav\'eho syna vrcholu $v$ jsou ohodnoceny prvky z posloupnosti 
$(a_{q+1},\dots,a_j)$ (po p\v rerovn\'an\'\i ).
D\'ale plat\'\i\ $\{a_l\mid i\le l\le j\}=\{b_l\mid i\le l\le j\}$.
\medskip

\flushpar O\v c\'\i slujeme vrcholy tohoto stromu prohled\'av\'an\'\i m do 
\v s\'\i\v rky za p\v redpokladu, \v ze lev\'y syn vrcholu p\v redch\'az\'\i\ 
prav\'emu synu. Nech\v t $(c_1,c_2,\dots,c_n)$ je posloupnost 
prvk\accent23u $\{a_i\mid 1\le i\le n\}$ v po\v rad\'\i\ dan\'em t\'\i mto 
o\v c\'\i slov\'an\'\i m. Pak plat\'\i , \v ze $X_{i,j}=1$, pr\'av\v e kdy\v z prvn\'\i\ 
prvek v posloupnosti $(c_1,c_2,\dots,c_n)$ z mno\v zi\-ny 
$\{b_l\mid i\le l\le j\}$ je bu\v d $b_i$ nebo $b_j$. Pravd\v epodobnost tohoto 
jevu je $\frac 2{j-i+1}$, tedy $p_{i,j}=\frac 2{j-i+1}$ pro $1\le 
i<j\le n$. Odtud 
o\v cek\'avan\'y po\v cet porovn\'an\'\i\ v algoritmu {\bf QUICKSORT} je
$$\sum_{i=1}^n\sum_{j=i+1}^np_{i,j}=\sum_{i=1}^n\sum_{j=i+1}^n\frac 
2{j-i+1}=\sum_{i=1}^n\sum_{k=2}^{n-i+1}\frac 2k\le 2n(\sum_{k=2}^
n\frac 1k)\le 2n\int_1^n\frac 1xdx=2n\ln n.$$
\medskip

\flushpar Druh\'y zp\accent23usob v\'ypo\v ctu:\newline 
Ozna\v cme $QS(n)$ o\v cek\'avan\'y po\v cet 
porovn\'an\'\i\ proveden\'ych algoritmem {\bf QUICKSORT} p\v ri 
t\v r\'\i d\v en\'\i\ $n$-\v clenn\'e posloupnosti. Pak plat\'\i
$$\gather QS(0)=QS(1)=0\text{\rm\ a }\\
QS(n)=\frac 1n\big(\sum_{k=0}^{n-1}n-1+QS(k)+QS(n-k-1)\big)=n-1+\frac 
2n(\sum_{k=0}^{n-1}QS(k)).\endgather$$
Z toho dost\'av\'ame, \v ze 
$$nQS(n)=n(n-1)+2\sum_{k=0}^{n-1}QS(k).$$
P\v rep\'\i\v seme je\v st\v e jednou tuto rovnici s $n+1$ m\'\i sto $
n$:
$$(n+1)QS(n+1)=(n+1)n+2\sum_{k=0}^nQS(k).$$
Od t\'eto rovnice ode\v cteme rovnici p\v redchoz\'\i\ a po 
jednoduch\'e \'uprav\v e z\'\i sk\'ame rekurentn\'\i\ vztah
$$QS(n+1)=\frac {2n}{n+1}+\frac {n+2}{n+1}QS(n).$$
Postupn\'ym dosazov\'an\'\i m dostaneme \v re\v sen\'\i\ 
$$\align QS(n)=&\sum_{i=2}^n\frac {n+1}{i+1}\frac {2(i-1)}i\le 2(
n+1)\big(\sum_{i=2}^n\frac 1{i+1}\big)=2(n+1)\big(\sum_{i=3}^{n+1}\frac 
1i\big)=\\
&2(n+1)\big(\sum_{i=2}^{n+1}\frac 1i-\frac 12\big)\le 2(n+1)\big(
(\int_{i=1}^{n+1}\frac 1xdx)-\frac 12\big)=\\
&2n\ln(n+1)+2\ln(n+1)-n-1.\endalign$$
Pro dostate\v cn\v e velk\'a $n$ tedy plat\'\i\ 
$$2n\ln(n+1)+2\ln(n+1)-n\le 2n\ln n.$$
\bigskip

\head
Porovn\'an\'\i\ t\v r\'\i dic\'\i ch algoritm\accent23u
\endhead

\flushpar Nyn\'\i\ porovn\'ame slo\v zitost algoritm\accent23u 
{\bf HEAPSORT}, {\bf MERGE\-SORT}, {\bf QUICKSORT}, {\bf A-sort} (byl pops\'an 
v kapitole o $(a,b)-$stromech), 
{\bf SELECTIONSORT} a {\bf INSERTIONSORT}.  P\v ripome\v nme si, \v ze 
{\bf SELECTIONSORT} t\v r\'\i\-d\'\i\ posloupnost tak, \v ze jedn\'\i m
pr\accent23ucho\-dem nalezne jej\'\i\ nej\-men\v s\'\i\ prvek, 
kter\'y vy\v rad\'\i\ a vlo\v z\'\i\ do v\'ysledn\'e 
posloupnosti (ve verzi, kter\'a t\v r\'\i d\'\i\ na m\'\i st\v e, ho vym\v en\'\i\ 
s lev\'ym krajn\'\i m prvkem pole). 
Tento proces pak opakuje se zbytkem 
p\accent23uvodn\'\i\ posloupnosti. Tato idea byla 
z\'akla\-dem algoritmu {\bf HEAPSORT}.  {\bf INSERTIONSORT} t\v r\'\i d\'\i\ 
tak, \v ze do ji\v z set\v r\'\i d\v en\'eho za\v c\'atku  
posloupnosti vkl\'ad\'a dal\v s\'\i\ prvek, kter\'y pomoc\'\i\ v\'ym\v en 
za\v rad\'\i\ na spr\'avn\'e m\'\i sto, a tento proces (za\v c\'\i n\'a druh\'ym 
prvkem zleva) opakuje.  
\medskip

\flushpar {\bf QUICKSORT} v nejhor\v s\'\i m p\v r\'\i pad\v e vy\v zaduje \v cas 
$\Theta (n^2)$, o\v ce\-k\'avan\'y \v cas je $9n\log n$, v 
nej\-hor\v s\'\i m p\v r\'\i pad\v e prov\'ad\'\i\ $\frac {n^2}2$ porov\-n\'an\'\i ,  
o\v cek\'avan\'y po\v cet porov\-n\'an\'\i\ je $1.44n\log n$. 
Pot\v rebuje $n+\log n+konst$ pam\v eti, pou\v z\'\i v\'a p\v r\'\i m\'y 
p\v r\'\i stup k pam\v eti a nen\'\i\ adaptivn\'\i\ na p\v redt\v r\'\i d\v en\'e 
posloupnosti.\newline 
{\bf HEAPSORT} v nejhor\v s\'\i m p\v r\'\i pad\v e vy\v zaduje \v cas $
20n\log n$, 
o\v ce\-k\'avan\'y \v cas je $\le 20n\log n$, v nejhor\v s\'\i m i v 
o\v cek\'avan\'em p\v r\'\i pad\v e prov\'ad\'\i\ $2n\log n$ porovn\'an\'\i . 
Pot\v rebuje $n+konst$ pam\v eti, pou\v z\'\i v\'a p\v r\'\i m\'y p\v r\'\i stup k pam\v eti a 
nen\'\i\ adaptivn\'\i\ na p\v redt\v r\'\i d\v en\'e posloupnosti.\newline 
{\bf MERGESORT} v nejhor\v s\'\i m p\v r\'\i pad\v e vy\v zaduje \v cas 
$12n\log n$, o\v cek\'avan\'y \v cas je $\le 12n\log n$, v nejhor\v s\'\i m 
i v o\v cek\'ava\-n\'em p\v r\'\i pad\v e prov\'ad\'\i\ $n\log n$ 
porovn\'an\'\i\ (nejmen\v s\'\i\ mo\v zn\'y po\v cet). Pot\v rebuje $
2n+konst$ 
pam\v eti, pou\v z\'\i v\'a 
sekven\v cn\'\i\ p\v r\'\i stup k pam\v eti a m\'a verzi, kter\'a je 
adaptivn\'\i\ na p\v redt\v r\'\i d\v en\'e posloupnosti s mal\'ym po\v ctem 
b\v eh\accent23u.\newline 
{\bf A-sort} v nejhor\v s\'\i m p\v r\'\i pad\v e i v o\v cek\'an\'em p\v r\'\i pad\v e vy\v zaduje \v cas 
$O(n\log\frac Fn)$, kde $F$ je po\v cet inverz\'\i\ ve vstupn\'\i\ 
posloupnosti,  v nejhor\v s\'\i m i v o\v cek\'avan\'em 
p\v r\'\i pad\v e prov\'ad\'\i\ $O(n\log\frac Fn)$ porovn\'an\'\i . Pot\v rebuje  
$5n+konst$ pam\v eti, pou\v z\'\i v\'a p\v r\'\i m\'y p\v r\'\i stup k pam\v eti 
a je adaptivn\'\i\ na p\v redt\v r\'\i d\v en\'e posloupnosti s mal\'ym 
po\v ctem inverz\'\i .\newline 
{\bf SELECTIONSORT} v nejhor\v s\'\i m i v o\v cek\'avan\'em p\v r\'\i pad\v e 
vy\v zaduje \v cas $2n^2$, po\v cet porov\-n\'an\'\i\ v nejhor\v s\'\i m 
i v o\v cek\'avan\'em p\v r\'\i pad\v e je $\frac {n^2}2$. Pot\v rebuje 
$n+konst$ pam\v eti, pou\v z\'\i v\'a p\v r\'\i m\'y p\v r\'\i stup k pam\v eti a 
nen\'\i\ adaptivn\'\i\ na p\v redt\v r\'\i d\v en\'e posloupnosti.\newline 
{\bf INSERTIONSORT} v nejhor\v s\'\i m i v o\v cek\'avan\'em p\v r\'\i pad\v e  
vy\-\v zaduje \v cas $O(n^2)$, po\v cet porov\-n\'an\'\i\ v nejhor\v s\'\i m 
p\v r\'\i pad\v e je $\frac {n^2}2$, v o\v ce\-k\'a\-van\'em p\v r\'\i pad\v e 
$\frac {n^2}4$. Pot\v rebuje $n+konst$ pam\v eti, pou\v z\'\i\-v\'a 
sekven\v cn\'\i\ p\v r\'\i stup k pam\v eti a m\'a verzi, kter\'a je 
adaptivn\'\i\ na p\v redt\v r\'\i d\v en\'e posloupnosti s mal\'ym po\v ctem 
inverz\'\i .
\medskip

\flushpar Prezentovan\'e  v\'ysledky byly spo\v c\'\i t\'any 
pro model RAM (viz Mehlhorn 1984).
\medskip

\flushpar O\v cek\'avan\'y \v cas pro {\bf HEAPSORT} je prakticky stejn\'y jako 
jeho nej\-hor\v s\'\i\ \v cas.  Byly navr\v zeny verze, kter\'e optimalizuj\'\i\ 
po\v cet porov\-n\'an\'\i , ale v\v et\v sinou maj\'\i\ v\v et\v s\'\i\ n\'aroky na 
\v cas, a proto a\v z na v\'yjimky ne\-jsou pro praktick\'e 
pou\v zit\'\i\ vhodn\'e.  
Situace pro {\bf MERGESORT} je komplikovan\v ej\v s\'\i , hod\-n\v e z\'avis\'\i\ 
na konkr\'etn\'\i\ verzi algoritmu.  Algoritmus 
{\bf MER\-GE\-SORT} je nejvhodn\v ej\v s\'\i\ pro extern\'\i\ pam\v eti se  
sekven\v cn\'\i m p\v r\'\i stupem k dat\accent23um, pro intern\'\i\ 
pam\v e\v t kv\accent23uli velk\'e prostorov\'e n\'aro\v cnosti 
nen\'\i\ doporu\v cov\'an (je nap\v r. dvojn\'asobn\'a proti {\bf HEAPSORTU }
a t\'em\v e\v r dvojn\'asobn\'a proti {\bf QUICKSORTU}). Tak\'e se hod\'\i\ pro 
n\'avrh pa\-ra\-leln\'\i ch algoritm\accent23u. Pro t\v r\'\i d\v en\'\i\ 
kr\'atk\'ych posloupnost\'\i\ je doporu\v cov\'ano m\'\i sto {\bf QUICKSORTU }
pro posloupnosti d\'elky $\le 22$ pou\v z\'\i t {\bf SELECTIONSORT} a pro 
posloupnosti d\'elky $\,\le 15$  {\bf INSERTIONSORT.}
To vede k n\'avrhu optimalizovan\'eh {\bf QUICKSORTU}, 
kter\'y{\bf ,}  kdy\v z 
vol\'a rekurzivn\v e s\'am sebe na kr\'atkou posloupnost, pak 
pou\v zije {\bf SELECTIONSORT} ne\-bo {\bf INSERTIONSORT}. V algoritmu {\bf A-sort }
se doporu\-\v cu\-je pou\v z\'\i t $(2,3)$-strom.  
Pom\v er \v cas\accent23u spot\v rebovan\'ych  
algoritmy {\bf QUICKSORT},  {\bf MERGESORT} a {\bf HEAPSORT} na 
klasick\'ych po\v c\'\i ta\v c\'\i ch uv\'ad\'\i\ Mehlhorn (1984) jako  
$1$ : $1.33$ : $2.22$. To 
v\v sak nemus\'\i\  b\'yt pravda pro sou\v casn\'e procesory, 
pam\v eti a opera\v cn\'\i\ syst\'emy. 
\bigskip

\head
Sl\'ev\'an\'\i\ nestejn\v e dlouh\'ych posloupnost\'\i
\endhead

\flushpar V algoritmu {\bf MERGESORT} jsme pou\v zili frontu, kter\'a 
\v r\'\i dila proces slu\v cov\'an\'\i\ rostouc\'\i ch posloupnost\'\i .  Tato metoda je uspokojuj\'\i c\'\i\ a 
d\'av\'a optim\'aln\'\i\ v\'ysledek (ve smyslu \v casov\'e n\'aro\v cnosti){\bf ,} pokud posloupnosti ve front\v e jsou 
stejn\v e dlouh\'e.  Pokud se ale jejich d\'elky hodn\v e li\v s\'\i , 
nedos\'ahneme t\'\i mto zp\accent23usobem  
optim\'aln\'\i ho v\'ysledku.  P\v ritom r\accent23uzn\'e verze tohoto 
probl\'emu se vyskytuj\'\i\ v mnoha \'uloh\'ach. Jednou z prvn\'\i ch \'uloh, 
kde jsme se s n\'\i m setkali, je konstrukce 
Huffmanova k\'odu -- to je minim\'aln\'\i\ redundantn\'\i\ k\'od, kter\'y byl 
nalezen v roce 1952. K optim\'aln\'\i mu \v re\v sen\'\i\ vede nap\v r. 
postup, kter\'y je kombinac\'\i\ `mergeov\'an\'\i ' a 
optimalizace a pou\v z\'\i v\'a metody dynamick\'eho programov\'an\'\i .  
Nejprve form\'aln\v e pop\'\i\v seme abstraktn\'\i\ verzi tohoto probl\'emu.  
\medskip

\flushpar Vstup: Mno\v zina rostouc\'\i ch navz\'ajem 
disjunktn\'\i ch posloupnost\'\i .\newline 
\'Ukol:  Pomoc\'\i\ operace {\bf MERGE} co nejrychleji spojit 
v\v sechny tyto posloupnosti do jedin\'e rostouc\'\i\ posloupnosti.  
\medskip

\flushpar P\v redpokl\'adejme, \v ze m\'ame postup, kter\'y z 
dan\'ych rostouc\'\i ch posloupnost\'\i\ vytvo\v r\'\i\ 
jedinou rostouc\'\i\ po\-sloupnost. Tento postup ur\v cuje \'upln\'y 
bin\'arn\'\i\ strom $T$, jeho\v z listy jsou ohodnoceny vstupn\'\i mi 
posloupnostmi a ka\v zd\'y vnit\v rn\'\i\ vr\-chol je ohodnocen 
posloupnost\'\i , kter\'a je slou\v cen\'\i m vstupn\'\i ch posloupnost\'\i\ 
ohodnocuj\'\i c\'\i ch listy v podstromu ur\v cen\'em t\'\i mto 
vr\-cholem. Tedy ko\v ren je ohodnocen v\'ystupn\'\i\ posloupnost\'\i . 
Form\'aln\v e pro ka\v zd\'y vnit\v rn\'\i\ vr\-chol  $v$ plat\'\i :
\roster
\item"{}"
kdy\v z  $v_1$ a $v_2$ jsou synov\'e $v$ a $P(v)$ je posloupnost 
ohodnocuj\'\i c\'\i\ vrchol $v$,  pak\newline 
$P(v)=${\bf MERGE$(P(v_1),P(v_2))$}.
\endroster
Ozna\v cme $l(P)$ d\'elku posloupnosti $P$. Pak sou\v cet 
\v cas\accent23u, kter\'e v tomto procesu vy\v zaduje 
podprocedura {\bf MERGE},
je $O(\sum \{l(P(v))\mid v\text{\rm \ je vnit\v rn\'\i\ vrchol stromu }
T\})$. Indukc\'\i\ 
lehce dostaneme, \v ze 
$$\sum \{l(P(v))\mid v\text{\rm \ vnit\v rn\'\i\ vrchol stromu }T
\}=\sum_{\{t\text{\rm \ je list }T\}}d(t)l(P(t)),$$
kde $d(t)$ je hloubka listu $t$.
\medskip

\flushpar Kdy\v z tedy $T$ je \'upln\'y bin\'arn\'\i\ strom, jeho\v z listy
jsou ohodnoceny navz\'ajem disjunktn\'\i mi rostouc\'\i mi  
posloupnostmi, pak n\'asleduj\'\i c\'\i\ algoritmus {\bf Slevani} spoj\'\i\ tyto  
posloupnosti do jedin\'e rostouc\'\i\ posloupnosti a procedury 
{\bf MERGE} budou vy\v zadovat celkov\'y \v cas 
$$O(\sum_{\{t\text{\rm \ je list }T\}}d(t)l(P(t))).$$
\bigskip

{\bf Slevani$(T,\{P(l)\mid l\text{\rm \ je list }T\})$\newline 
while} $P(\text{\rm ko\v ren }T)$ nen\'\i\ definov\'ano {\bf do\newline 
\phantom{{\rm---}}}$v:=$ vrchol $T$ takov\'y, \v ze $P(v)$ nen\'\i\ 
definov\'ano a\newline 
\phantom{---}pro oba syny $v_1$ a $v_2$ vrcholu $v$ jsou $P(v_1)$ a $
P(v_2)$ definov\'any\newline 
\phantom{---}$P(v):=${\bf MERGE$(P(v_1),P(v_2))$\newline 
enddo
\bigskip

}\flushpar Nyn\'\i\ m\accent23u\v zeme p\v reformulovat p\accent23uvodn\'\i\ 
probl\'em:\newline 
Vstup: $n$ \v c\'\i sel $x_1,x_2,\dots,x_n$\newline 
V\'ystup: \'upln\'y bin\'arn\'\i\ strom $T$ s $n$ listy a bijekce $
\phi$ z 
mno\v ziny $\{1,2,\dots,n\}$ do list\accent23u $T$ takov\'a, \v ze 
$\sum_{i=1}^nd(\phi (i))x_i$ je minim\'aln\'\i\ (kde $d(\phi (i))$ je hloubka 
listu $\phi (i)$).\newline 
\v Rekneme, \v ze dvojice $(T,\phi )$ je \emph{optim\'aln\'\i} 
\emph{strom} vzhledem k $x_1,x_2,\dots,x_n$. 
\medskip

\flushpar V p\v reformulov\'an\'e \'uloze u\v z nepracujeme s 
posloupnostmi, ale jen s jejich d\'elkami. 
To znamen\'a, \v ze kdy\v z pro p\accent23uvodn\'\i\ \'ulohu 
byly vstupem posloupnosti $P_1,P_2,\dots,
P_n$, pak pro 
p\v reformulovanou \'ulohu jsou vstupem jen d\'elky  
$l(P_1),l(P_2),\dots,l(P_n)$. Strom vytvo\v ren\'y pro 
p\v reformulovanou \'ulohu  
je pou\v zit v algoritmu {\bf Slevani}  
tak, \v ze  posloupnost $P_i$ ohodnocuje list, kter\'y byl v 
p\v reformulovan\'e \'uloze ohodnocen d\'elkou $l(P_i)$, a hledan\'a posloupnost 
v p\accent23uvodn\'\i\ \'uloze ohodnocuje ko\v ren stromu. 
\medskip

\flushpar M\v ejme mno\v zinu $\{x_i\mid i=1,2,\dots,n\}$.  Pro \'upln\'y bin\'arn\'\i\ 
strom $T$ s $n$ listy a bijekci $\phi$ z mno\v ziny $\{1,2,\dots,
n\}$ do 
list\accent23u stromu $T$ definujme 
$$\Cont(T,\phi )=\sum_{i=1}^nd(\phi (i))x_i,$$
kde $d(\phi (i))$ je hloubka listu $\phi (i)$, tj.  d\'elka cesty z 
ko\v rene do listu $\phi (i)$  pro $i=1,2,\dots,n$.  
Chceme zkonstruovat \'upln\'y bin\'arn\'\i\ strom s $n$ listy, 
kter\'y minimalizuje hodnotu $\Cont$. K \v re\v sen\'\i\ 
pou\v zijeme n\'asleduj\'\i c\'\i\ algoritmus, kter\'y je upravenou verz\'\i\ 
hladov\'eho algoritmu pro n\'a\v s probl\'em.
\medskip

{\bf Optim$(x_1,x_2,\dots x_n)$}:\newline 
$V$ je mno\v zina $n$ jednoprvkov\'ych strom\accent23u\newline 
$\phi$ je bijekce mezi $\{1,2,\dots,n\}$ a mno\v zinou $V$\newline 
{\bf for every} $v\in V$ {\bf do} $c(v):=x_{\phi^{-1}(v)}$ {\bf enddo\newline 
while} $|V|>1$ {\bf do\newline 
\phantom{{\rm---}}}vezmi z $V$ dva stromy $v_1$ a $v_2$ s nejmen\v s\'\i m ohodnocen\'\i m\newline 
\phantom{---}odstra\v n je z $V$\newline 
\phantom{---}vytvo\v r nov\'y strom $v$ spojen\'\i m strom\accent23u $
v_1$ a $v_2$\newline 
\phantom{---}$c(v):=c(v_1)+c(v_2)$, strom $v$ vlo\v z do $V$\newline 
{\bf enddo\newline 
V\'ystup}: $(T,\phi )$, kde $T$ je strom v mno\v zin\v e $V$
\bigskip

\flushpar Vytvo\v ren\'\i\ nov\'eho stromu $v$ spojen\'\i m strom\accent23u 
$v_1$ a $v_2$ znamen\'a vytvo\v ren\'\i\ nov\'eho vrcholu, kter\'y bude 
ko\v renem stromu $v$ a jeho\v z synov\'e budou ko\v reny strom\accent23u 
$v_1$ a $v_2$. To je analogick\'e procedu\v re {\bf spoj}.
\medskip

\proclaim{V\v eta}Pro danou posloupnost \v c\'\i sel 
$(x_1,x_2,\dots,x_n)$ algoritmus {\bf Optim} nalezne optim\'aln\'\i\ strom 
pro mno\v zinu $x_1,x_2,\dots,x_n$ a pokud je posloupnost 
$(x_1,x_2,\dots,x_n)$ neklesaj\'\i c\'\i , pak vy\v za\-du\-je \v case 
$O(n)$.  
\endproclaim

\demo{D\accent23ukaz}D\accent23ukaz m\'a dv\v e \v c\'asti. V 
prvn\'\i\ dok\'a\v zeme korektnost algoritmu a ve druh\'e pop\'\i\v seme 
reprezentaci mno\v ziny $V$ a vypo\v cteme \v casovou slo\v zitost.
\medskip

\flushpar Nejprve p\v ripome\v nme, \v ze $\phi (i)$ je list $T$ 
pro ka\v zd\'e $i\in \{1,2,\dots,n\}$.  Proto\v ze na za\v c\'atku $
V$ 
obsahuje jen 
jednoprvko\-v\'e stromy, tak tvrzen\'\i\ plat\'\i .  Ka\v zd\'y b\v eh cyklu 
{\bf while do} zmen\v s\'\i\ po\v cet strom\accent23u $V$ o jeden, ale nezm\v en\'\i\ 
mno\v zinu list\accent23u.  Proto $T$ je strom s $n$ listy, $\phi$ je 
bijekce z $\{1,2,\dots,n\}$ do mno\v ziny list\accent23u $T$ a algoritmus 
v\v zdy kon\v c\'\i .  Dok\'a\v zeme indukc\'\i\ podle $n$, \v ze zkonstruovan\'a 
dvojice $(T,\phi )$ je optim\'aln\'\i\ strom vzhledem k $(x_1,x_2
,\dots,x_n)$.  
Kdy\v z $n=2$, tvrzen\'\i\ z\v rejm\v e plat\'\i .  P\v redpokl\'adejme, \v ze 
plat\'\i\ pro ka\v zdou posloupnost \v c\'\i sel $(y_1,y_2,\dots,$$
y_{n-1})$, a 
nech\v t $x_1\le x_2\le\dots\le x_n$ je neklesaj\'\i c\'\i\ posloupnost \v c\'\i sel.  Bez \'ujmy na 
obecnosti m\accent23u\v zeme p\v redpokl\'adat, \v ze v prvn\'\i m kroku 
algoritmus {\bf Optim} zvolil stromy $\phi (1)$ a $\phi (2)$.  Uva\v zujme mno\v zinu 
$(y_1,y_2,\dots,y_{n-1})$, kde $y_i=x_{i+2}$ pro $i=1,2,\dots,n-2$, 
$y_{n-1}=x_1+x_2$.  Nech\v t $T'$ je strom z\'\i skan\'y ze stromu $
T$  
odstran\v en\'\i m list\accent23u $\phi (1)$ a $\phi (2)$ a nech\v t $
\psi$ je bijekce z 
mno\v ziny $\{1,2,\dots,n-1\}$ takov\'a, \v ze $\psi (i)=\phi (i+
2)$ pro 
$i=1,2,\dots,n-2$ a $\psi (n-1)$ je otec listu $\phi (1)$.  Pak 
m\accent23u\v zeme p\v redpokl\'adat, \v ze algoritmus 
{\bf Optim$(y_1,y_2,\dots,y_{n-1})$} zkonstruoval strom $(T',\psi 
)$, a podle induk\v cn\'\i ho 
p\v redpokladu je to optim\'aln\'\i\ strom pro $(y_1,y_2,\dots,y_{
n-1})$.  Nech\v t 
$(U,\theta )$ je optim\'aln\'\i\ strom vzhledem k $(x_1,x_2,\dots
,x_n)$.  Zvolme 
vnit\v rn\'\i\ vrchol $u$ stromu $U$ takov\'y, \v ze d\'elka cesty z ko\v rene 
do vrcholu $u$ je nej\-v\v et\v s\'\i\ mezi v\v semi vnit\v rn\'\i mi vrcholy 
stromu $U$.  Nech\v t $u_1$ a $u_2$ jsou synov\'e $u$, pak nutn\v e $
u_1$ 
a $u_2$ jsou listy stromu $U$.  Nech\v t $i,j\in \{1,2,\dots,n\}$ takov\'e, \v ze 
$\theta (i)=u_1$, $\theta (j)=u_2$.  Po eventu\'aln\'\i m p\v rejmenov\'an\'\i\ m\accent23u\v zeme 
p\v redpokl\'adat, \v ze kdy\v z $i,j\in \{1,2\}$, pak $i=1$ a 
$j=2$.  Definujme $\eta$ z $\{1,2,\dots,n\}$ do list\accent23u $U$ tak, \v ze 
$\eta (1)=u_1$, $\eta (2)=u_2$, $\eta (i)=\theta (1)$, $\eta (j)=
\theta (2)$ a $\eta (k)=\theta (k)$ pro 
v\v sechna $k\in \{3,4,\dots,n\}\setminus \{i,j\}$.  Pak $\eta$ je bijekce a 
$$\Cont(U,\eta )-\Cont(U,\theta )=(d(u_1)-d(\theta (1))(x_1-x_i)+
(d(u_2)-d(\theta (2))(x_2-x_j).$$
Z volby $u$ plyne, \v ze $d(u_1)\ge d(\theta (1))$, $d(u_2)\ge d(
\theta (2))$, 
$x_1\le x_i$ a $x_2\le x_j$. Odtud  
$$(d(u_1)-d(\theta (1))(x_1-x_i)+(d(u_2)-d(\theta (2))(x_2-x_j)\le 
0$$
a proto\v ze $(U,\theta )$ je optim\'aln\'\i\ strom pro $(x_1,x_2
,\dots,x_n)$, 
dost\'a\-v\'a\-me, \v ze $(U,\eta )$ je tak\'e optim\'aln\'\i\ strom pro 
$(x_1,x_2,\dots,x_n)$.  Od\-stran\v en\'\i m list\accent23u $u_1$ a $
u_2$ ze stromu 
$U$ dostaneme strom $U'$.  Definujme $\tau$ z $\{1,2,\dots,n-1\}$ 
p\v redpisem $\tau (i)=\eta (i+2)$ pro $i=1,2,\dots,n-2$ a $\tau 
(n-1)=u$.  Pak 
$\tau$ je bijekce z $\{1,2,\dots,n-1\}$ do mno\v ziny list\accent23u $
U'$ a 
proto\v ze $(T',\psi )$ je optim\'aln\'\i\ strom pro $(y_1,y_2,\dots
,y_{n-1})$, 
plat\'\i , \v ze 
$$\Cont(T',\psi )\le\Cont(U',\tau ).$$
Proto\v ze  
$$\gather\Cont(T,\phi )=\Cont(T,\psi )+x_1+x_2\text{\rm \ , }\\
\Cont(U,\eta )=\Cont(U',\tau )+x_1+x_2\endgather$$
pak z\'av\v er je, \v ze $(T,\phi )$ je optim\'aln\'\i\ strom pro 
$(x_1,x_2,\dots,x_n)$.
\medskip

\flushpar P\v redpokl\'adejme op\v et, \v ze $x_1\le x_2\le\dots\le 
x_n$ a \v ze v dan\'em 
okam\v ziku jsou $v_1,v_2,\dots,v_k$ postupn\v e vytvo\v ren\'e v\'\i ceprvkov\'e stromy
(tj. strom $v_i$ byl vytvo\v ren p\v red stromem $v_j$, kdy\v z $
i<j$). V 
tomto okam\v ziku je mno\v zina $V$ sjednocen\'\i m mno\v ziny $\{
v_1,v_2,\dots,v_k\}$ 
a mno\v ziny jednoprvkov\'ych strom\accent23u, kter\'e nebyly je\v st\v e 
zpracov\'any. Nyn\'\i\ vytvo\v r\'\i me strom $w$ spojen\'\i m 
strom\accent23u $t_1$ a $t_2$ s nejmen\v s\'\i m ohodnocen\'\i m. Z popisu algoritmu plyne, \v ze kdy\v z 
strom $v_i$ pro $i=1,2,\dots,k$ vznikl spojen\'\i m strom\accent23u $
u_1$ 
a $u_2$, pak $\max\{c(u_1),c(u_2)\}\le\min\{c(t_1),c(t_2)\}$, a proto $
c(w)\ge c(v_i)$ 
pro ka\v zd\'e $i=1,2,\dots,k$. Pak indukc\'\i\ 
okam\v zit\v e dost\'av\'ame, \v ze $c(v_1)\le c(v_2)\le\dots\le 
c(v_k)$.  Tedy sta\v c\'\i , 
abychom m\v eli rostouc\'\i\ posloupnost list\accent23u a v n\'\i\ 
ukazatel na nej\-men\v s\'\i\ list, kter\'y je je\v st\v e 
nezpracovan\'ym jednoprvkov\'ym stro\-mem (tj. p\v red ukazatelem jsou 
listy, kter\'e u\v z nejsou stromy v mno\v zin\v e $V$, 
za ukazatelem jsou listy, kter\'e jsou je\v st\v e jednoprvkov\'e 
stro\-my v mno\v zin\v e $V$) a frontu v\'\i ceprvkov\'ych 
strom\accent23u (z n\'\i\v z stromy ke zpracov\'an\'\i\ odeb\'\i r\'ame 
zp\v redu a nov\v e vytvo\v ren\'e  
ukl\'ad\'ame na konec).  Udr\v zovat tyto struktury vy\v za\-duje \v cas 
$O(1)$ stejn\v e jako nalezen\'\i\ dvou strom\accent23u s nej\-men\v s\'\i m 
ohodnocen\'\i m. M\accent23u\v zeme tedy shrnout, \v ze algoritmus 
{\bf Optim} konstruuje optim\'aln\'\i\ stromy v \v case $O(n)$, kde $
n$ je po\v cet 
zadan\'ych \v c\'\i sel $x_i$. \qed
\enddemo
\bigskip

\flushpar Pro aplikaci na na\v si p\accent23uvodn\'\i\ \'ulohu je 
t\v reba  
je\v st\v e set\v r\'\i dit vstupn\'\i\ posloupnost d\'elek pro p\v reformu\-lo\-vanou \'ulohu.
Tato posloupnost je tvo\v rena p\v rirozen\'ymi \v c\'\i sly a 
k jej\'\i mu set\v r\'\i d\v en\'\i\  
m\accent23u\v zeme pou\v z\'\i t algoritmus {\bf BUCKETSORT} (bude 
pops\'an d\'ale v textu), kter\'y 
vy\v zaduje \v cas $O(n+m)$, kde $n$ je po\v cet posloupnost\'\i\ a $
m$ je 
maxim\'aln\'\i\ d\'elka posloupnosti.
\medskip

\proclaim{V\v eta}Uveden\'y algoritmus mno\v zinu 
disjunktn\'\i ch rostouc\'\i ch 
posloupnost\'\i\ $P_1,P_2,\dots,P_n$ o d\'elk\'ach $l(P_1),l(P_2)
,\dots,l(P_n)$ spoj\'\i\ 
do jedin\'e rostouc\'\i\ posloupnosti v \v case 
$O(\sum_{i=1}^nl(P_i))$.
\endproclaim
\bigskip

\centerline{\bigbigrm{VIII. Rozhodovac\'\i\ stromy}}
\medskip

\flushpar V\v et\v sina obecn\'ych t\v r\'\i dic\'\i ch algoritm\accent23u 
pou\v z\'\i v\'a jedinou primitivn\'\i\ operaci mezi prvky vstupn\'\i\ 
posloupnosti, a to jejich vz\'a\-jemn\'e porov\-n\'an\'\i .  
To znamen\'a, \v ze pr\'aci 
takov\'eho algoritmu lze po\-psat 
bin\'arn\'\i m stro\-mem, jeho\v z vnit\v rn\'\i\ vrcholy jsou ohodnoceny 
porovn\'an\'\i mi dvojic prvk\accent23u vstupn\'\i\ posloupnosti 
(nap\v r. $a_i<a_j$). Bez \'ujmy na obecnosti p\v redpokl\'adejme, \v ze 
vstupn\'\i\ posloupnost je permutace $\pi$
mno\v ziny $\{1,2,\dots,n\}$. Tato permutace proch\'az\'\i\ 
stromem takto:
\roster
\item"{}"
Za\v c\'\i n\'a v ko\v reni stromu. Kdy\v z je ve vnit\v rn\'\i m vrcholu $
v$ 
ohodnocen\'em porovn\'an\'\i m $a_i\le a_j$, pak kdy\v z $\pi (i)
<\pi (j)$, 
pokra\v cuje v lev\'em synu vrcholu $v$, a kdy\v z $\pi (j)<\pi (
i)$, 
pokra\v cuje v prav\'em synu vrcholu $v$. Proces 
t\v r\'\i d\v en\'\i\ kon\v c\'\i , kdy\v z se dostane do listu.
\endroster
Aby byl algoritmus korektn\'\i , mus\'\i\ platit, \v ze dv\v e 
r\accent23uzn\'e permutace skon\v c\'\i\ v r\accent23uzn\'ych 
listech.  Tedy strom popisuj\'\i c\'\i\ korektn\'\i\ algoritmus 
pro set\v r\'\i d\v en\'\i\ $n$-prvkov\'yvh posloupnost\'\i\ mus\'\i\ m\'\i t 
alespo\v n $n!$ list\accent23u.  D\'elka cesty z ko\v rene do listu, 
kde skon\v cila permutace $\pi$, reprezentuje po\v cet porovn\'an\'\i , kter\'e pot\v rebuje 
dan\'y algoritmus k set\v r\'\i d\v en\'\i\ dan\'e posloupnosti $
\pi$. Proto\v ze porovn\'an\'\i\ 
vy\v zaduje alespo\v n jednotku \v casu, dost\'av\'ame t\'\i m i doln\'\i\ odhad na 
\v cas pot\v rebn\'y k set\v r\'\i d\v en\'\i\ t\'eto posloupnosti  algoritmem 
odpov\'\i daj\'\i c\'\i m dan\'emu stromu. Doln\'\i\ odhad  po\v ctu 
porovn\'an\'\i\ i \v casu pro dan\'y algoritmus a v\v sechny  
$n$-prvkov\'e posloupnosti je pak d\'elka nejdel\v s\'\i\ cesty z ko\v rene 
do listu v odpov\'\i daj\'\i c\'\i m stromu. To n\'am umo\v z\v nuje 
z\'\i skat obecn\v e platn\'y doln\'\i\ odhad \v casu pot\v rebn\'eho k set\v r\'\i d\v en\'\i\ 
$n-$prvkov\'e posloupnosti, kter\'ym je  
minimum p\v res v\v sechny bin\'arn\'\i\ stromy s alespo\v n $n!$ 
listy z jejich maxim\'aln\'\i ch d\'elek cest z ko\v rene do listu. 
Korekt\-nost t\v echto \'uvah plyne z pozorov\'an\'\i , 
\v ze kdy\v z porovn\'an\'\i\ je jedin\'a primi\-tivn\'\i\ operace, pak 
algoritmus nen\'\i\ z\'avisl\'y na konkr\'etn\'\i ch prvc\'\i ch vstup\-n\'\i\ 
posloupnosti, ale jen na jejich vz\'ajemn\'em vzta\-hu. 
Proto sta\v c\'\i\ uva\v zovat pouze permutace $n$-prvkov\'e mno\v ziny, 
proto\v ze za\-chycuj\'\i\ v\v sechny mo\v zn\'e vztahy v $n$-prvkov\'e 
posloupnosti. D\'a\-le je t\v reba si uv\v edomit, \v ze vztah mezi 
stromem pro $n$-prvkov\'e posloupnosti a stromem pro 
$(n+1)$-prvkov\'e posloupnosti je d\'an konkr\'etn\'\i m algoritmem a 
ned\'a se popsat obecn\v e. 
\medskip

\flushpar V nevhodn\'em algoritmu se m\accent23u\v ze st\'at, \v ze 
v n\v ekter\'em listu neskon\v c\'\i\ \v z\'adn\'a permutace. To  
nastane, kdy\v z strom pro $n$-prvkov\'e posloupnosti m\'a v\'\i ce 
ne\v z $n!$ list\accent23u, nebo, jinak \v re\v ceno, kdy\v z 
porovn\'an\'\i\ dvou stejn\'ych prvk\accent23u se na n\v ejak\'e cest\v e 
vyskytne alespo\v n dvakr\'at. 
\medskip

\flushpar N\'asleduj\'\i c\'\i\ obr\'azek ilustruje na\v se \'uvahy na 
{\bf SELECTIONSORTU} pro $3$-prvkov\'e posloupnosti. Listy jsou 
ohodnoceny permutacemi vstupn\'\i\ mno\v ziny 
$\{a_1,a_2,a_3\}$, kter\'e v nich 
skon\v c\'\i , nebo jsou pr\'azdn\'e. 
\midinsert
\centerline{\input fig11.tex}
\botcaption{Obr. 1}
\endcaption
\endinsert
\medskip

\definition{Definice}M\v ejme t\v r\'\i dic\'\i\ algoritmus {\bf A}, kter\'y 
jako jedinou pri\-mitivn\'\i\ operaci s prvky vstupn\'\i\ 
posloupnosti pou\v z\'\i v\'a jejich porov\-n\'an\'\i . \v Rekneme, \v ze bin\'arn\'\i\ 
strom $T$, jeho\v z vnit\v rn\'\i\ vrcholy jsou ohodnoceny 
porovn\'an\'\i mi  $a_i\le a_j$ pro $i,j=1,2,\dots,n$, $i\ne j$, je 
\emph{rozhodovac\'\i m} \emph{stromem} algoritmu {\bf A} pro 
$n$-prvkov\'e posloupnosti, kdy\v z pro ka\v zdou permutaci $\pi$
$n$-prvkov\'e mno\v ziny plat\'\i\ 
\roster
\item"{}"
posloupnost porovn\'an\'\i\ p\v ri t\v r\'\i d\v en\'\i\ permutace $
\pi$ 
algoritmem {\bf A} je stejn\'a jako po\-sloupnost porovn\'an\'\i\ p\v ri 
pr\accent23uchodu permutace $\pi$ stromem $T$.
\endroster
\enddefinition
\medskip

\flushpar Pak korektnost algoritmu zaji\v s\v tuje, \v ze dv\v e r\accent23uzn\'e 
permutace mno\v ziny $\{1,2,\dots,n\}$ skon\v c\'\i\ v 
r\accent23uzn\'ych listech stromu $T$ a dol\-n\'\i m odhadem pro 
\v cas algoritmu {\bf A} v nej\-hor\v s\'\i m p\v r\'\i pad\v e je d\'elka nejdel\v s\'\i\ 
cesty z ko\v rene do listu.  
P\v ri rovnom\v ern\'em 
rozd\v elen\'\i\ vstupn\'\i ch po\-sloupnost\'\i\ je o\v cek\'avan\'y \v cas 
algoritmu {\bf A} roven pr\accent23um\v ern\'e d\'elce cesty z ko\v rene do 
listu.  
\medskip

\flushpar Definujme\newline 
$S(n)$ jako minimum p\v res v\v sechny stromy $T$ s alespo\v n $n
!$ listy z 
d\'elek nejdel\v s\'\i ch cest z ko\v rene 
do listu v $T$, \newline 
$A(n)$ jako minimum p\v res v\v sechny stromy $T$ s alespo\v n $n
!$ listy z
pr\accent23um\v ern\'ych d\'elek cest z ko\v rene do listu v $T$.\newline 
Na\v s\'\i m c\'\i lem je spo\v c\'\i tat doln\'\i\ odhady t\v echto veli\v cin.
\medskip

\flushpar Kdy\v z nejdel\v s\'\i\ cesta z ko\v rene do listu v 
bin\'arn\'\i m strom\v e $T$ m\'a d\'elku $k$, pak $T$ m\'a nejv\'y\v se $
2^k$ 
list\accent23u. Proto $n!\le 2^{S(n)}$. Odtud plyne, \v ze $S(n)\ge\log_
2n!$.
P\v ripome\v nme si Stirling\accent23uv vzorec pro faktori\'al: 
$$n!=\sqrt {2\pi n}\big(\frac ne\big)^n(1+\frac 1{12n}+O(\frac 1{
n^2})).$$
Proto\v ze pro $n\ge 1$ je $\frac 1{12n},\frac 1{n^2}\ge 0$, m\accent23u\v zeme 
p\v redpokl\'adat, \v ze $(1+\frac 1{12n}+O(\frac 1{n^2}))\ge 1$ pro v\v sechna $
n\ge 1$. Po 
zlogaritmov\'an\'\i\ vzorce dost\'av\'ame 
$$\log_2n!\ge\frac 12\log_2n+n(\log_2n-\log_2e)+\log_2\sqrt {2\pi}
\ge (n+\frac 12)\log_2n-n\log_2e.$$
Proto\v ze  
$$e^1=e=2^{\log_2e}=(e^{\ln2})^{\log_2e}=e^{\ln2\log_2e},$$
plat\'\i , \v ze $\frac 1{\ln2}=\log_2e$, a tedy  
$$S(n)\ge\log_2n!\ge (n+\frac 12)\log_2n-\frac n{\ln2}.$$
\medskip

\flushpar D\'ale pro bin\'arn\'\i\ strom $T$ ozna\v cme $B(T)$ sou\v cet v\v sech 
d\'elek cest z ko\v rene do list\accent23u a polo\v zme 
$$B(k)=\min\{B(T)\mid T\text{\rm\ je bin\'arn\'\i\ strom s $k$ listy}
\}.$$
Kdy\v z uk\'a\v zeme, \v ze $B(k)\ge k\log_2k$, pak bude 
$$A(n)\ge\frac {B(n!)}{n!}\ge\frac {n!\log_2n!}{n!}=\log_2n!\ge (
n+\frac 12)\log_2n-\frac n{\ln2}.$$
Doka\v zme tedy, \v ze $B(T)\ge k\log_2k$ pro ka\v zd\'y bin\'arn\'\i\ 
strom $T$ s $k$ listy. Kdy\v z ve strom\v e $T$ vynech\'ame ka\v zd\'y 
vrchol, kter\'y m\'a jen jednoho syna, a tohoto syna spoj\'\i me 
s jeho p\v redch\accent23udcem, dostaneme \'upln\'y bin\'arn\'\i\ 
strom $T'$ s $k$ listy tako\-v\'y, \v ze $B(T')\le B(T)$. Proto se sta\v c\'\i\  
omezit na \'upln\'e bin\'arn\'\i\ stromy. Kdy\v z $T$ je \'upln\'y 
bin\'arn\'\i\ strom s jedn\'\i m listem, pak $B(T)=0=1\log_21$, 
kdy\v z $T$ je \'upln\'y bin\'arn\'\i\ strom se dv\v ema listy, pak 
$B(T)=2=2\log_22$. Tedy plat\'\i\ $B(1)\ge 1\log_21$ a $B(2)\ge 2\log_
22$. 
P\v redpokl\'adejme, \v ze $B(i)\ge i\log_2i$ pro $i<k$, a nech\v t $
T$ je 
\'upln\'y bin\'arn\'\i\ strom s $k$ listy. Nech\v t $T_1$ a $T_2$ jsou 
podstromy ur\v cen\'e syny ko\v rene a nech\v t $T_i$ m\'a $k_i$ 
list\accent23u, kde $i=1,2$. Pak $1\le k_1,k_2$ a $k_1+k_2=k$, tedy 
$k_1,k_2<k$ a podle induk\v cn\'\i ho p\v redpokladu $B(k_i)\ge k_
i\log_2k_i$. 
Odtud 
$$B(T)=k_1+B(T_1)+k_2+B(T_2)\ge k+B(k_1)+B(k_2)\ge k+k_1\log_2k_1
+k_2\log_2k_2.$$
\medskip

\flushpar Tedy sta\v c\'\i\ uk\'azat, \v ze 
$$k+k_1\log_2k_1+k_2\log_2k_2\ge k\log_2k$$
pro v\v sechna $k_1,k_2>0$ takov\'a, \v ze $k=k_1+k_2$. To je 
ekvivalentn\'\i\ s tvrzen\'\i m, \v ze pro $k>0$ plat\'\i\ 
$$f(x)=x\log_2x+(k-x)\log_2(k-x)+k-k\log_2k\ge 0,$$
kde $x\in (0,k)$. 
Abychom to dok\'azali, v\v simn\v eme si, \v ze $f(\frac k2)=0$ a  
po\v c\'\i\-tejme derivaci $f$.
$$f'(x)=\log_2x+\log_2e-\log_2(k-x)-\log_2e=\log_2\frac x{k-x}.$$
Nyn\'\i\ kdy\v z $x\in (0,\frac k2)$, pak $f'(x)<0$ a $f$ je na tomto intervalu 
klesaj\'\i c\'\i , kdy\v z $x\in (\frac k2,k)$, pak $f'(x)>0$ a $
f$ je na tomto 
intervalu rostouc\'\i . Odtud plyne, \v ze $f(x)\ge 0$ pro $x\in 
(0,k)$.
T\'\i m jsme dok\'azali, \v ze $A(n)\ge (n+\frac 12)\log_2n-\frac 
n{\ln2}$. Shrneme 
na\v se v\'ysledky.
\medskip

\proclaim{V\v eta}Ka\v zd\'y t\v r\'\i dic\'\i\ algoritmus, jeho\v z jedinou 
primitivn\'\i\ ope\-rac\'\i\ s prvky vstupn\'\i\ posloupnosti je 
porovn\'an\'\i , vy\v zaduje v nejhor\v s\'\i m i v o\v cek\'avan\'em p\v r\'\i pad\v e 
alespo\v n $cn\log n$ \v casu pro n\v ejakou konstantu $c>0$.  V 
nejhor\v s\'\i m p\v r\'\i pad\v e pou\v zije alespo\v n $\lceil 
(n+\frac 12)\log_2n-\frac n{\ln2}\rceil$ 
porovn\'an\'\i\ a o\v cek\'avan\'y po\v cet porovn\'an\'\i\ p\v ri rovnom\v ern\'em 
rozd\v elen\'\i\ vstupn\'\i ch posloupnost\'\i\ je alespo\v n 
$(n+\frac 12)\log_2n-\frac n{\ln2}$.  
\endproclaim
\medskip

\flushpar Tato v\v eta 
plat\'\i\ i pro \v sir\v s\'\i\ t\v r\'\i du primitivn\'\i ch operac\'\i , proto v n\'\i\ 
lze oslabit p\v redpklady. 
Doln\'\i\ odhad (v nejhor\v s\'\i m i pr\accent23um\v ern\'em p\v r\'\i pad\v e) 
bude platit i za 
p\v red\-pokladu, \v ze t\v r\'\i dic\'\i\ algoritmus nepou\v z\'\i v\'a nep\v r\'\i m\'e 
adreso\-v\'an\'\i\ a celo\-\v c\'\i seln\'e d\v elen\'\i . 
(Na druh\'e stran\v e n\'asleduj\'\i c\'\i\ klasick\'y algoritmus 
{\bf BUCKETSORT} ukazuje, \v ze p\v red\-poklady
ve v\v et\v e nelze zcela vynechat.) 
Tato metoda pro 
nalezen\'\i\ doln\'\i ho odhadu se pou\v z\'\i v\'a i pro vy\v c\'\i slov\'an\'\i\ 
algebraick\'ych funkc\'\i\ a p\v ri algoritmick\'em \v re\v sen\'\i\ 
geometrick\'ych \'uloh.
\bigskip

\centerline{\bigbigrm{IX. P\v rihr\'adkov\'e t\v r\'\i d\v en\'\i}}
\bigskip


\flushpar V n\'asleduj\'\i c\'\i ch algoritmech 
p\v red\-po\-kl\'a\-d\'ame, \v ze $Q_i$ jsou spojov\'e seznamy, nov\'y 
prvek se vkl\'ad\'a na konec seznamu a konkatenace 
seznam\accent23u z\'avis\'\i\ na jejich po\v rad\'\i . V 
seznamech m\'ame okam\v zit\'y p\v r\'\i stup k prvn\'\i mu a posledn\'\i mu 
prvku (pomoc\'\i\ ukazatel\accent23u na tyto prvky). 
Algoritmus {\bf BUCKETSORT} t\v r\'\i d\'\i\ posloupnost p\v rirozen\'ych 
\v c\'\i sel $a_1,a_2,\dots,a_n$ z intervalu $<0,m>$.
\bigskip

{\bf BUCKETSORT$(a_1,a_2,\dots,a_n,m)$}:\newline 
{\bf for every} $i=0,1,\dots,m$ {\bf do} $Q_i=\emptyset$ {\bf enddo\newline 
for every} $i=1,2,\dots,n$ {\bf do\newline 
\phantom{{\rm ---}}$a_i$} vlo\v z na konec seznamu $Q_{a_i}$\newline 
{\bf enddo\newline 
$i:=0$}, $P:=\emptyset$\newline 
{\bf while} $i\le m$ {\bf do\newline 
\phantom{{\rm ---}}$P:=$}konkatenace $P$ a $Q_i$, $i:=i+1$\newline 
{\bf enddo\newline 
V\'ystup: $P$} je neklesaj\'\i c\'\i\ posloupnost prvk\accent23u 
$a_1,a_2,\dots,a_n$
\bigskip

\flushpar Algoritmus nevy\v zaduje, aby prvky ve 
vstupn\'\i\ posloupnosti by\-ly r\accent23uzn\'e. Ve v\'ystupn\'\i\ 
posloupnosti se dan\'y prvek opakuje tolikr\'at, kolikr\'at se 
opakoval ve vstupn\'\i\ posloupnosti, se zachov\'an\'\i m po\v rad\'\i\ 
(tj. t\v r\'\i d\v en\'\i\ je stabiln\'\i ). 
Konkatenace dvou seznam\accent23u a vlo\v zen\'\i\ prvku do seznamu 
vy\v zaduj\'\i\ \v cas $O(1)$. Proto prvn\'\i\ a t\v ret\'\i\ cyklus vy\v zaduj\'\i\ 
\v cas $O(m)$ a druh\'y cyklus \v cas $O(n)$. Celkem 
algoritmus vy\v zaduje $O(n+m)$ \v casu a pam\v eti. Z\v rejm\v e kdy\v z  
$m=O(n)$, tak pro tento algoritmus neplat\'\i\ tvrzen\'\i\ v\v ety z 
p\v redchoz\'\i ho odstavce. 
D\accent23uvodem je, 
\v ze nejsou spln\v eny p\v redpoklady, proto\v ze druh\'y cyklus 
pou\v z\'\i v\'a nep\v r\'\i m\'e adresov\'an\'\i .
\medskip

\flushpar Nyn\'\i\ uvedeme dv\v e sofistikovan\v ej\v s\'\i\ verze tohoto 
algoritmu. V prvn\'\i\ p\v redpokl\'ad\'ame, \v ze 
$a_1,a_2,\dots,a_n$ je posloupnost navz\'ajem r\accent23uzn\'ych re\'aln\'ych 
\v c\'\i sel z intervalu $<0,1>$ a $\alpha$ je pevn\v e zvolen\'e kladn\'e 
re\'aln\'e \v c\'\i slo.
\medskip

{\bf HYBRIDSORT$(a_1,a_2,\dots,a_n)$}:\newline 
$k:=\alpha n$\newline 
{\bf for every} $i=0,1,\dots,k$ {\bf do} $Q_i=\emptyset$ {\bf enddo\newline 
for every} $i=1,2,\dots,n$ {\bf do\newline 
\phantom{{\rm ---}}$a_i$} vlo\v z na konec seznamu $Q_{\lceil ka_
i\rceil}$\newline 
{\bf enddo\newline 
$i:=0$}, $P:=\emptyset$\newline 
{\bf while} $i\le k$ {\bf do\newline 
\phantom{{\rm ---}}HEAPSORT$(Q_i)$
$P:=$}konkatenace $P$ a $Q_i$, $i:=i+1$\newline 
{\bf enddo\newline 
V\'ystup: $P$} je rostouc\'\i\ posloupnost prvk\accent23u 
$a_1,a_2,\dots,a_n$
\bigskip

\proclaim{V\v eta}Algoritmus {\bf HYBRIDSORT} set\v r\'\i d\'\i\ 
posloupnost re\'al\-n\'ych \v c\'\i sel z intervalu $<0,1>$ v 
nejhor\v s\'\i m p\v r\'\i pad\v e v \v case $O(n\log n)$. Kdy\v z prvky $
a_i$ 
maj\'\i\ rovnom\v ern\'e rozlo\v zen\'\i\ a jsou na sob\v e nez\'a\-visl\'e, 
pak o\v cek\'avan\'y \v cas je $O(n)$.
\endproclaim
\medskip

\demo{D\accent23ukaz}Prvn\'\i\ dva cykly v algoritmu 
vy\v zaduj\'\i\ \v cas $O(n)$, $i$-t\'y b\v eh t\v ret\'\i ho cyklu vy\v zaduje 
nejv\'y\v se \v cas $O(1+|Q_i|\log|Q_i|)$. Proto \v cas cel\'eho t\v ret\'\i ho cyklu je 
$$O(\sum_{i=0}^k(1+|Q_i|\log|Q_i|)=O(\sum_{i=0}^k(1+|Q_i|\log n)=
O(k+(\sum_{i=0}^k|Q_i|)\log n)=O(n\log n)$$
a celkov\'y \v cas {\bf HYBRIDSORTU} v nejhor\v s\'\i m p\v r\'\i pad\v e je nejv\'y\v se $
O(n\log n)$.
\medskip

\flushpar Nyn\'\i\ odhadneme o\v cek\'avan\'y \v cas. Polo\v zme 
$X_i=|Q_i|$. Pak 
$X_i$ je n\'a\-hodn\'a prom\v enn\'a a 
proto\v ze pravd\v epodobnost, \v ze $x\in Q_i$, je $\frac 1k$, dost\'a\-v\'ame, \v ze 
$$\Prob(X_i=q)=\binom nq(\frac 1k)^q(1-\frac 1k)^{n-q}.$$
O\v cek\'avan\'y \v cas vy\v zadovan\'y t\v ret\'\i m cyklem se pak rovn\'a 
$$E(\sum_{i=0}^k1+X_i\log X_i)\le k+k\sum_{q=2}^nq^2\binom nq(\frac 
1k)^q(1-\frac 1k)^{n-q}=k+k(\frac {n(n-1)}{k^2}+\frac nk)=O(n),$$
proto\v ze $k=\alpha n$ a 
$$q^2\binom nq=(q(q-1)+q)\binom nq=n(n-1)\binom {n-2}{q-2}+n\binom {
n-1}{q-1}.$$
(Jedn\'a se vlastn\v e o zn\'am\'y v\'ypo\v cet 2. momentu 
binomick\'eho rozd\v elen\'\i ). \qed
\enddemo
\medskip

\flushpar Pozn\'amka: V d\accent23ukazu jsme pou\v zili odhad 
$q\log q\le q^2$ a d\accent23usledkem toho je, \v ze jsme 
dok\'azali, \v ze o\v cek\'avan\'a slo\v zitost {\bf HYBRIDSORTU }
z\accent23ustane line\'arn\'\i , i kdybychom v n\v em m\'\i sto 
{\bf HEAPSORTU} pou\v zili n\v ejak\'y t\v r\'\i dic\'\i\ algoritmus s 
kvadratickou slo\v zitost\'\i , nap\v r. {\bf INSERTIONSORT}.
\medskip

\flushpar Nyn\'\i\ pou\v zijeme modifikaci {\bf BUCKETSORTU} pro 
t\v r\'\i d\v en\'\i\ slov.  M\'ame tot\'aln\v e 
uspo\v r\'adanou abecedu a chceme lexikograficky set\v r\'\i dit slova 
$a_1,a_2,\dots,a_n$ nad touto abecedou. P\v ripo\-me\v n\-me, \v ze kdy\v z 
$a=x_1x_2\dots x_n$ a $b=y_1y_2\dots y_m$ jsou dv\v e slova nad tot\'aln\v e 
uspo\v r\'a\-danou abecedou $\Sigma$, pak $a<b$ v lexikografick\'em 
uspo\v r\'ad\'an\'\i , pr\'av\v e kdy\v z existuje $i=0,1,\dots,\min
\{n,m\}$ takov\'e, \v ze 
$x_j=y_j$  pro ka\v zd\'e $j=1,2,\dots,i$ a bu\v d $n=i<m$ nebo $
i<\min\{n,m\}$ 
a $x_{i+1}<y_{i+1}$. P\v redpokl\'adejme, \v ze $a_i=a_i^1a_i^2\dots 
a_i^{l(i)}$, kde 
$a_i^j\in\Sigma$ a $l(i)$ je d\'elka $i$-t\'eho slova $a_i$.
\bigskip

{\bf WORDSORT$(a_1,a_2,\dots,a_n)$}:\newline 
{\bf for every} $i=1,2,\dots,n$ {\bf do} $l(i):=$d\'elka slova $a_
i$ {\bf enddo\newline 
$l=\max\{l(i)\mid i=1,2,\dots,n\}$\newline 
for every} $i=1,2,\dots,l$ {\bf do} $L_i=\emptyset$ {\bf enddo\newline 
for every} $i=1,2,\dots,n$ {\bf do\newline 
\phantom{{\rm---}}$a_i$} vlo\v z do $L_{l(i)}$\newline 
{\bf enddo\newline}
Koment\'a\v r: Pro ka\v zd\'e $i$ obsahuje $L_i$ v\v sechna slova z mno\v ziny 
$\{a_1,a_2,\dots,a_n\}$ d\'elky $i$.\newline 
$P_{}:=\{(j,a_i^j)\mid 1\le i\le n,\,1\le j\le l(i)\}$\newline 
$P_1:=${\bf BUCKETSORT$(P)$} podle druh\'e komponenty\newline 
$P_2:=${\bf BUCKETSORT$(P_1)$} podle prvn\'\i\ komponenty\newline 
{\bf for every} $i=1,2,\dots,l$ {\bf do} $S_i=\emptyset$ {\bf enddo\newline 
$(i,x):=$}prvn\'\i\ prvek $P_2$\newline 
{\bf while} $(i,x)\ne NIL$ {\bf do\newline 
\phantom{{\rm---}}$(i,x)$} vlo\v z do $S_i$\newline 
\phantom{---}{\bf while}$(i,x)=$n\'asledn\'\i k $(i,x)$ v $P_2$ {\bf do}\newline
\phantom{{\rm------}}$(i,x):=$n\'asledn\'\i k $(i,x)$ v $P_2$\newline 
\phantom{{\rm---}}{\bf enddo}\newline
\phantom{---}$(i,x):=$n\'asledn\'\i k $(i,x)$ v $P_2$\newline 
{\bf enddo}\newline
Koment\'a\v r: V $S_i$ jsou v\v sechny dvojice $(i,x)$ takov\'e, \v ze $
x$ 
je $i$-t\'ym p\'\i smenem n\v ekter\'eho vstupn\'\i ho slova a kdy\v z $
x<y$, pak 
$(i,x)$ je p\v red $(i,y)$.\newline 
{\bf for every} $s\in\Sigma$ {\bf do} $T_s:=\emptyset$ {\bf enddo\newline 
$T:=\emptyset$}, $i:=l$\newline 
{\bf while} $i>0$ {\bf do\newline 
\phantom{{\rm---}}$T:=$} konkatenace $L_i$ a $T$, $a:=$prvn\'\i\ slovo v $
T$\newline 
\phantom{---}{\bf while} $a\ne NIL$ {\bf do\newline 
\phantom{{\rm------}}$s:=i$}-t\'e p\'\i smeno $a$, vlo\v z $a$ do $
T_s$\newline 
\phantom{------}$a:=$n\'asledn\'\i k $a$ v $T$\newline 
\phantom{---}{\bf enddo\newline 
\phantom{{\rm ---}}$(i,x):=$}prvn\'\i\ prvek v $S_i$, $T:=\emptyset$\newline 
\phantom{---}{\bf while} $(i,x)\ne NIL$ {\bf do\newline 
\phantom{{\rm------}}$T:=$} konkatenace $T$ a $T_x$, $T_x:=\emptyset$\newline 
\phantom{------}$(i,x):=$n\'asledn\'\i k $(i,x)$ v $S_i$\newline 
\phantom{---}{\bf enddo\newline 
\phantom{{\rm ---}}$i:=i-1$\newline 
enddo\newline 
V\'ystup: $T$} je set\v r\'\i d\v en\'a posloupnost slov $a_1,a_2
,\dots,a_n$
\bigskip

\flushpar Uva\v zujme jeden b\v eh posledn\'\i ho cyklu algortimu 
pro ur\v cit\'e $i$. Po 
jeho skon\v cen\'\i\ jsou v $T$ v\v sechna slova z mno\v ziny 
$a_1,a_2,\dots,a_n$, kter\'a maj\'\i\ d\'elku alespo\v n $i$, a kdy\v z slovo 
$a_r$ je p\v red $a_q$ v seznamu $T$, pak existuje $j=i-1,i,\dots
,l$
takov\'e, \v ze $a^k_r=a^k_q$ pro ka\v zd\'e $k=i,i+1,\dots,j$ a bu\v d 
$l(r)=j\le l(q)$ nebo $j<\min\{l(r),l(q)\}$ a $a_r^{j+1}<a_q^{j+1}$. To plyne z 
vlastnost\'\i\ algoritmu {\bf BUCKETSORT} indukc\'\i\ podle $i$. Jedin\'y a 
hlavn\'\i\ rozd\'\i l proti {\bf BUCKETSORTU} je, \v ze neproch\'az\'\i me 
v\v sechny p\v rihr\'adky $T_x$, ale pouze nepr\'azdn\'e. 
To n\'am zaji\v s\v tuje mno\v zina $S_i$ (viz Koment\'a\v r). 
\medskip

\flushpar Ozna\v cme $L=\sum_{i=1}^nl(i)$ a p\v ripome\v nme, \v ze 
$l=\max\{l(i)\mid i=1,2,\dots,n\}$. Pak prvn\'\i\ cyklus (v\'ypo\v cet  
d\'elek slov) vy\-\v zaduje \v cas $O(L)$. Druh\'y cyklus (inicializace 
seznam\accent23u $L_i$) vy\v zaduje 
\v cas $O(l)=O(L)$ a t\v ret\'\i\ cyklus (za\v razen\'\i\ slov do $
L_i$ podle 
d\'elek) \v cas $O(n)=O(L)$. Vytvo\v ren\'\i\ 
seznamu $P_{}$ vy\v zaduje \v cas $O(L)$ a jeho set\v r\'\i d\v en\'\i\ podle obou 
komponent  
\v cas $O(L+l)=O(L)$, proto\v ze $P_{}$ i $P_1$ maj\'\i\ nejv\'y\v se $
L$ 
prvk\accent23u. Dal\v s\'\i\ cyklus (zalo\v zen\'\i\ seznam\accent23u 
$S_i$) vy\v zaduje \v cas $O(l)$ a n\'asleduj\'\i c\'\i\ cyklus vytv\'a\v rej\'\i c\'\i\ 
seznamy $S_i$ \v cas $O(L)$. Cyklus zakl\'adaj\'\i c\'\i\ 
seznamy $T_x$ vy\v zaduje \v cas $O(|\Sigma |)$. B\v ehy dal\v s\'\i ho cyklu jsou 
indexov\'any $i=1,2,\dots,l$. Pro ka\v zd\'e $i$ ozna\v cme $m_i$ po\v cet slov z 
mno\v ziny $\{a_1,a_2,\dots,a_n\}$, kter\'a maj\'\i\ d\'elku alespo\v n $
i$. Pak 
$L=\sum_{i=1}^lm_i$ a prvn\'\i\ vnit\v rn\'\i\ cyklus v $i$-t\'em b\v ehu vn\v ej\v s\'\i ho 
cyklu vy\v zaduje \v cas $O(m_i)$ a druh\'y vnit\v rn\'\i\ cyklus 
\v cas $O(|S_i|)=O(m_i)$. Tedy 
celkov\'y \v cas algoritmu je $O(L+m)$, kde $m=|\Sigma |$ a $L$ 
je sou\v cet d\'elek v\v sech slov z mno\v ziny $a_1,a_2,\dots,a_
n$.
\bigskip

\centerline{\bigbigrm{X. Po\v r\'adkov\'e statistiky}}
\bigskip

\flushpar Na z\'av\v er pop\'\i\v seme dva algoritmy pro hled\'an\'\i\ $
k$-t\'eho 
nejmen\v s\'\i\-ho prvku v dan\'e podmno\v zin\v e tot\'aln\v e 
uspo\v r\'adan\'e\-ho univerza. Prv\-n\'\i\ z nich vyu\v z\'\i v\'a  
stejn\'y princip jako {\bf QUICKSORT}. Nejprve zad\'ame 
p\v resn\'e zn\v e\-n\'\i\ na\v s\'\i\ \'ulohy (\'uloha i algoritmy se daj\'\i\ 
snadno p\v reformulovat pro p\v r\'\i pad, kdy hled\'ame $k-$t\'y 
nejv\v et\v s\'\i\ prvek).
\medskip

\flushpar Pracujeme s tot\'aln\v e uspo\v r\'adan\'ym univerzem $
U$.\newline 
Vstup: mno\v zina prvk\accent23u $M=\{a_1,a_2,\dots,a_n\}\subseteq 
U$ a \v c\'\i slo $i$ takov\'e, 
\v ze $1\le i\le n$.\newline 
V\'ystup: prvek $a_k$ takov\'y, \v ze 
$|\{j\mid 1\le j\le n,\,a_j\le a_k\}|=i$.\newline 
Kdy\v z $i=\frac n2$, pak $a_k$ se naz\'yv\'a \emph{medi\'an}.
\bigskip

{\bf FIND$(M=(a_1,a_2,\dots,a_n),i)$}:\newline 
zvol $a\in M$\newline
$M_1:=\{b\in M\mid b<a\}$, 
$M_2:=\{b\in M\mid b>a\}$\newline 
{\bf if} $|M_1|>i-1$ {\bf then\newline 
\phantom{{\rm ---}}FIND$(M_1,i)$\newline 
else\newline 
\phantom{{\rm ---}}if} $|M_1|<i-1$ {\bf then\newline 
\phantom{{\rm ------}}FIND$(M_2,i-|M_1|-1)$\newline 
\phantom{{\rm ---}}else\newline 
\phantom{{\rm------}}V\'ystup: $a$} je hledan\'y prvek\newline 
\phantom{---}{\bf endif\newline 
endif}
\bigskip

\flushpar D\accent23ukaz korektnosti algoritmu je zalo\v zen na n\'asleduj\'\i c\'\i m 
jednoduch\'em pozorov\'an\'\i : m\v ejme mno\v zinu $M$ a prvek $
x$ a 
polo\v zme $M_1=\{m\in M\mid m<x\}$. Kdy\v z $k\le |M_1|$, pak $k$-t\'y 
nejmen\v s\'\i\ 
prvek v $M_1$ je stejn\'y jako $k$-t\'y nejmen\v s\'\i\ prvek v $
M$. Kdy\v z 
$k>|M_1|$, pak $(k-|M_1|)$-t\'y nejmen\v s\'\i\ prvek v $M\setminus 
M_1$ je $k$-t\'y 
nejmen\v s\'\i\ prvek v $M$.  
Zb\'yv\'a vy\v set\v rit slo\v zitost.
\medskip
\flushpar V nejhor\v s\'\i m p\v r\'\i pad\v e 
vol\'ame {\bf FIND} $n$-kr\'at a jedno vol\'an\'\i\ vy\v zaduje \v cas $
O(|M|)$.  
Tedy \v casov\'a slo\v zitost algoritmu 
{\bf FIND} v nejhor\v s\'\i m p\v r\'\i pad\v e je $O(n^2)$.  Dobr\'e volby 
prvku $a$ mohou algoritmus zna\v cn\v e zrychlit.  V tomto 
p\v r\'\i pad\v e plat\'\i\ stejn\'a 
diskuse jako pro {\bf QUICKSORT}.  Spo\v c\'\i t\'ame o\v cek\'avan\'y \v cas 
za p\v redpokladu, \v ze  
pr\-vek $a$ byl vybr\'an n\'ahodn\v e.  Pak pravd\v epodobnost, \v ze je 
$k$-t\'ym nejmen\v s\'\i m prvkem, je $\frac 1n$, kde $n=|M|$.  Ozna\v cme $
T(n,i)$ 
o\v cek\'avan\'y \v cas algoritmu {\bf FIND} pro nalezen\'\i\ $i$-t\'eho nejmen\v s\'\i ho 
prvku v $n$-prvkov\'e mno\v zin\v e $M$.  Plat\'\i\ 
$$T(n,i)=n+\frac 1n(\sum_{k=1}^{i-1}T(n-k,i-k)+\sum_{k=i+1}^nT(k,
i)),$$
proto\v ze procedura {\bf FIND} bez rekurzivn\'\i ho vol\'an\'\i\ sebe sama 
vy\v zaduje \v cas $O(n)$. P\v redpokl\'a\-dej\-me, \v ze $T(m,i)
\le 4m$ 
pro ka\v zd\'e $m<n$ a ka\v zd\'e $i$ takov\'e, \v ze $1\le i\le 
m$. Pak 
$$\align T(n,i)=&n+\frac 1n(\sum_{k=1}^{i-1}T(n-k,i-k)+\sum_{k=i+
1}^nT(k,i))\le n+\frac 1n(\sum_{k=1}^{i-1}4(n-k)+\sum_{k=i+1}^n4k
)=\\
&n+\frac 4n(\frac {(2n-i)(i-1)}2+\frac {(n+i+1)(n-i)}2)=n+\frac 4
n(\frac {n^2+2ni-n-2i^2}2).\endalign$$
V\'yraz v \v citateli zlomku nab\'yv\'a sv\'eho maxima pro 
$i=\frac n2$ a jeho maximaln\'\i\ hodnota je 
$\frac 32n^2-n=\frac {3n^2-2n}2$. Tedy
$$T(n,i)\le n+\frac 4n(\frac {3n^2-2n}4)=n+3n-2=4n-2<4n.$$
Proto\v ze tento odhad plat\'\i\ tak\'e pro $n=1$ a $n=2$, dok\'azali 
jsme indukc\'\i , \v ze $T(n,i)\le 4n$ pro v\v sechna $n$ a v\v sechna $
i$ 
takov\'a, \v ze $1\le i\le n$. Plat\'\i\ tedy 
\medskip

\proclaim{V\v eta}Algoritmus {\bf FIND} nalezne $i$-t\'y nejmen\v s\'\i\ 
prvek v $n$ prv\-kov\'e tot\'aln\v e uspo\v r\'adan\'e mno\-\v zi\-n\v e a v nejhor\v s\'\i m 
p\v r\'\i pad\v e vy\v za\-du\-je \v cas $O(n^2)$. Kdy\v z se pivot vol\'\i\ 
n\'ahodn\v e nebo kdy\v z v\v sechny vstupn\'\i\ mno\v ziny maj\'\i\ stejnou 
pravd\v epodobnost, pak o\v cek\'avan\'y \v cas je $O(n)$.
\endproclaim
\medskip

\flushpar Pro velmi mal\'a $i$ nebo pro $i$ velmi bl\'\i zk\'a $n$ 
pracuje rychleji p\v r\'\i m\'y p\v rirozen\'y algoritmus (udr\v zuje si 
posloupnost $i$ nejmen\-\v s\'\i ch nebo $n-i$ nejv\v et\v s\'\i ch prvk\accent23u 
a k n\'\i\ p\v rid\'av\'a dal\v s\'\i\ tak, \v ze ten prvek, kter\'y p\v re\-kro\v cil 
danou hranici, je 
zapomenut). Tento algoritmus v\v sak nen\'\i\ 
efektivn\'\i\ pro obecn\'a $i$. 
\medskip

\flushpar N\'asleduj\'\i c\'\i\ algoritmus nalezne $i$-t\'y nejmen\v s\'\i\ 
prvek v line\'arn\'\i m \v case i v nejhor\v s\'\i m p\v r\'\i pad\v e.  Vstupem 
je op\v et podmno\v zina $M$ tot\'aln\v e 
uspo\v r\'adan\'eho univerza $U$ a p\v rirozen\'e \v c\'\i slo $i$ takov\'e, \v ze 
$1\le i\le |M|$.  
\medskip

{\bf SELECT$(M,i)$}:\newline 
$n:=|M|$\newline 
{\bf if} $n\le 100$ {\bf then\newline 
\phantom{{\rm---}}}set\v ri\v d mno\v zinu $M$, $m:=$ $i$-t\'y nejmen\v s\'\i\ 
prvek $M$\newline 
{\bf else\newline 
\phantom{{\rm---}}}rozd\v el $M$ do navz\'ajem disjunktn\'\i ch 
p\v etiprvkov\'ych podmno\v zin $A_1,A_2,\dots,A_{\lceil\frac n5\rceil}$\newline 
\phantom{---}(posledn\'\i\ z podmno\v zin m\accent23u\v ze m\'\i t m\'en\v e ne\v z 5 prvk\accent23u).\newline 
\phantom{---}{\bf for every} $j=1,2,\dots,\lceil\frac n5\rceil$ {\bf do\newline 
\phantom{{\rm------}}}najdi medi\'an $m_j$ mno\v ziny $A_j$\newline 
\phantom{---}{\bf enddo\newline 
\phantom{{\rm ---}}$\bar {m}:=$SELECT$(\{m_j\mid j=1,2,\dots,\lceil\frac 
n5\rceil \},\lceil\frac n{10}\rceil )$\newline 
\phantom{{\rm ---}}$M_1:=\{m\in M\mid m<\bar {m}\}$}, $M_2:=\{m\in 
M\mid\bar {m}<m\}$\newline 
\phantom{---}{\bf if} $|M_1|>i-1$ {\bf then\newline 
\phantom{{\rm ------}}$m:=$SELECT$(M_1,i)$\newline 
\phantom{{\rm ---}}else\newline 
\phantom{{\rm------}}if} $|M_1|<i-1$ {\bf then\newline 
\phantom{{\rm ---------}}$m:=$SELECT$(M_2,i-|M_1|-1)$\newline 
\phantom{{\rm ------}}else\newline 
\phantom{{\rm ---------}}$m:=\bar {m}$\newline 
\phantom{{\rm ------}}endif\newline 
\phantom{{\rm ---}}endif\newline 
\phantom{{\rm ---}}V\'ystup}: $m$\newline 
{\bf endif
\medskip

}\flushpar D\accent23ukaz korektnosti algoritmu je stejn\'y 
jako u algoritmu {\bf FIND}. 
Zb\'yv\'a vy\v set\v rit slo\v zitost. Nej\-prve dok\'a\v zeme n\'asleduj\'\i c\'\i\ 
lemma.
\medskip

\proclaim{Lemma}Kdy\v z $n\ge 100$, pak $|M_1|,|M_2|\le\frac {8n}{
11}$.
\endproclaim

\demo{D\accent23ukaz} Pro $j\le\lfloor\frac n5\rfloor$ plat\'\i , \v ze kdy\v z $
m_j<\bar {m}$, pak 
$|A_j\cap M_1|\ge 3$, kdy\v z $m_j>\bar {m}$, pak $|A_j\cap M_2|\ge 
3$, kdy\v z $m_j=\bar {m}$, 
pak $|A_j\cap M_1|=|A_j\cap M_2|=2$.  Proto\v ze 
$|\{j=0,1,\dots,\lfloor\frac n5\rfloor\mid m_j<\bar {m}\}|,|\{j=0
,1,\dots,\lfloor\frac n5\rfloor\mid m_j>\bar {m}\}|\ge\lfloor\frac 
n{10}\rfloor$, 
dost\'av\'ame, \v ze $|M_1|,|M_2|\ge\lfloor\frac {3n}{10}\rfloor 
-1$.  D\'ale plat\'\i\ $M_1\cap M_2=\emptyset$, 
$M_1\cup M_2=M\setminus \{\bar {m}\}$ a proto\v ze $\frac {8n}{11}
+\lfloor\frac {3n}{10}\rfloor -1\ge\frac {113n}{110}-2\ge n$ 
kdy\v z $n>100$, dost\'av\'ame po\v zadovan\'y odhad. \qed
\enddemo

\flushpar Maxim\'aln\'\i\ \v cas vy\v zadovan\'y algoritmem  
{\bf SELECT$(M,i)$} pro $|M|=n$ ozna\v cme $T(n)$.  Kdy\v z $n\le 
100$, pak z\v rejm\v e 
exis\-tuje konstanta $a$ takov\'a, \v ze $T(n)\le an$.  Kdy\v z $
n>100$, 
pak $\lceil\frac n5\rceil\le\frac {21n}{100}$, a proto\v ze {\bf SELECT$
(M,i)$} pro $|M|>100$ bez 
rekurentn\'\i ch vol\'an\'\i\ vy\v zaduje \v cas $O(|M|)$, plat\'\i , \v ze $
T(n)\le T(\frac {21n}{100})+T(\frac {8n}{11})+bn$ pro n\v ejakou konstantu 
$b$.  Zvolme $c\ge\max\{a,\frac {1100b}{69}\}$. Uk\'a\v zeme, \v ze $
T(n)\le cn$ pro 
v\v sechna $n$.  
Kdy\v z $n\le 100$, tak tvrzen\'\i\ z\v rejm\v e plat\'\i , proto\v ze $
a\le c$.  Kdy\v z 
$n>100$, pak $\lceil\frac {21n}{100}\rceil ,\lceil\frac {8n}{11}\rceil 
<n$, a proto\v ze z volby $c$ plyne 
$b\le\frac {69}{1100}c$, 
dost\'av\'ame 
$$T(n)\le c\frac {21n}{100}+c\frac {8n}{11}+bn=(\frac {1031c}{110
0}+b)n\le cn.$$
Tedy
\medskip

\proclaim{V\v eta}Algoritmus {\bf SELECT} nalezne $i$-t\'y nejmen\v s\'\i\ 
prvek v line\'arn\'\i m \v case.
\endproclaim
\medskip

\flushpar Algoritmus {\bf FIND} je ve velk\'e v\v et\v sin\v e 
p\v r\'\i pad\accent23u 
rychlej\v s\'\i\ ne\v z algoritmus {\bf SELECT}, proto je v praxi 
doporu\v cov\'an, i kdy\v z existuj\'\i\ 
p\v r\'\i pady (velmi \v r\'\i dk\'e), kdy pot\v rebuje kvadratick\'y \v cas.
Je zn\'amo, \v ze medi\'an $n$-prvkov\'e mno\v ziny lze nal\'ezt s 
m\'en\v e ne\v z $3n$ porovn\'an\'\i mi a \v ze ka\v zd\'y algoritmus hledaj\'\i c\'\i\ 
medi\'an a pou\v z\'\i vaj\'\i c\'\i\ porovn\'an\'\i\ jako jedinou primitivn\'\i\ 
operaci mezi prvky mno\v ziny vy\v zaduje v\'\i ce ne\v z $2n$ 
porovn\'an\'\i .
\bigskip

\head
Historick\'y p\v rehled
\endhead

\flushpar Algoritmus {\bf HEAPSORT }
navrhl v roce 1964 Williams a vylep\v sil Floyd (rovn\v e\v z 1964).  N\'avrh 
na pou\v zit\'\i\ $d$-regul\'ar\-n\'\i ch hald je folklor stejn\v e tak jako  
algoritmus {\bf MER\-GESORT}.  Algoritmy {\bf QUICKSORT} a {\bf FIND }
zavedl Hoare (1962).  Anal\'yza ope\-race {\bf MERGE} a 
hled\'an\'\i\ optim\'aln\'\i ho stro\-mu poch\'az\'\i\ od Huffmana 
(1952) a line\'arn\'\i\ implementaci algoritmu navrhl van 
Leeuwen (1976).  Anal\'yza rozhodovac\'\i ch strom\accent23u je 
folklor. Algoritmus {\bf HYBRIDSORT} navrhli Meijer a Akl (1980), 
vylep\v se\-n\'a verze {\bf BUCKETSORTU} (nazvan\'a {\bf WORDSORT}) 
poch\'az\'\i\ od Aho, Hopcrofta a Ullmana (1974). Algoritmus {\bf SELECT} byl 
navr\v zen Blumem, Floydem, Prattem, Rivestem a 
Tarjanem (1972).  

\end{document}
