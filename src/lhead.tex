%%%%%%%%%%%%%%%%%%%%%%%%%%%%%%%%%%%%%%%%%%%%%%%%%%%%%%%%
%%%%%%%%%%%%%%%%%%%%%%%%%%%%%%%%%%%%%%%%%%%%%%%%%%%%%%%%
% Following are two definitions of \LPic.  The first definition
% is the standard Leo definition of \LPic.  The second definition
% can be used in conjunction with macros provided by Springer Verlag for 
% publishing books through submission of a \TeX\ document.
% You can rewrite \LPic for your own purposes.
%
% The first argument is the horizontal dimension of the picture in 
% centimeters.
% The second argument is the vertical dimension of the picture in
% centimeters.
% The third argument is the Postscript file name.
% The fourth argument is the Figure number.
% The fifth argument is the Figure caption.
% The sixth argument are the text labels in the picture.
%%%%%%%%%%%%%%%%%%%%%%%%%%%%%%%%%%%%%%%%%%%%%%%%%%%%%%%%
\def\LPic #1 #2 #3 #4#5#6{%
\hbox to #1cm{\vbox{%
\hbox to #1cm{\vbox to#2cm{%
\offinterlineskip\vfill\hbox to #1cm{%
\hbox{}\ignorespaces#6%
\hbox{}\includegraphics{#3}%
\hfil}%
}\hfil}%
\hbox to #1cm{\hfil#4 #5\hfil}}\hfil}}
%%%%%%%%%%%%
%%%%%%%%%%%% for Springer Verlag
%\def\LPic #1 #2 #3 #4#5#6{%
%\begfig #2cm%
%\Lpic #1 #2 #3 {#6}%
%\figure{#4}{#5}%
%\endfig
%}
%%%%%%%%%%%
%%%%%%%%%%% Used in definition of \LPic
\def\Lpic #1 #2 #3 #4{%
\vbox to 0cm%
{\offinterlineskip\vfill\hbox to\hsize{\hfill\hbox to#1cm{%
\hbox{}\hbox to0pt{\includegraphics{#3}}\hbox to0pt{\ignorespaces#4}%
\hfill}\hfill}}}
%%%%%%%%%%%
% These macros allow text labels to be drawn at given coordinates in a 
% picture.  We use "bp" (big points) to match the dvips default scale
% of 72 units per inch.
\def\atxy #1 #2 #3 {
\count1=#1
\count2=#2
\advance\count1 by -10
\advance\count2 by -10
\smash{\rlap{\kern\count1 bp\raise\count2 bp\hbox to0pt{#3}}}}
\def\LText#1%
{{\vbox to 0pt{\vss\vbox{\hbox to0pt{#1\hss}\vbox{}}\vss}}}
\def\RText#1%
{\smash{\vbox to0pt{\vss\vbox{\hbox to0pt{\hss#1}\vbox{}}\vss}}}
\def\CText#1{\vbox to0pt{\vss{\hbox to0pt{\hss#1\hss}}\vss}}
\def\BText#1%
{\smash{\vbox to0pt{\vbox{\hbox to0pt{\hss#1\hss}\vbox{}}\vss}}}
\def\AText#1%
{\smash{\vbox to0pt{\vss\vbox{\hbox to0pt{\hss#1\hss}\vbox{}}}}}
%%%%%%%%%%%%
% Table macros
\def\LTable#1{\vbox{\tabskip=0pt\offinterlineskip\halign{#1}}}
\def\LTt{&\strut&}
\def\LWt{&\strut\vrule&}
\def\LHl{\noalign{\hrule}}
\def\LCc#1{\strut\hfil\quad#1\quad\hfil}
\def\LLc#1{\strut\quad#1\hfil}
\def\LRc#1{\strut\hfil#1\quad}
\def\LLFc#1{\strut#1\hfil}
\def\LRFc#1{\strut\hfil#1}
\def\LSp#1{\multispan{#1}}

